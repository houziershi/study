
\chapter{基本套接字}

	\section{套接字地址}

		\subsection{例子}

		\lstinputlisting[]{src/ch02/InetAddressExample.java}

		运行与输出结果:

\begin{lstlisting}[language=Bash]
% java InetAddressExample \
	www.baidu.com www.jadehahaha.com \
	129.35.69.7

Interface eth0:
	Address (v6): fe80:0:0:0:225:b3ff:fe69:1659%2
	Address (v4): 172.16.7.1
Interface lo:
	Address (v6): 0:0:0:0:0:0:0:1%1
	Address (v4): 127.0.0.1
www.baidu.com: 
	www.baidu.com/220.181.111.147
www.jadehahaha.com: 
	Unable to find address for : www.jadehahaha.com
129.35.69.7: 
	129.35.69.7/129.35.69.7
\end{lstlisting}

		\subsection{InetAddress类API简介}

			InetAddress类中创建与访问实例方法:
\begin{lstlisting}
static InetAddress [] getAllByName(String host);
static InetAddress getByName(String host);
static InetAddress getLocalHost();
/**
 * return byte array as ip address
 * size if 4 byte in ipv4
 * or size is 16 in ipv6
 */
byte[] getAddress();
\end{lstlisting}

			InetAddress类中字符串显示方法:
\begin{lstlisting}
/**
 * return value format like:
 *   hostname.example.com/192.0.2.127
 *   or
 *   never.example.net/2000::620:1a30:95b2
 */
String toString();
String getHostAddress();
/**
 * return host name
 */
String getHostName();
/**
 * return host name ()
 */
String getCanonicalHostName();
\end{lstlisting}

			InetAddress类中检查属性的方法:
\begin{lstlisting}
boolean isAnyLocalAddress();
boolean isLinkLocalAddress();
boolean isLoopBackAddress();
// 是否为多播地址
boolean isMulticastAddress();
// 测试多播地址范围:global
boolean isMCGlobal();
// 测试多播地址范围:link local
boolean isMCLinkLocal();
// 测试多播地址范围:node local
boolean isMCNodeLocal();
// 测试多播地址范围:org local
boolean isMCOrgLocal();
// 测试多播地址范围:site local
boolean isMCSiteLocal();
boolean isReachable(int timeout);
boolean isReachable(NetworkInterface netif, int ttl, int timeout)
\end{lstlisting}


	\section{TCP套接字}
