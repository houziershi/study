\chapter{Perl}

\section{元字符}

\subsection{空白字符}

\begin{table}[htbp]
	\caption{Perl正则元字符}
	\label{tab:perl.regex.meta}
	\centering
	\begin{tabular}{ll}
		\hline
		元字符 & 作用 \\
		\hline
		\verb|\t| & 制表符 \\
		\verb|\n| & 换行 \\
		\verb|\b| & 一般情况表示下单词分界,但在字符范围中表示退格。\\
		\verb|\s| & 空白字符(包括空格、制表符、换行)\\
		\verb|\S| & 除了“\verb|\s|”以外的任何字符\\
		\verb|\w| & “\verb|[A-Za-z0-9]|”\\
		\verb|\W| & 除了“\verb|\W|”以外的任何字符\\
		\verb|\d| & “\verb|[0-9]|”\\
		\verb|\D| & 除了“\verb|\d|”以外的任何字符\\
		\hline
	\end{tabular}
\end{table}



\section{基本使用}

查找文件中接连重复出现的单词

\lstinputlisting[language=Perl]{src/ch02/exp01.pl}

\subsection{变量的声明与引用}

普通变量以美元符开头,而且可以在输出语句中直接使用。

以下是一个转摄氏度为华氏度的例子:

\lstinputlisting[language=Perl]{src/ch02/exp02.pl}

\subsection{控制结构}

\lstinputlisting[language=Perl]{src/ch02/exp03.pl}

运行时可以通过参数“\verb|-w|”打开编译警告:

\begin{lstlisting}[language=bash]
perl -w exp03.pl
\end{lstlisting}


\section{用正则匹配文本}

“\verb|=~|”指定正则操作的对象。

“\verb|m/.../|”表示通过正则进行的操作是匹配操作(可以省略m,但是加上看起来更加清楚)。

“\verb|==|”用来比较两个数字是否相等。

“\verb|eq|”用来比较两个字符串是否相等。

查找是否是数字:

\lstinputlisting[language=Perl]{src/ch02/exp04.pl}



\section{取得用户输入}

增加能够处理小数部分;并通过函数“\verb|printf|”格式化输出。

\lstinputlisting[language=Perl]{src/ch02/exp05.pl}



\section{引用匹配的内容}

通过\verb|$num|取得匹配的表达式内容:

\lstinputlisting[language=Perl]{src/ch02/exp06.pl}

通过\verb|$(?:...)|只用来分组,但是不取得匹配内容:

\lstinputlisting[language=Perl]{src/ch02/exp07.pl}

回到温度转换的例子,根据用户输入最后是C还是F来判断输入的类型:

\lstinputlisting[language=Perl]{src/ch02/exp08.pl}



\section{修饰符}

\subsection{忽略大小写}

在正则以后加个修饰符“\verb|/i|”表示忽略大小写。

\begin{lstlisting}[language=Perl]
$input =~ m/^aaa$/i;
\end{lstlisting}

更加完整的温度转换例子:

\lstinputlisting[language=Perl]{src/ch02/exp08.pl}

\subsection{全局匹配}

修饰符“\verb|/g|”表示全局匹配。就是在完成了一次匹配以后,再继续匹配剩下的内容。

\begin{lstlisting}[language=Perl]
$input =~ m/^aaa$/g;
\end{lstlisting}

\subsection{宽松排列表达式}

修饰符“\verb|/x|”表示。

\begin{lstlisting}[language=Perl]
$input =~ m/^aaa$/x;
\end{lstlisting}



\section{替换文本}

“\verb|$var =~ s/regex/replacement/|”以变量“\verb|$var|”为对象,把符合正则的内容替换掉。

例:无视大小写,把“peter”替换成“Peter”。

\begin{lstlisting}[language=Perl]
$var =~ s/\bpeter\b/Peter/i;
print "$var";
\end{lstlisting}

\subsection{使用perl自动替换文本}

参数“-p”表示对目标文件每一行进行查找和替换;
参数“-i”表示替换的结果写回文件;
参数“-e”表示后面的字符串就是程序的代码;

\begin{lstlisting}[language=Perl]
perl -p -i -e 's/sysread/read/g' filename
\end{lstlisting}

可以合并参数为:

\begin{lstlisting}[language=Perl]
perl -pi -e 's/sysread/read/g' filename
\end{lstlisting}

\subsection{生成邮件回复的例子}

原始内容在文件“file.in”,通过程序“mkreply.pl”把结果存放在“file.out”。

file.in

\lstinputlisting[language=Perl]{src/ch02/file.in}

希望程序能自动生成回复的样式:

\lstinputlisting[language=Perl]{src/ch02/out.exp}

调用的方法:

\begin{lstlisting}[language=Perl]
perl -w mkreply.pl file.in > file.out
\end{lstlisting}

\section{从文件读取}

Perl提供了操作符“\verb|<>|”把每一行读取到变量中,我们通过“\verb|^\s*$|”检查空行(表示邮件head结束)。

\lstinputlisting[language=Perl,firstline=0,lastline=6]{src/ch02/exp11.pl}

从邮件头中提取信息的方法,以Subject为例:

\lstinputlisting[language=Perl,firstline=0,lastline=3]{src/ch02/exp10.pl}

分别取得回信地址和昵称:

\lstinputlisting[language=Perl,firstline=5,lastline=9]{src/ch02/exp10.pl}

就连输出原文引用的部人也可以通过正则来实现:

\lstinputlisting[language=Perl,firstline=13,lastline=14]{src/ch02/exp10.pl}

\section{增强锚点}

通常来说,锚点“\verb|^|”、“\verb|$|”匹配的不是逻辑行的开头与结尾,而是整个字符串的开头与结尾。如果要匹配逻辑行可以切换到增强锚点(enhanced line anchor)模式下。在Perl语言中修饰符为“\verb|/m|”。例如,把空白行替换为HTML的段落符“\verb|<p>|”:

\lstinputlisting[language=Perl,firstline=8,lastline=8]{src/ch02/exp11.pl}


\section{格式化}

为了增加可读性,修饰符“\verb|/x|”允许对正则表达式进行排版(还可以用花括号代替斜线):

\lstinputlisting[language=Perl,firstline=10,lastline=19]{src/ch02/exp11.pl}


\section{重用正则对象}

修饰符“\verb|qr/.../|”表示不应用到字符串,建立一个对象以后再用:

\lstinputlisting[language=Perl,firstline=21,lastline=25]{src/ch02/exp11.pl}

