\chapter{拓展}

\section{引用匹配的内容}

在Perl语言中通过\verb|$num|取得匹配的表达式内容:

\lstinputlisting[language=Perl]{src/ch02/exp06.pl}

通过\verb|$(?:...)|只用来分组,但是不取得匹配内容:

\lstinputlisting[language=Perl]{src/ch02/exp07.pl}

回到温度转换的例子,根据用户输入最后是C还是F来判断输入的类型:

\lstinputlisting[language=Perl]{src/ch02/exp08.pl}


\section{环视功能(lookaround)}

环视具体有以下四种:

顺序肯定环视“\verb|(?=...)|”:某个位置的右边符合子表达式。

顺序否定环视“\verb|(?!...)|”:某个位置的右边不符合子表达式。

逆序肯定环视“\verb|(?<=...)|”:某个位置的左边符合子表达式。

逆序否定环视“\verb|(?<!...)|”:某个位置的左边不符合子表达式。

\subsection{环视只匹配位置}

环视功能只匹配位置,而不匹配具体的字符(就像是行头“\verb|^|”、字符分界符“\verb|\b|”)。它匹配的是某一个位置前后的内容是否符合。

例如:表达式“\verb|Jeffrey|”匹配的是一串文本:

\begin{verbatim}
by Jeffrey Friedl.
   ^-----^
\end{verbatim}

而环视“\verb|(?=Jeffrey)|”匹配的的两个字符之间的位置:

\begin{verbatim}
by Jeffrey Friedl.
 -><-
\end{verbatim}

再看一个例子,“\verb|(?=Jeffery)Jeff|”和“\verb|Jeff(?=rey)|”是等价的:

“\verb|(?=Jeffery)Jeff|”:从“Jeffery”的开头位置开始找“Jeff”。

“\verb|Jeff(?=rey)|”:找到后面有“rey”的“Jeff”。



\subsection{利用环视来查找替换}

把“Jeffs”替换为“Jeff's”可以有很多种实现:

不用环视(性能最好):“\verb|s/Jeffs/Jeff's/g|”。

单词分界锚点(同上):“\verb|s/\bJeffs\b/Jeff's/g|”。

使用先分组然后再替换:“\verb|s/\b(Jeff)(s)\b/$1'$2/g|”

通过环视:“\verb|s/\bJeff(?=s\b)/Jeff'/g|”

在环视的例子中,环视的内容并不在最终匹配的文本中,因为环视只匹配位置而不包括任何字符。更进一步,我们可以把前面的“Jeff”也放入环视:

\verb|s/(?<=\bJeff)(?=s\b)/'/g|

这样我们只要对应的位置插入了一个字符。



\subsection{利用环视来格式化数字}

以格式化数字“123456789”为“123,456,789”为例,说明环视功能。

算法:左边有数字“\verb|\d|”,而且右边的数字个数正好是3的倍数“\verb|(\d\d\d)+$|”。
