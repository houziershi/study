% 纸张与版面大小:电纸书版本
% \documentclass[10pt,a4paper]{report}
% \addtolength{\textheight}{-4cm}
% \addtolength{\textwidth}{+3cm}
%------------------------------------------------------------------------------------------------------
% 纸张与版面大小:普通A4纸版本
\documentclass[a4paper]{report}
\usepackage[margin=4cm]{geometry}      % 使用geometry包设置边距为2cm
%------------------------------------------------------------------------------------------------------
\usepackage{xltxtra,fontspec,xunicode}
\usepackage[slantfont,boldfont]{xeCJK} % 允许斜体和粗体

\setCJKmainfont{WenQuanYi Micro Hei}             % 缺省中文字体
\setCJKmonofont{WenQuanYi Micro Hei Mono}        % 中文等宽字体
\setmainfont{DejaVu Serif}                       % 英文衬线字体
\setsansfont{DejaVu Sans}                        % 英文无衬线字体
\setmonofont{Monaco}                             % 英文等宽字体
%-------------------------------------------------------------------------------------------------------
\linespread{1.3}                                 % 1.5倍行距,值1.6产生双倍行距
% \setlength{\parindent}{0pt}                    % 段落首行缩进
\setlength{\parskip}{1ex plus 0.5ex minus 0.2ex} % 段落间距为1ex,可让TeX在+0.5到-0.8范围内微调
                                                 % 即实际范围在0.8ex~1.5ex之间
%-------------------------------------------------------------------------------------------------------
% 页眉与页脚设置
\usepackage{fancyhdr}
\pagestyle{fancy}
\fancyhf{}                                       %清空页眉页脚
\fancyhead[LE,RO]{\thepage}                      %页眉偶数页左,奇数页右
\fancyhead[RE]{\leftmark}                        %页眉偶数页右
\fancyhead[LO]{\rightmark}                       %页眉奇数页左
% \fancyfoot[LE,RO]{\thepage}                    %页脚偶数页左,奇数页右
% \fancyfoot[RE]{\leftmark}                      %页脚偶数页右
% \fancyfoot[LO]{\rightmark}                     %页脚奇数页左
\fancypagestyle{plain}{                          %重定义plain页面样式
    \fancyhf{}
    \renewcommand{\headrulewidth}{0pt}
}
%-------------------------------------------------------------------------------------------------------
% 使用跨行与跨列的表格
\usepackage{multirow}
%-------------------------------------------------------------------------------------------------------
% listings 与 xcolor 配合实现源代码的语法高亮
\usepackage{xcolor}
\usepackage{listings}
\lstset{
	language=Java,
	frame = shadowbox,
	basicstyle = \ttfamily\small,
	columns = fixed,
	numbers = left,
	numberstyle = \footnotesize,
	stepnumber = 1,
	tabsize = 2,
	showspaces = false,
	showstringspaces = false,
	showtabs = false,
	captionpos = b,
	breaklines = tr[],
	breakatwhitespace = false,
	backgroundcolor = \color{white},
	keywordstyle=\color{blue},
	numberstyle=\color[RGB]{0,192,192},
	commentstyle=\color[RGB]{0,96,96},
	stringstyle=\ttfamily\slshape\color[RGB]{128,0,0},
	escapeinside=``
}
%-------------------------------------------------------------------------------------------------------
\usepackage{graphicx}                            % 引入图片
\graphicspath{{img/}{images/}}                   % 要导入的图片的位置。可以有多个目录,但就算只有一个目录,也要用两级花括号
%-------------------------------------------------------------------------------------------------------
	\title{正则表达式学习笔记}                             % 文章的标题
	\author{                                     % 作者与致谢
		阿左 \thanks{感谢档} \and 
		Nobody \thanks{感谢郭嘉}
	}
	\date{\today}                                % 日期
	
%-------------------------------------------------------------------------------------------------------
\begin{document}
	\maketitle                                   % 制作标题
	\tableofcontents                             % 生成章节目录
	\setcounter{tocdepth}{5}                     % 生成章节的目录深度
	\listoffigures                               % 生成图片目录
	\listoftables                                % 生成表格目录


	\begin{abstract}                             % 英文摘要
		Regex study note
	\end{abstract}

	\renewcommand{\abstractname}{摘要}           % 中文摘要
	\begin{abstract}
		Regex study note
	\end{abstract}

	\part{基本概念}

		
\chapter{元字符(Metacharacters)}

\section{基本元字符}

\subsection{任意字符}

点号“\verb|.|”匹配任意一个字符。

\subsection{行开始与结束}

脱字符与美元符分别代表行的开始与结束位置。注意这两个元字符只表示两个特殊的位置,位置上是没有字符的。
\lstinputlisting[firstline=1,lastline=1]{src/ch01/00.txt}

匹配空白行:
\lstinputlisting[firstline=2,lastline=2]{src/ch01/00.txt}

匹配所有的行(因为所有的行都有一个开头):
\lstinputlisting[firstline=3,lastline=3]{src/ch01/00.txt}

\subsection{单词分界符(Word Boundaries)}

“\verb|\<|”与“\verb|\>|”匹配单词的开始与结束。注意匹配的是位置,而不是字符。







\section{字符范围(Character Classes)}

“\verb|[...]|”可以定义一个位置上可以出现的字符的范围。“\verb|<H1>|”、“\verb|<H2>|”、“\verb|<H3>|”可以用:“\verb|<H[123]>|”来表示。

“\verb|[^...]|”表示排除指定字符。没有列出来的任何字符都可以。

表达式“\verb|q[^u]|”匹配不了单词“Iraq”,因为表达式的意义不是“q”后面没有u,而是“q”后面要“有”一个字符,这个字符不能是“u”,其他的都行。

可以用连字符来表示连续的字符:“\verb|[0-9a-zA-Z_!.?]|”;只有在也只有连字符是特殊字符。后面的下划线、问号、点号等都是普通字符。

如果连字符在开头,那也表示普通字符,不表示连续字符。
\begin{lstlisting}[language=bash]
echo '-123456789' | egrep '[a-b]'   # not match
echo '-123456789' | egrep '[-ab]'   # match
\end{lstlisting}





\section{选择结构(Alternation)}

括号构成子表达式,“\verb|||”表示逻辑“或”。
\begin{lstlisting}[language=bash]
Jeffrey|Jeffery
Jeff(re|er)y
\end{lstlisting}




\section{重复控制}

\subsection{选项元素(Optional Items)}

“\verb|?|”表示前一个元素是可选的。“\verb|July?|”可以匹配“Jul”或“July”。

\subsection{其他量词:重复出现(Other Quantifier: Repetition)}

“\verb|+|”表示前一元素出现一次以上;“\verb|*|”表示前一元素可不出现或出现多次。

\subsection{区间量次(Interval Quantifier)}

“\verb|{min,max}|”规定重复出现的次数:
\begin{lstlisting}[language=bash]
echo '1234567890' | egrep '[0-9]{8,15}' 
\end{lstlisting}





\section{括号与反向引用(Parentheses and Backreferences)}

在很多版本的正则表达式中,括号中的子表达式能“记住”匹配的内容。“verb|/num|”可以代表第几个子表达式匹配的内容。如,要查找重复的单词:

\begin{lstlisting}[language=bash]
echo 'that that' | egrep '\<([A-Za-z]+) +\1\>'
\end{lstlisting}

“\verb|([a-z])([0-9])\1\2|”这个表达式中,“\verb|\1|”表示第一个表达式“\verb|[0-9]|”匹配的内容;“\verb|\2|”表示第二个表达式“\verb|[0-9]|”匹配的内容。

\section{转义元字符}

反斜线“\verb|\|”实现元字符的转义。大多数正则工具会把字符范围“\verb|[...]|”中的“\verb|\|”作为普通字符。






		\chapter{拓展}

\section{引用匹配的内容}

在Perl语言中通过\verb|$num|取得匹配的表达式内容:

\lstinputlisting[language=Perl]{src/ch02/exp06.pl}

通过\verb|$(?:...)|只用来分组,但是不取得匹配内容:

\lstinputlisting[language=Perl]{src/ch02/exp07.pl}

回到温度转换的例子,根据用户输入最后是C还是F来判断输入的类型:

\lstinputlisting[language=Perl]{src/ch02/exp08.pl}


\section{环视功能(lookaround)}

环视具体有以下四种:

顺序肯定环视“\verb|(?=...)|”:某个位置的右边符合子表达式。

顺序否定环视“\verb|(?!...)|”:某个位置的右边不符合子表达式。

逆序肯定环视“\verb|(?<=...)|”:某个位置的左边符合子表达式。

逆序否定环视“\verb|(?<!...)|”:某个位置的左边不符合子表达式。

\subsection{环视只匹配位置}

环视功能只匹配位置,而不匹配具体的字符(就像是行头“\verb|^|”、字符分界符“\verb|\b|”)。它匹配的是某一个位置前后的内容是否符合。

例如:表达式“\verb|Jeffrey|”匹配的是一串文本:

\begin{verbatim}
by Jeffrey Friedl.
   ^-----^
\end{verbatim}

而环视“\verb|(?=Jeffrey)|”匹配的的两个字符之间的位置:

\begin{verbatim}
by Jeffrey Friedl.
 -><-
\end{verbatim}

再看一个例子,“\verb|(?=Jeffery)Jeff|”和“\verb|Jeff(?=rey)|”是等价的:

“\verb|(?=Jeffery)Jeff|”:从“Jeffery”的开头位置开始找“Jeff”。

“\verb|Jeff(?=rey)|”:找到后面有“rey”的“Jeff”。



\subsection{利用环视来查找替换}

把“Jeffs”替换为“Jeff's”可以有很多种实现:

不用环视(性能最好):“\verb|s/Jeffs/Jeff's/g|”。

单词分界锚点(同上):“\verb|s/\bJeffs\b/Jeff's/g|”。

使用先分组然后再替换:“\verb|s/\b(Jeff)(s)\b/$1'$2/g|”

通过环视:“\verb|s/\bJeff(?=s\b)/Jeff'/g|”

在环视的例子中,环视的内容并不在最终匹配的文本中,因为环视只匹配位置而不包括任何字符。更进一步,我们可以把前面的“Jeff”也放入环视:

\verb|s/(?<=\bJeff)(?=s\b)/'/g|

这样我们只要对应的位置插入了一个字符。



\subsection{利用环视来格式化数字}

以格式化数字“123456789”为“123,456,789”为例,说明环视功能。

算法:左边有数字“\verb|\d|”,而且右边的数字个数正好是3的倍数“\verb|(\d\d\d)+$|”。


	\part{语言与工具}

		\chapter{egrep}



\section{基本使用}

在邮件中查找发信人与主题的例子:

\begin{lstlisting}[language=bash]
egrep '^(From|Subject):' ./*
\end{lstlisting}



\section{忽略大小写}

\begin{lstlisting}[language=bash]
egrep -i '^(From|Subject):' ./*
\end{lstlisting}



\section{反向引用}

有些版本的egrep有个bug:使用“-i”忽略大小写时会对反向引用无效,即可以查到“the the”但是查不到“The the”。

\begin{lstlisting}[language=bash]
echo 'the the' | egrep '\<([A-Za-z]+) +\1\>'
the the
echo 'the The' | egrep '\<([A-Za-z]+) +\1\>'
\end{lstlisting}





		\chapter{Perl}

\section{元字符}

\subsection{空白字符}

\begin{table}[htbp]
	\caption{Perl正则元字符}
	\label{tab:perl.regex.meta}
	\centering
	\begin{tabular}{ll}
		\hline
		元字符 & 作用 \\
		\hline
		\verb|\t| & 制表符 \\
		\verb|\n| & 换行 \\
		\verb|\b| & 一般情况表示下单词分界,但在字符范围中表示退格。\\
		\verb|\s| & 空白字符(包括空格、制表符、换行)\\
		\verb|\S| & 除了“\verb|\s|”以外的任何字符\\
		\verb|\w| & “\verb|[A-Za-z0-9]|”\\
		\verb|\W| & 除了“\verb|\W|”以外的任何字符\\
		\verb|\d| & “\verb|[0-9]|”\\
		\verb|\D| & 除了“\verb|\d|”以外的任何字符\\
		\hline
	\end{tabular}
\end{table}



\section{基本使用}

查找文件中接连重复出现的单词

\lstinputlisting[language=Perl]{src/ch02/exp01.pl}

\subsection{变量的声明与引用}

普通变量以美元符开头,而且可以在输出语句中直接使用。

以下是一个转摄氏度为华氏度的例子:

\lstinputlisting[language=Perl]{src/ch02/exp02.pl}

\subsection{控制结构}

\lstinputlisting[language=Perl]{src/ch02/exp03.pl}

运行时可以通过参数“\verb|-w|”打开编译警告:

\begin{lstlisting}[language=bash]
perl -w exp03.pl
\end{lstlisting}


\section{用正则匹配文本}

“\verb|=~|”指定正则操作的对象。

“\verb|m/.../|”表示通过正则进行的操作是匹配操作(可以省略m,但是加上看起来更加清楚)。

“\verb|==|”用来比较两个数字是否相等。

“\verb|eq|”用来比较两个字符串是否相等。

查找是否是数字:

\lstinputlisting[language=Perl]{src/ch02/exp04.pl}



\section{取得用户输入}

增加能够处理小数部分;并通过函数“\verb|printf|”格式化输出。

\lstinputlisting[language=Perl]{src/ch02/exp05.pl}



\section{引用匹配的内容}

通过\verb|$num|取得匹配的表达式内容:

\lstinputlisting[language=Perl]{src/ch02/exp06.pl}

通过\verb|$(?:...)|只用来分组,但是不取得匹配内容:

\lstinputlisting[language=Perl]{src/ch02/exp07.pl}

回到温度转换的例子,根据用户输入最后是C还是F来判断输入的类型:

\lstinputlisting[language=Perl]{src/ch02/exp08.pl}



\section{修饰符}

\subsection{忽略大小写}

在正则以后加个修饰符“\verb|/i|”表示忽略大小写。

\begin{lstlisting}[language=Perl]
$input =~ m/^aaa$/i;
\end{lstlisting}

更加完整的温度转换例子:

\lstinputlisting[language=Perl]{src/ch02/exp08.pl}

\subsection{全局匹配}

修饰符“\verb|/g|”表示全局匹配。就是在完成了一次匹配以后,再继续匹配剩下的内容。

\begin{lstlisting}[language=Perl]
$input =~ m/^aaa$/g;
\end{lstlisting}

\subsection{宽松排列表达式}

修饰符“\verb|/x|”表示。

\begin{lstlisting}[language=Perl]
$input =~ m/^aaa$/x;
\end{lstlisting}



\section{替换文本}

“\verb|$var =~ s/regex/replacement/|”以变量“\verb|$var|”为对象,把符合正则的内容替换掉。

例:无视大小写,把“peter”替换成“Peter”。

\begin{lstlisting}[language=Perl]
$var =~ s/\bpeter\b/Peter/i;
print "$var";
\end{lstlisting}

\subsection{使用perl自动替换文本}

参数“-p”表示对目标文件每一行进行查找和替换;
参数“-i”表示替换的结果写回文件;
参数“-e”表示后面的字符串就是程序的代码;

\begin{lstlisting}[language=Perl]
perl -p -i -e 's/sysread/read/g' filename
\end{lstlisting}

可以合并参数为:

\begin{lstlisting}[language=Perl]
perl -pi -e 's/sysread/read/g' filename
\end{lstlisting}

\subsection{生成邮件回复的例子}

原始内容在文件“file.in”,通过程序“mkreply.pl”把结果存放在“file.out”。

file.in

\lstinputlisting[language=Perl]{src/ch02/file.in}

希望程序能自动生成回复的样式:

\lstinputlisting[language=Perl]{src/ch02/out.exp}

调用的方法:

\begin{lstlisting}[language=Perl]
perl -w mkreply.pl file.in > file.out
\end{lstlisting}

\section{从文件读取}

Perl提供了操作符“\verb|<>|”把每一行读取到变量中,我们通过“\verb|^\s*$|”检查空行(表示邮件head结束)。

\lstinputlisting[language=Perl,firstline=0,lastline=6]{src/ch02/exp11.pl}

从邮件头中提取信息的方法,以Subject为例:

\lstinputlisting[language=Perl,firstline=0,lastline=3]{src/ch02/exp10.pl}

分别取得回信地址和昵称:

\lstinputlisting[language=Perl,firstline=5,lastline=9]{src/ch02/exp10.pl}

就连输出原文引用的部人也可以通过正则来实现:

\lstinputlisting[language=Perl,firstline=13,lastline=14]{src/ch02/exp10.pl}

\section{增强锚点}

通常来说,锚点“\verb|^|”、“\verb|$|”匹配的不是逻辑行的开头与结尾,而是整个字符串的开头与结尾。如果要匹配逻辑行可以切换到增强锚点(enhanced line anchor)模式下。在Perl语言中修饰符为“\verb|/m|”。例如,把空白行替换为HTML的段落符“\verb|<p>|”:

\lstinputlisting[language=Perl,firstline=8,lastline=8]{src/ch02/exp11.pl}


\section{格式化}

为了增加可读性,修饰符“\verb|/x|”允许对正则表达式进行排版(还可以用花括号代替斜线):

\lstinputlisting[language=Perl,firstline=10,lastline=19]{src/ch02/exp11.pl}


\section{重用正则对象}

修饰符“\verb|qr/.../|”表示不应用到字符串,建立一个对象以后再用:

\lstinputlisting[language=Perl,firstline=21,lastline=25]{src/ch02/exp11.pl}



\end{document}
