% 纸张与版面大小:电纸书版本
% \documentclass[10pt,a4paper]{report}
% \addtolength{\textheight}{-4cm}
% \addtolength{\textwidth}{+3cm}
%------------------------------------------------------------------------------------------------------
% 纸张与版面大小:普通A4纸版本
\documentclass[a4paper]{report}
\usepackage[margin=4cm]{geometry}      % 使用geometry包设置边距为2cm
%------------------------------------------------------------------------------------------------------
\usepackage{xltxtra,fontspec,xunicode}
\usepackage[slantfont,boldfont]{xeCJK} % 允许斜体和粗体

\setCJKmainfont{WenQuanYi Micro Hei}             % 缺省中文字体
\setCJKmonofont{WenQuanYi Micro Hei Mono}        % 中文等宽字体
\setmainfont{DejaVu Serif}                       % 英文衬线字体
\setsansfont{DejaVu Sans}                        % 英文无衬线字体
\setmonofont{Monaco}                             % 英文等宽字体
%-------------------------------------------------------------------------------------------------------
\linespread{1.3}                                 % 1.5倍行距,值1.6产生双倍行距
% \setlength{\parindent}{0pt}                    % 段落首行缩进
\setlength{\parskip}{1ex plus 0.5ex minus 0.2ex} % 段落间距为1ex,可让TeX在+0.5到-0.8范围内微调
                                                 % 即实际范围在0.8ex~1.5ex之间
%-------------------------------------------------------------------------------------------------------
% 页眉与页脚设置
\usepackage{fancyhdr}
\pagestyle{fancy}
\fancyhf{}                                       %清空页眉页脚
\fancyhead[LE,RO]{\thepage}                      %页眉偶数页左,奇数页右
\fancyhead[RE]{\leftmark}                        %页眉偶数页右
\fancyhead[LO]{\rightmark}                       %页眉奇数页左
% \fancyfoot[LE,RO]{\thepage}                    %页脚偶数页左,奇数页右
% \fancyfoot[RE]{\leftmark}                      %页脚偶数页右
% \fancyfoot[LO]{\rightmark}                     %页脚奇数页左
\fancypagestyle{plain}{                          %重定义plain页面样式
    \fancyhf{}
    \renewcommand{\headrulewidth}{0pt}
}
%-------------------------------------------------------------------------------------------------------
% 使用跨行与跨列的表格
\usepackage{multirow}
%-------------------------------------------------------------------------------------------------------
% listings 与 xcolor 配合实现源代码的语法高亮
\usepackage{xcolor}
\usepackage{listings}
\lstset{
	language=Java,
	frame = shadowbox,
	basicstyle = \ttfamily\small,
	columns = fixed,
	numbers = left,
	numberstyle = \footnotesize,
	stepnumber = 1,
	tabsize = 2,
	showspaces = false,
	showstringspaces = false,
	showtabs = false,
	captionpos = b,
	breaklines = tr[],
	breakatwhitespace = false,
	backgroundcolor = \color{white},
	keywordstyle=\color{blue},
	numberstyle=\color[RGB]{0,192,192},
	commentstyle=\color[RGB]{0,96,96},
	stringstyle=\ttfamily\slshape\color[RGB]{128,0,0},
	escapeinside=``
}
%-------------------------------------------------------------------------------------------------------
\usepackage{graphicx}                            % 引入图片
\graphicspath{{img/}{images/}}                   % 要导入的图片的位置。可以有多个目录,但就算只有一个目录,也要用两级花括号
%-------------------------------------------------------------------------------------------------------
	\title{正则表达式学习笔记}                             % 文章的标题
	\author{                                     % 作者与致谢
		阿左 \thanks{感谢档} \and 
		Nobody \thanks{感谢郭嘉}
	}
	\date{\today}                                % 日期
	
%-------------------------------------------------------------------------------------------------------
\begin{document}
	\maketitle                                   % 制作标题
	\tableofcontents                             % 生成章节目录
	\setcounter{tocdepth}{5}                     % 生成章节的目录深度
	\listoffigures                               % 生成图片目录
	\listoftables                                % 生成表格目录


	\begin{abstract}                             % 英文摘要
		Regex study note
	\end{abstract}

	\renewcommand{\abstractname}{摘要}           % 中文摘要
	\begin{abstract}
		Regex study note
	\end{abstract}

	\part{基本概念}

		
\chapter{元字符(Metacharacters)}

\section{基本元字符}

\subsection{任意字符}

点号“\verb|.|”匹配任意一个字符。

\subsection{行开始与结束}

脱字符与美元符分别代表行的开始与结束位置。注意这两个元字符只表示两个特殊的位置,位置上是没有字符的。
\lstinputlisting[firstline=1,lastline=1]{src/ch01/00.txt}

匹配空白行:
\lstinputlisting[firstline=2,lastline=2]{src/ch01/00.txt}

匹配所有的行(因为所有的行都有一个开头):
\lstinputlisting[firstline=3,lastline=3]{src/ch01/00.txt}

\subsection{单词分界符(Word Boundaries)}

“\verb|\<|”与“\verb|\>|”匹配单词的开始与结束。注意匹配的是位置,而不是字符。







\section{字符范围(Character Classes)}

“\verb|[...]|”可以定义一个位置上可以出现的字符的范围。“\verb|<H1>|”、“\verb|<H2>|”、“\verb|<H3>|”可以用:“\verb|<H[123]>|”来表示。

“\verb|[^...]|”表示排除指定字符。没有列出来的任何字符都可以。

表达式“\verb|q[^u]|”匹配不了单词“Iraq”,因为表达式的意义不是“q”后面没有u,而是“q”后面要“有”一个字符,这个字符不能是“u”,其他的都行。

可以用连字符来表示连续的字符:“\verb|[0-9a-zA-Z_!.?]|”;只有在也只有连字符是特殊字符。后面的下划线、问号、点号等都是普通字符。

如果连字符在开头,那也表示普通字符,不表示连续字符。
\begin{lstlisting}[language=bash]
echo '-123456789' | egrep '[a-b]'   # not match
echo '-123456789' | egrep '[-ab]'   # match
\end{lstlisting}





\section{选择结构(Alternation)}

括号构成子表达式,“\verb|||”表示逻辑“或”。
\begin{lstlisting}[language=bash]
Jeffrey|Jeffery
Jeff(re|er)y
\end{lstlisting}




\section{重复控制}

\subsection{选项元素(Optional Items)}

“\verb|?|”表示前一个元素是可选的。“\verb|July?|”可以匹配“Jul”或“July”。

\subsection{其他量词:重复出现(Other Quantifier: Repetition)}

“\verb|+|”表示前一元素出现一次以上;“\verb|*|”表示前一元素可不出现或出现多次。

\subsection{区间量次(Interval Quantifier)}

“\verb|{min,max}|”规定重复出现的次数:
\begin{lstlisting}[language=bash]
echo '1234567890' | egrep '[0-9]{8,15}' 
\end{lstlisting}





\section{括号与反向引用(Parentheses and Backreferences)}

在很多版本的正则表达式中,括号中的子表达式能“记住”匹配的内容。“verb|/num|”可以代表第几个子表达式匹配的内容。如,要查找重复的单词:

\begin{lstlisting}[language=bash]
echo 'that that' | egrep '\<([A-Za-z]+) +\1\>'
\end{lstlisting}

“\verb|([a-z])([0-9])\1\2|”这个表达式中,“\verb|\1|”表示第一个表达式“\verb|[0-9]|”匹配的内容;“\verb|\2|”表示第二个表达式“\verb|[0-9]|”匹配的内容。

\section{转义元字符}

反斜线“\verb|\|”实现元字符的转义。大多数正则工具会把字符范围“\verb|[...]|”中的“\verb|\|”作为普通字符。






\end{document}
