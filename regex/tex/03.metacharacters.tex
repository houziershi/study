\chapter{元字符}

\section{数值传义}

\subsection{通过八进制转义}

格式为“\verb|\num|”,例如:“\verb|\015\012|”。取值范围一般在“\verb|\000|”到“\verb|\377|”之间,而且通常要求以0开头。

\subsection{通过十六进制转义}

格式有:“\verb|\xnum|”、“\verb|\x{num}|”、“\verb|\unum|”、“\verb|Unum|”。

\section{字符集合}

\begin{table}[htbp]
	\caption{常用正则元字符}
	\label{tab:part.meta.charset}
	\centering
	\begin{tabular}{ll}
		\hline
		元字符 & 作用 \\
		\hline
		\verb|\s| & 空白字符(包括空格、制表符、换行)\\
		\verb|\S| & 除了“\verb|\s|”以外的任何字符\\
		\verb|\w| & “\verb|[A-Za-z0-9]|”\\
		\verb|\W| & 除了“\verb|\W|”以外的任何字符\\
		\verb|\d| & “\verb|[0-9]|”\\
		\verb|\D| & 除了“\verb|\d|”以外的任何字符\\
		\hline
	\end{tabular}
\end{table}

\section{Unicode属性}

Unicode不仅是字符的映射,还记录了每个字符的属性(是大写还是小写字符、是从右向左读的……)。

匹配属性的格式为“\verb|\p{Prop}|”或“\verb|\P{Prop}|”。如属性“\verb|\p{L}|”匹配了字符属性(相对于数字、标点、口音等),还有相等的多字母表示方式“\verb|p{Letter}|”。还有些系统中在单字符版本中可以省略花括号。还有些系统中可以加上条件“\verb|In|”或“\verb|Is|”来限制条件,如“\verb|\p{IsL}|”。

许多属性还可以进一步加上子属性,如字符可以再一步描述是大写字符还是小写字符。

\subsection{字母属性}

“\verb|\p{L}|”或“\verb|\p{Letter}|”字母。

“\verb|\p{Ll}|”或“\verb|\p{Lowercase_Letter}|”小写字母。

“\verb|\p{Lu}|”或“\verb|\p{Uppercase_Letter}|”大写字母。

“\verb|\p{Lt}|”或“\verb|\p{Titlecase_Letter}|”出现在单词开头的字母(某些语言单词组合中会有)。

“\verb|\p{L&}|”包含了“\verb|\p{Ll}|”“\verb|\p{Lu}|”“\verb|\p{Lt}|”三者的集合。

“\verb|\p{Lm}|”或“\verb|\p{Modifier_Letter}|”少数的样子像字母,其实是特殊用途的字符。

“\verb|\p{Lo}|”或“\verb|\p{Other_Letter}|”没有大小写、也不是修饰符的字母。包括希伯来语、阿拉伯语、日语中的字母。

“\verb|\p{M}|”或“\verb|\p{Mark}|”重音符号等修饰符号,不能单独出现。

“\verb|\p{Mn}|”或“\verb|\p{Non_Spacing_Mark}|”修饰其他字符的重音符、变音符等。

“\verb|\p{Mc}|”或“\verb|\p{Spacing_Combining_Mark}|”会占一定宽度的修饰符,孟加拉语、马来语中有。

“\verb|\p{Me}|”或“\verb|\p{Encolsing_Mark}|”可以围住其他字符的标记,圆圈、方框等。

“\verb|\p{Z}|”或“\verb|\p{Separator}|”空白的分隔符。

“\verb|\p{Zs}|”或“\verb|\p{Space_Separator}|”空格、制表符等。

“\verb|\p{Zl}|”或“\verb|\p{Line_Separator}|”LINE SEPAPATOR(U+2028)。

“\verb|\p{Zp}|”或“\verb|\p{Paragraph_Separator}|”PAPAGRAPH SEPARATOR(U+2029)。

“\verb|\p{S}|”或“\verb|\p{Symbol}|”图形与符号。

“\verb|\p{Sm}|”或“\verb|\p{Math_Symbol}|”数学符号,加减乘除等。

“\verb|\p{Sc}|”或“\verb|\p{Currency_Symbol}|”货币符号。

“\verb|\p{Sk}|”或“\verb|\p{Modifier_Symbol}|”组合字符,但作为功能完整的字符有自己的意义。

“\verb|\p{So}|”或“\verb|\p{Other_Symbol}|”印刷符号、框图等。

“\verb|\p{N}|”或“\verb|\p{Number}|”数字。

“\verb|\p{Nd}|”或“\verb|\p{Decimal_Digit_Number}|”各种表示0到9的数字(但不包括中日韩)。

“\verb|\p{Nl}|”或“\verb|\p{Letter_Number}|”几乎所有的罗马数字。

“\verb|\p{No}|”或“\verb|\p{Other_Number}|”作为加密符号与记数符号,(但不包括中日韩)。

“\verb|\p{P}|”或“\verb|\p{Punctuation}|”标点符号。

“\verb|\p{Pd}|”或“\verb|\p{Dash_Punctuation}|”各种连字符与短划线。

“\verb|\p{Ps}|”或“\verb|\p{Open_Punctuation}|”像是(、《等开符号。

“\verb|\p{Pe}|”或“\verb|\p{Close_Punctuation}|”像是)、》等闭符号。

“\verb|\p{Pi}|”或“\verb|\p{Initial_Punctuation}|”像是“、<等。

“\verb|\p{Pf}|”或“\verb|\p{Final_Punctuation}|”像是”、>等。

“\verb|\p{Pc}|”或“\verb|\p{Connector_Punctuation}|”少数有特殊语法含义的标点,如下划线。

“\verb|\p{Po}|”或“\verb|\p{Other_Punctuation}|”其他标点,句点、感叹号等。

“\verb|\p{C}|”或“\verb|\p{Other}|”其他任何字符。

“\verb|\p{Cc}|”或“\verb|\p{Control}|”ASCII和Latin-1编码中的控制字符。

“\verb|\p{Cf}|”或“\verb|\p{Format}|”表示格式的不可见字符。

“\verb|\p{Co}|”或“\verb|\p{Private_Use}|”私人用途,如公司的Logo等。

“\verb|\p{Cn}|”或“\verb|\p{Unassigned}|”末分配的代码点。

\subsection{字母表(Scripts)}

字母表匹配一个语系中独有的字符,如“\verb|\p{Hebrew}|”匹配只有希伯莱文才有的字符。

有些字符不属于任何字母表(如句点或空格)而属于通用字母表,用“\verb|\p{IsCommon}|”匹配。还有一个伪字母表Inherited包括从其所属的字母表中基本字符继承而来的组合字符。

\subsection{区块(Block)}

区块代码了一段连续的代码(通常与语言地区相关),如西藏字符在Perl与java.util.regex中可以用“\verb|\p{InTibetan}|”来匹配。

\subsection{字母表与区块}

属于某个字母表的字符可能同时包含多个区块,而且字母表和区块很容易混淆。如Unicode同时提供了Tibetan字母表和Tibetan区块。



\section{字符集合的集合运算}

\subsection{简单的排除运算}

.NET提供了简单的排除(减法)运算,如“\verb|[[a-z]-[aeiou]]|”就取出了所有的辅音。再看一个排除标点符号中除了书名号括号等成对符号的例子:“\verb|[\p{P}-[\p{Ps}\p{Pe}]|”。

\subsection{完整的字符集合运算}

Sun的Java正则包提供了完整的字符集合运算(并、交、减)。

并集运算:“\verb|[abcxyz]|”相同的表示有“\verb|[[abc][xyz]]|”、“\verb|[abc[xyz]]|”、“\verb|[[abc]xyz]|”。

交集运算:“\verb|[\p{InThai}&&\p{Cn}]|”

排除运算:“\verb|[\p{InThai}&&[^\p{Cn}]]|”。上一节中.NET取辅音的例子可以写为:“\verb|[[a-z]&&[^aeiou]]|”。






