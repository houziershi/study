
\chapter{元字符(Metacharacters)}

\section{基本元字符}

\subsection{任意字符}

点号“\verb|.|”匹配任意一个字符。

\subsection{行开始与结束}

脱字符与美元符分别代表行的开始与结束位置。注意这两个元字符只表示两个特殊的位置,位置上是没有字符的。
\lstinputlisting[firstline=1,lastline=1]{src/ch01/00.txt}

匹配空白行:
\lstinputlisting[firstline=2,lastline=2]{src/ch01/00.txt}

匹配所有的行(因为所有的行都有一个开头):
\lstinputlisting[firstline=3,lastline=3]{src/ch01/00.txt}

\subsection{单词分界符(Word Boundaries)}

“\verb|\<|”与“\verb|\>|”匹配单词的开始与结束。注意匹配的是位置,而不是字符。







\section{字符范围(Character Classes)}

“\verb|[...]|”可以定义一个位置上可以出现的字符的范围。“\verb|<H1>|”、“\verb|<H2>|”、“\verb|<H3>|”可以用:“\verb|<H[123]>|”来表示。

“\verb|[^...]|”表示排除指定字符。没有列出来的任何字符都可以。

表达式“\verb|q[^u]|”匹配不了单词“Iraq”,因为表达式的意义不是“q”后面没有u,而是“q”后面要“有”一个字符,这个字符不能是“u”,其他的都行。

可以用连字符来表示连续的字符:“\verb|[0-9a-zA-Z_!.?]|”;只有在也只有连字符是特殊字符。后面的下划线、问号、点号等都是普通字符。

如果连字符在开头,那也表示普通字符,不表示连续字符。
\begin{lstlisting}[language=bash]
echo '-123456789' | egrep '[a-b]'   # not match
echo '-123456789' | egrep '[-ab]'   # match
\end{lstlisting}





\section{选择结构(Alternation)}

括号构成子表达式,“\verb|||”表示逻辑“或”。
\begin{lstlisting}[language=bash]
Jeffrey|Jeffery
Jeff(re|er)y
\end{lstlisting}




\section{重复控制}

\subsection{选项元素(Optional Items)}

“\verb|?|”表示前一个元素是可选的。“\verb|July?|”可以匹配“Jul”或“July”。

\subsection{其他量词:重复出现(Other Quantifier: Repetition)}

“\verb|+|”表示前一元素出现一次以上;“\verb|*|”表示前一元素可不出现或出现多次。

\subsection{区间量次(Interval Quantifier)}

“\verb|{min,max}|”规定重复出现的次数:
\begin{lstlisting}[language=bash]
echo '1234567890' | egrep '[0-9]{8,15}' 
\end{lstlisting}





\section{括号与反向引用(Parentheses and Backreferences)}

在很多版本的正则表达式中,括号中的子表达式能“记住”匹配的内容。“verb|/num|”可以代表第几个子表达式匹配的内容。如,要查找重复的单词:

\begin{lstlisting}[language=bash]
echo 'that that' | egrep '\<([A-Za-z]+) +\1\>'
\end{lstlisting}

“\verb|([a-z])([0-9])\1\2|”这个表达式中,“\verb|\1|”表示第一个表达式“\verb|[0-9]|”匹配的内容;“\verb|\2|”表示第二个表达式“\verb|[0-9]|”匹配的内容。

\section{转义元字符}

反斜线“\verb|\|”实现元字符的转义。大多数正则工具会把字符范围“\verb|[...]|”中的“\verb|\|”作为普通字符。




