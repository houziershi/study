\chapter{基本工具}



\section{日期时间处理}



\subsection{格里高利历}

通过GregorianCalendar进行日期操作:

\lstinputlisting[]{code/02.base/src/example/CalendarExample.java}



\section{文本}



\subsection{换行符}

在不同操作系统下取得换行符:

\lstinputlisting[]{code/02.base/src/stringtools/StringConstans.java}



\section{枚举类型}

\lstinputlisting[]{code/02.base/src/stringtools/Gender.java}

\lstinputlisting[]{code/02.base/src/stringtools/Color.java}

\lstinputlisting[]{code/02.base/test/test/EnumTest.java}



\section{数学}



\subsection{NaN与无穷大}

NaN表示非数字,定义在\verb|java.lang.Float|与\verb|java.lang.Double|中。这两个类中同样还定义了正负无穷大的常量\verb|POSITIVE_INFINITY|和\verb|NEGATIVE_INFINITY|。整数除以0会导致错误,但double和float会在数学上生产合理的无穷大。

\lstinputlisting[]{code/02.base/test/test/MathTest.java}



\subsection{通过位逻辑处理权限}

记录权限的枚举类:

\lstinputlisting[]{code/02.base/src/example/UserAuth.java}

系统管理员类:

\lstinputlisting[]{code/02.base/src/example/SysAdmin.java}

权限判断:

\lstinputlisting[]{code/02.base/test/test/UserAuthTest.java}



\subsection{异或操作实现奇偶检验}

基本的思想就是数一下位的值为1的个数是奇数还是偶数:

\lstinputlisting[]{code/02.base/src/example/ParityChecker.java}

更加严格的检验除了给整个字节流加一位检验以外,还给每一个字节加上一个检验位。



\subsection{BitSet}

\verb|java.util.BitSet|类封装了一个以二进制位为元素的向量,并且长度可变方便进行位操作。这个类的优点不多,但是它的范围超过int类的取值范围。



\subsection{数字的不同进制显示}

\lstinputlisting[]{code/02.base/test/test/NumberStringTest.java}



\subsection{随机数}

Math类提供的random方法返回一个从$0.0$到$1.0$之间的double类伪随机数。

\verb|java.util.Random|功能更全面,产生boolean、byte、int、long、float、double、甚至高斯型结果的伪随机数。如:\verb|Random|类的\verb|nextBoolean|方法根据提供的种子(没有种子就用系统时间当种子)返回布尔型的随机数,相同的种子产相同的数字序列。

还有一个\verb|java.util.SecureRandom|类用来生成标准的、强加密的伪随机数。







