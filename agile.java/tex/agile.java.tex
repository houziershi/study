\documentclass[10pt,a4paper]{report}
% \addtolength{\textheight}{-4cm}
% \addtolength{\textwidth}{+2cm}
%------------------------------------------------------------------------------------------------------
\usepackage{xltxtra,fontspec,xunicode}
\usepackage[slantfont,boldfont]{xeCJK} % 允许斜体和粗体

\setCJKmainfont{WenQuanYi Micro Hei}             % 缺省中文字体
\setCJKmonofont{WenQuanYi Micro Hei Mono}        % 中文等宽字体
\setmainfont{DejaVu Serif}                       % 英文衬线字体
\setsansfont{DejaVu Sans}                        % 英文无衬线字体
\setmonofont{Monaco}                             % 英文等宽字体
%-------------------------------------------------------------------------------------------------------
\linespread{1.3}                                 % 1.5倍行距,值1.6产生双倍行距
% \setlength{\parindent}{0pt}                    % 段落首行缩进
\setlength{\parskip}{1ex plus 0.5ex minus 0.2ex} % 段落间距为1ex,可让TeX在+0.5到-0.8范围内微调
                                                 % 即实际范围在0.8ex~1.5ex之间
%-------------------------------------------------------------------------------------------------------
% 页眉与页脚设置
\usepackage{fancyhdr}
\pagestyle{fancy}
\fancyhf{}                                       %清空页眉页脚
\fancyhead[LE,RO]{\thepage}                      %页眉偶数页左,奇数页右
\fancyhead[RE]{\leftmark}                        %页眉偶数页右
\fancyhead[LO]{\rightmark}                       %页眉奇数页左
% \fancyfoot[LE,RO]{\thepage}                    %页脚偶数页左,奇数页右
% \fancyfoot[RE]{\leftmark}                      %页脚偶数页右
% \fancyfoot[LO]{\rightmark}                     %页脚奇数页左
\fancypagestyle{plain}{                          %重定义plain页面样式
    \fancyhf{}
    \renewcommand{\headrulewidth}{0pt}
}
%-------------------------------------------------------------------------------------------------------
% listings 与 xcolor 配合实现源代码的语法高亮
\usepackage{xcolor}
\usepackage{listings}
\lstset{
	language=Java,
	frame = shadowbox,
	basicstyle = \ttfamily\small,
	columns = fixed,
	numbers = left,
	numberstyle = \footnotesize,
	stepnumber = 1,
	tabsize = 2,
	showspaces = false,
	showstringspaces = false,
	showtabs = false,
	captionpos = b,
	breaklines = tr[],
	breakatwhitespace = false,
	backgroundcolor = \color{white},
	keywordstyle=\color{blue},
	numberstyle=\color[RGB]{0,192,192},
	commentstyle=\color[RGB]{0,96,96},
	stringstyle=\ttfamily\slshape\color[RGB]{128,0,0},
	escapeinside=``
}
%-------------------------------------------------------------------------------------------------------
\usepackage{graphicx}                            % 引入图片
\graphicspath{{img/}{images/}}                   % 要导入的图片的位置。可以有多个目录,但就算只有一个目录,也要用两级花括号
%-------------------------------------------------------------------------------------------------------
	\title{Agile Java}                             % 文章的标题
	\author{                                     % 作者与致谢
		阿左 \thanks{感谢读者} \and 
		Nobody \thanks{感谢国家}
	}
	\date{\today}                                % 日期
	
%-------------------------------------------------------------------------------------------------------
\begin{document}
	\maketitle                                   % 制作标题
	\tableofcontents                             % 生成章节目录
	\setcounter{tocdepth}{5}                     % 生成章节的目录深度
	\listoffigures                               % 生成图片目录
	\listoftables                                % 生成表格目录

	\part{基本概念}
		
\chapter{开发环境}

\section{JUnit4}

	基本的JUnit4单元测试例子:

	\lstinputlisting[]{code/01.dev.env/src/net/jade/HelloTest.java}

	如果用jdk自带的方式编译与运行很麻烦:

	\lstinputlisting[language=bash]{code/01.dev.env/build.sh}

	有了ant的帮助就方便很多了:

	\lstinputlisting[language=xml]{code/01.dev.env/build.xml}

	\part{常用工具}
		\chapter{基本工具}



\section{日期时间处理}



\subsection{格里高利历}

通过GregorianCalendar进行日期操作:

\lstinputlisting[]{code/02.base/src/example/CalendarExample.java}



\section{文本}



\subsection{换行符}

在不同操作系统下取得换行符:

\lstinputlisting[]{code/02.base/src/stringtools/StringConstans.java}



\section{枚举类型}

\lstinputlisting[]{code/02.base/src/stringtools/Gender.java}

\lstinputlisting[]{code/02.base/src/stringtools/Color.java}

\lstinputlisting[]{code/02.base/test/test/EnumTest.java}



\section{数学}



\subsection{NaN与无穷大}

NaN表示非数字,定义在\verb|java.lang.Float|与\verb|java.lang.Double|中。这两个类中同样还定义了正负无穷大的常量\verb|POSITIVE_INFINITY|和\verb|NEGATIVE_INFINITY|。整数除以0会导致错误,但double和float会在数学上生产合理的无穷大。

\lstinputlisting[]{code/02.base/test/test/MathTest.java}



\subsection{通过位逻辑处理权限}

记录权限的枚举类:

\lstinputlisting[]{code/02.base/src/example/UserAuth.java}

系统管理员类:

\lstinputlisting[]{code/02.base/src/example/SysAdmin.java}

权限判断:

\lstinputlisting[]{code/02.base/test/test/UserAuthTest.java}



\subsection{异或操作实现奇偶检验}

基本的思想就是数一下位的值为1的个数是奇数还是偶数:

\lstinputlisting[]{code/02.base/src/example/ParityChecker.java}

更加严格的检验除了给整个字节流加一位检验以外,还给每一个字节加上一个检验位。



\subsection{BitSet}

\verb|java.util.BitSet|类封装了一个以二进制位为元素的向量,并且长度可变方便进行位操作。这个类的优点不多,但是它的范围超过int类的取值范围。



\subsection{数字的不同进制显示}

\lstinputlisting[]{code/02.base/test/test/NumberStringTest.java}



\subsection{随机数}

Math类提供的random方法返回一个从$0.0$到$1.0$之间的double类伪随机数。

\verb|java.util.Random|功能更全面,产生boolean、byte、int、long、float、double、甚至高斯型结果的伪随机数。如:\verb|Random|类的\verb|nextBoolean|方法根据提供的种子(没有种子就用系统时间当种子)返回布尔型的随机数,相同的种子产相同的数字序列。

还有一个\verb|java.util.SecureRandom|类用来生成标准的、强加密的伪随机数。








		\chapter{输入输出}



\section{字符流}

Writer接口提供了对字符流的输出。下面的例子接受一个writer用来输出文本,而不关心具体输出到哪里:

\lstinputlisting[]{code/03.IO/src/example/ReportExample.java}

\lstinputlisting[]{code/03.IO/test/test/ReportTest.java}



\subsection{文本文件}

把输出写入一个文本文件中:

\lstinputlisting[]{code/03.IO/src/example/FileExample.java}

测试检查文件有内容是否正确,当然在操作之前与之后要删除文件:

\lstinputlisting[]{code/03.IO/test/test/ReportTest.java}

其他File类的常用方法:

创建临时文件:createTempFIle

创建空白文件:createNewFile

文件操作:delete、deleteOnExit、renameTo

查询文件或路径名:getAbsoluteFile、getAbsolutePath、getCanonicalFile、
getCanonicalPath、getName、getPath、toURI、toURL

级别:isFile、isDirectory

属性操作:isHidden、lastModified、length、canRead、canWrite、setLastModified、
setReadOnly

目录操作:exists、list、listFiles、listRoots、mkdir、mkdirs、getParent、
getParentFile



\subsection{字节流的转换}

InputStreamReader和OutputStreamWriter包装了InputStream和OutputStream。将其转换为
字符流。





\section{数据流}



\subsection{基本数据类型}

DataInputStream和DataInputStream提供了基本类型的读写操作,如:readInt、
writeint等:

\lstinputlisting[]{code/03.IO/src/example/BaseTypeExample.java}

测试检查文件有内容是否正确,当然在操作之前与之后要删除文件:

\lstinputlisting[]{code/03.IO/test/test/BaseTypeTest.java}



\subsection{序列化对象}

实现了\verb|java.io.Serializable|接口的类(包括子类)能够被序列化。对于类中不
需要被序列化的成员可以通过关键字“transient”来修饰。加上serialVersionUID来记录类
的版本变化(不同的版本中可能会改变类的成员):

\lstinputlisting[]{code/03.IO/src/example/LogRecSe.java}

直接使用ObjectInputStream与ObjectOutputStream来读写对象:

\lstinputlisting[]{code/03.IO/src/example/ObjectExample.java}

\lstinputlisting[]{code/03.IO/test/test/ObjectSaveTest.java}

		\chapter{类、接口与反射}


\section{内部类}

内部类可以访问定义在外部类中的实例变量;静态内部类不可以访问定义在外部类中的实例
变量。

如果静态内部类不是private,就可以被外部代码使用。

静态内部类可以被序列化,内部类不可以。

匿名内部类使用接口名加实现接口中的方法直接new出一个对象。但因为匿名内部类没有类
名,所以也没有办法定义一个构造函数出来,不过可以用初始化代码块来完成初始化工作(
用花括号括起来)。

内部类和匿名内部类不能访问局部变量,它所在的方法的参数就是典型的局部变量。为了
可以访问,我们常常把方法的参数设为final。因为内部类存在可能超过声明它的方法之外
。而局部变量在离开代码块后就不存在了的。所以要声明为final才能被内部类访问。


\section{适配器(Adapter)}

由于在写匿名内部类时,要实现接口所有抽象方法的实现。这样会把内部类代码写得很长,
为了在写内部类时简短一点,可以先写一个类实现对应的接口,但这个类实现的方法内容为
空:方法中什么事也不做,或仅仅为了符合语法而return null。这样一个有等于没有的类
被叫作适配器。以后根据它来写内部类,就只要实现你要用到的方法就可以了,其他的方法
虽然没有内容,但是在语法上已经合法了。

应用场景:第三方系统提供的接口Dct,只提供了我们接口,但还没有拿到实现,我们要先
根据接口进行测试,保证我们的调用没有问题:

\lstinputlisting[]{code/04.refect/src/example/Dct.java}

\lstinputlisting[]{code/04.refect/src/example/DctAdapter.java}

\lstinputlisting[]{code/04.refect/test/test/DctTest.java}



\section{反射}

通过Object的getClass方法可以在程序运行过程中判断一个对象的类型。一个类的成员方法
isAssignableFrom可以判断这个类是不是参数类的子类。Class类的静态方法forName可以根
据类名在运行期间生成一个类。

java.lang.Class类的getMethods、getConstructors、getFields等方法可以取得类的大部
分信息。如通过isInterface判断是接口还是类;getModifiers取得类的修饰符,比如常用
的final、public、abstract等都在Modifier类中有定义对应的常量。

\lstinputlisting[]{code/04.refect/src/example/RefectClassExp.java}

\lstinputlisting[]{code/04.refect/test/test/ClassRefTest.java}

\section{代理模式}

java.lang.reflectProxy可以提供动态代理类,在运行时动态实现一个或多个接口。



\subsection{代理应用场景}

对于一个账号服务我们抽象出接口AccountService。它有两个实现:AccountServiceProxy
在客户端,AccountServiceImpl在服务器端。服务器端还有一个可能接收请示的ServerStub
。

客户端程序只知道调用了AccountService接口的方法,但实际上生产出来的类是
AccountServiceProxy。它的工作是与服务器端的ServiceStub通讯。ServiceStub会分析对
应的请示,调用具体实现功能的AccountServiceImpl处理业务逻辑。ServiceStub再把处理
结果返回给客户端的AccountServiceProxy。

整个过程在客户端程序看起来好像是在调用本地代码一样。



\subsection{通过代理实现权限控制}

我们要通过权限类与异常来控制程序:

\lstinputlisting[]{code/04.refect/src/sis/security/Permission.java}

\lstinputlisting[]{code/04.refect/src/sis/security/PermissionException.java}

首先抽象出业务逻辑的接口并实现类:

\lstinputlisting[]{code/04.refect/src/sis/studentinfo/Accountable.java}

\lstinputlisting[]{code/04.refect/src/sis/studentinfo/Account.java}

现在能够实现业务逻辑的是Account类,但它没有实现权限控制的功能。所以我们创建一个
代理类SecureProsy来管理权限。它的构造函数接收两个参数:

一个参数为具体实现业务逻辑的类,这样就可以通过反射执行具体的业务。

另一个参数是一组方法名,指定哪些功能是要通过授权才能访问的。如果要调用这些方法,
就直接抛出异常。

具体代码如下:

\lstinputlisting[]{code/04.refect/src/sis/security/SecureProxy.java}

现在创建一个工厂类,根据传入的权限级别来创造一个有权限限制的SecureProxy或原始的
Account业务类:

\lstinputlisting[]{code/04.refect/src/sis/studentinfo/AccountFactory.java}


\end{document}
