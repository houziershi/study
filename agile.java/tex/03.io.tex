\chapter{输入输出}



\section{字符流}

Writer接口提供了对字符流的输出。下面的例子接受一个writer用来输出文本,而不关心具体输出到哪里:

\lstinputlisting[]{code/03.IO/src/example/ReportExample.java}

\lstinputlisting[]{code/03.IO/test/test/ReportTest.java}



\subsection{文本文件}

把输出写入一个文本文件中:

\lstinputlisting[]{code/03.IO/src/example/FileExample.java}

测试检查文件有内容是否正确,当然在操作之前与之后要删除文件:

\lstinputlisting[]{code/03.IO/test/test/ReportTest.java}

其他File类的常用方法:

创建临时文件:createTempFIle

创建空白文件:createNewFile

文件操作:delete、deleteOnExit、renameTo

查询文件或路径名:getAbsoluteFile、getAbsolutePath、getCanonicalFile、
getCanonicalPath、getName、getPath、toURI、toURL

级别:isFile、isDirectory

属性操作:isHidden、lastModified、length、canRead、canWrite、setLastModified、
setReadOnly

目录操作:exists、list、listFiles、listRoots、mkdir、mkdirs、getParent、
getParentFile



\subsection{字节流的转换}

InputStreamReader和OutputStreamWriter包装了InputStream和OutputStream。将其转换为
字符流。





\section{数据流}



\subsection{基本数据类型}

DataInputStream和DataInputStream提供了基本类型的读写操作,如:readInt、
writeint等:

\lstinputlisting[]{code/03.IO/src/example/BaseTypeExample.java}

测试检查文件有内容是否正确,当然在操作之前与之后要删除文件:

\lstinputlisting[]{code/03.IO/test/test/BaseTypeTest.java}



\subsection{序列化对象}

实现了\verb|java.io.Serializable|接口的类(包括子类)能够被序列化。对于类中不
需要被序列化的成员可以通过关键字“transient”来修饰。加上serialVersionUID来记录类
的版本变化(不同的版本中可能会改变类的成员):

\lstinputlisting[]{code/03.IO/src/example/LogRecSe.java}

直接使用ObjectInputStream与ObjectOutputStream来读写对象:

\lstinputlisting[]{code/03.IO/src/example/ObjectExample.java}

\lstinputlisting[]{code/03.IO/test/test/ObjectSaveTest.java}
