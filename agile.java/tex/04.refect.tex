\chapter{类、接口与反射}


\section{内部类}

内部类可以访问定义在外部类中的实例变量;静态内部类不可以访问定义在外部类中的实例
变量。

如果静态内部类不是private,就可以被外部代码使用。

静态内部类可以被序列化,内部类不可以。

匿名内部类使用接口名加实现接口中的方法直接new出一个对象。但因为匿名内部类没有类
名,所以也没有办法定义一个构造函数出来,不过可以用初始化代码块来完成初始化工作(
用花括号括起来)。

内部类和匿名内部类不能访问局部变量,它所在的方法的参数就是典型的局部变量。为了
可以访问,我们常常把方法的参数设为final。因为内部类存在可能超过声明它的方法之外
。而局部变量在离开代码块后就不存在了的。所以要声明为final才能被内部类访问。


\section{适配器(Adapter)}

由于在写匿名内部类时,要实现接口所有抽象方法的实现。这样会把内部类代码写得很长,
为了在写内部类时简短一点,可以先写一个类实现对应的接口,但这个类实现的方法内容为
空:方法中什么事也不做,或仅仅为了符合语法而return null。这样一个有等于没有的类
被叫作适配器。以后根据它来写内部类,就只要实现你要用到的方法就可以了,其他的方法
虽然没有内容,但是在语法上已经合法了。

应用场景:第三方系统提供的接口Dct,只提供了我们接口,但还没有拿到实现,我们要先
根据接口进行测试,保证我们的调用没有问题:

\lstinputlisting[]{code/04.refect/src/example/Dct.java}

\lstinputlisting[]{code/04.refect/src/example/DctAdapter.java}

\lstinputlisting[]{code/04.refect/test/test/DctTest.java}



\section{反射}

通过Object的getClass方法可以在程序运行过程中判断一个对象的类型。一个类的成员方法
isAssignableFrom可以判断这个类是不是参数类的子类。Class类的静态方法forName可以根
据类名在运行期间生成一个类。

java.lang.Class类的getMethods、getConstructors、getFields等方法可以取得类的大部
分信息。如通过isInterface判断是接口还是类;getModifiers取得类的修饰符,比如常用
的final、public、abstract等都在Modifier类中有定义对应的常量。

\lstinputlisting[]{code/04.refect/src/example/RefectClassExp.java}

\lstinputlisting[]{code/04.refect/test/test/ClassRefTest.java}

\section{代理模式}

java.lang.reflectProxy可以提供动态代理类,在运行时动态实现一个或多个接口。



\subsection{代理应用实例}

对于一个账号服务我们抽象出接口AccountService。它有两个实现:AccountServiceProxy
在客户端,AccountServiceImpl在服务器端。服务器端还有一个可能接收请示的ServerStub
。

客户端程序只知道调用了AccountService接口的方法,但实际上生产出来的类是
AccountServiceProxy。它的工作是与服务器端的ServiceStub通讯。ServiceStub会分析对
应的请示,调用具体实现功能的AccountServiceImpl处理业务逻辑。ServiceStub再把处理
结果返回给客户端的AccountServiceProxy。

整个过程在客户端程序看起来好像是在调用本地代码一样。


AccountServiceProxy。它不能提供服务,但能和服务器通讯上的

