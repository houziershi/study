
\chapter{模块}
	
	\section{模块与路径}
		
		在linux环境下查找python的安装目录:

\begin{lstlisting}
whereis python
\end{lstlisting}

		在python交互环境中查看python路径:

\begin{lstlisting}
sys.path
\end{lstlisting}

	\section{导入模块}

		导入模块使用\verb|import|语句:

\begin{lstlisting}
import module01
\end{lstlisting}

		当模块被导入后,会包含源文件的目录信息:<目录>/<文件名>.<扩展名>。

		一个模块只能有一个实例,导入一个模块以后不能再次导入。当一个模块的代码被修改以后必须使用\verb|reload|语句重新加载模块才能生效:

\begin{lstlisting}
reload module01
\end{lstlisting}

		如果只导入模块中的部分变量,则使用\verb|from|语句:
		
\begin{lstlisting}
from module01 import a, b, c
\end{lstlisting}

		\verb|from|语句不导入模块,只复制模块中的变量到本地。复制模块中全部变量的例子:

\begin{lstlisting}
from module01 import *
\end{lstlisting}

		\verb|import|与\verb|from|语句都都有赋值效果,例如:

\begin{lstlisting}
from module1 import a, b, c
\end{lstlisting}

		等价于:

\begin{lstlisting}
import module1

a = module1.a
b = module1.b
c = module1.c
\end{lstlisting}

%		\lstinputlisting[label=ch12:baseOOP, caption=类与实例]{py/expoo00.py}


	\section{直接执行脚本文件}
		
		可以不加载模块而是直接以脚本文件的方式执行:

\begin{lstlisting}
execfile('mymodule.py')
\end{lstlisting}
