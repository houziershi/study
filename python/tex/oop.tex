\chapter{OOP与类}

一个模块只能有一个实例,当一个模块的代码被修改以后必须重新加载模块才能生效;类可以同时创建多个实例(即对象)。

\section{最简单的Python类}

Python的类模型相当动态,类与实例只是命名空间对象。

可以仅用pass作为占位语句生成一个没有任何成员的class,这样的class仅仅是一个空的命名空间对象。可以在以后通过赋值给这个类加上新的成员:

\begin{lstlisting}
class C1: pass             # Empty class
                           # as an namespace object

a = C1()
a.name = "morgan"          # a.name is "morgan"

C1.name = "Class C1"       # C1.name is "C1"
b = C1()                   # b.name is "C1"
                           # a.name still is "morgan"

def upperName(self):
	print self.name.upper()

C1.showName = upperName    # add func to class
a.showName()
b.showName()
\end{lstlisting}

\section{类与实例的基本概念}

类与类的实例都是对象。类是一个对象,该类的每个实例又各对应了一个关联到该类对象的实例对象。

像调用函数一样调用类对象会创建该类的实例对象。每个实例对象根据具体类的创建属性并获得自己的命名空间,每个成员都有自己的实例。

对象的数据成员(无论是数据还是函数)在没有被第一次赋值之前,都不能被访问(就像是没有被声明的变量一样)。同样地也可以简章地通过赋值给变量增加一个成员。

class语句会创建一个类对象并赋值给变量名。class语句块内的语句会创建数据对象和方法对象赋值给类的成员变量。类的成员变成只属于该类,不属于该类的实例,但可以被所有的实例共享访问。

\begin{lstlisting}
class Ca:
	v = 42

a = Ca()   # a.v = 42
b = Ca()   # b.v = 42

Ca.v = 55  # Ca.v = 55
           # a.v = 55
		   # b.v = 55

a.v = 11   # Ca.v = 55
           # a.v = 11
		   # b.v = 55
\end{lstlisting}

类中的方法被所有函数共享。通过实例调用方法时,会把这个实例对象作为第一个参数self传递给方法。

\begin{lstlisting}
class c1():
	count = 0
	def show(self):
		print 'c1.show()'
\end{lstlisting}

类的构造函数是\verb|__init__|;析构函数是\verb|__del__|:

\begin{lstlisting}
class Life:
	def __init__(self, name='unknow'):
		print 'hello ', name
		self.name = name
	def __del__(self):
		print 'goodbye ', self.name

brian = Life('Brian')     # prints: hello Brian
brian = 'loretta'         # prints: goodbye Brian
\end{lstlisting}

		实例对象的\verb|__class__|属性记录了这个实例的类对象。

\begin{lstlisting}
print a.__class__           # class of instance
\end{lstlisting}

\section{类的成员}

\subsection{属性与方法}

python中对于类和对象,无论数据还是方法都作为变量处理。对象的数据成员(无论是数据还是函数)在没有被第一次赋值之前,都不能被访问(就像是没有被声明的变量一样)。同样地也可以简单地通过赋值给变量增加一个成员。

\begin{lstlisting}
class Employee():
	def __init__(self, name):   # __init__ defing construter
		self.name = name
	def showName(self):
		print self.name

morgan = Employee("Morgan")
morgan.showName()

morgan.showName = "new name"    # set value
morgan.nickName = "Jade"        # create new variable
\end{lstlisting}

类对象与实例对象的\verb|__dict__|属性是大多数基于类的对象的命名空间字典。

\begin{lstlisting}
print Employee.__dict__.keys()  # 大多数基于类的对象的命名空间字典
print morgan.__dict__.keys()    # 大多数基于类的对象的命名空间字典
\end{lstlisting}

\subsection{类属性与对象属性}

修改了类的属性以后会影响到所有的类实例:

\begin{lstlisting}
class C:
	a = 1

print C.a     # is 1
o = C()
print o.a     # is 1

C.a = 2
print o.a    # is 2, also change too

j = X()
print j.a   # is 2
\end{lstlisting}

\subsection{方法与对象的绑定}

对于以下的类:

\begin{lstlisting}
class Spam:
	def doit(self, message):
		print message
\end{lstlisting}

可以通过新建对象调用:

\begin{lstlisting}
obj = Spam()
obj.doit('hello world')
\end{lstlisting}

已经绑定的方法可以再赋值给一个变量,效果和对象点号调用一样:

\begin{lstlisting}
x = obj.doit
x('hello world')
\end{lstlisting}

但是类中的方法是没有绑定到对象的,要传入对象:

\begin{lstlisting}
t = Spam.doit
t(obj, 'hello world')
\end{lstlisting}

同理,如果要引致类中的函数,相同的规则也适用于类的方法:

\begin{lstlisting}
class Eggs:
	def m1(self, n):
		print n

	def m2(self):
		x = self.m1  # bound to object
		x(42)

Eggs().m2()
\end{lstlisting}

\section{类的继承}

类对象的成员\verb|__bases__|是超类构成的元组

\begin{lstlisting}
print c3.__bases__              # 超类的元组
\end{lstlisting}

通过\verb|Super|调用超类:

\begin{lstlisting}
class Super:
	def __init__(self):
		pass

class Sub(Super):
	def __init__(self):
		Super.__init__(self):
			pass
\end{lstlisting}

\subsection{多重继承}

子类会继承父类的成员。如果当多重继承时遇到重名成员时顺序就很重要,优先级按声明继承时从左到右顺序:

\begin{lstlisting}
class c1():
	myname = 'aaa'
	def show(self):
		print 'c1.show()'
	def helloC1(self):
		print 'c1.helloC1()'

class c2():
	def show(self):
		print 'c2.show()'
	def helloC2(self):
		print 'c2.helloC2()'

class c3(c1, c2):               # 超类写在括号中,支持多重继承
	def show(self):
		print 'c3.show()'

c1.myname                       # c1.myname is 'aaa'
c3.myname                       # c3.myname is 'aaa'
\end{lstlisting}

对象成员赋值后不影响类成员的值:

\begin{lstlisting}
i1 = c3()                       # i1.myname is 'aaa'
i2 = c3()                       # i2.myname is 'aaa'

i1.myname = 'this is i1'        # i1.myname is 'this is i1'
i2.myname = 'this is i2'        # i2.myname is 'this is i2'
print c1.myname                 # c1.myname still is 'aaa'
print c3.myname                 # c3.myname still is 'aaa'

i1.show()                       # 调用重写的方法
i1.helloC1()                    # 调用到父类的方法
i1.helloC2()                    # 调用到父类的方法
\end{lstlisting}

虽然顺序很重要,但是也可以手动指定要调用哪一个:

\begin{lstlisting}
class A:
	def __repr__(self): pass
	def other(self): pass

class B:
	def __repr__(self): pass
	def other(self): pass

class C(A, B):
	__repr__ = A.__repr__
	other = B.other
\end{lstlisting}

\subsection{压缩成员变量名}

常见的变量名重复的情况:

\begin{lstlisting}
class C1:
	def meth1(self): self.x = 88
	def meth2(self): print self.x

class C2:
	def metha(self): self.x = 99
	def methb(self): print self.x

class C3(C1, C2): pass

I = C3()   # only 1 x in I !!!
\end{lstlisting}

在这样的情况下,x只有一个。两个类的四个方法读写的都是同一个x成员。这个x是值取决最后是哪一个方法调用的它。 

在\verb|__class__|语句中以两个下划线开头的变量\verb|__verb|会扩展成包含类名的新变量名\verb|_classname__verb|。这样就不会和同一层次中其他类所创建的类似变量名相冲突。

\begin{lstlisting}
class C1:
	def meth1(self): self.__x = 88
	def meth2(self): print self.__x

class C2:
	def metha(self): self.__x = 99
	def methb(self): print self.__x

class C3(C1, C2): pass

I = C3()   # two x in I: _C1__x and _C2__x

I.meth1()
I.meth2()  # result is 88

I.metha()
I.methb()  # result is 99
\end{lstlisting}


\subsection{继承的应用场景}

一般由超类提供默认的方法或是等待实现的方法:

\begin{lstlisting}
class Super:
	def method(self):
		print 'Super.method()'           # default behavior
	def delegate(self):
		self.action()                    # expected to defined
\end{lstlisting}

子类可以不去实现超类:

\begin{lstlisting}
class Inheritor(Super):
	pass                                 # inherit method verbatim
\end{lstlisting}

子类可以把超类的方法完全覆盖掉:

\begin{lstlisting}
class Replacer(Super):
	def method(self):
		print 'Replacer.method()'        # replace method completely
\end{lstlisting}

子类可以扩展超类的方法:

\begin{lstlisting}
class Extender(Super):
	def method(self):                    # extend method behavior
		print 'Extender.method() start'
		Super.method(self)
		print 'Extender.method() end'
\end{lstlisting}

子类可以实现超类有待实现的方法:
		
\begin{lstlisting}
class Provider(Super):                   # fill in required method
	def action(self):
		print 'Provider.action'
\end{lstlisting}

在迭代中创建类:

\begin{lstlisting}
if __name__ == '__main__':
	for clazz in (Inheritor, Replacer, Extender):
		print '\n' + clazz.__name__ + '...'
		clazz().method()
	print '\nProvider...'
	p = Provider()
	p.delegate()
\end{lstlisting}

\section{命名空间}

\subsection{命名空间范围}

python中作用域是由源代码中赋值语句位置来决定的,不会受到导入关系的影响。通过下面的例子总结了python的命名空间:

\lstinputlisting[]{py/chap.oop/manynames.py}

需要注意的是在调用I.m()之前实例中是不存在成员变量X的。实例的成员变量也是变量,在执行到赋值语句之前是不会存在的。

在外部文件中使用:

\lstinputlisting[]{py/chap.oop/otherfile.py}

\subsection{命名空间字典}

无论是模块、对象还是实例,它们的命名空间实际都是以字典形式实现的。这个字典就是对象中的成员\verb|__dict__|变量。

\begin{lstlisting}
>>> class super:
...     def hello(self):
...             self.data1 = 'spam'
... 
>>> class sub(super):
...     def hola(self):
...             self.data2 = 'eggs'
... 
>>> X = sub()
>>> X.__dict__
{}
>>> X.__class__
<class __main__.sub at 0x7f35b8248ef0>
>>> sub.__bases__
(<class __main__.super at 0x7f35b8248e88>,)
>>> super.__bases__
()
\end{lstlisting}

以\verb|self|为目标赋值会把成员加入到对象,而不是类中:

\begin{lstlisting}
>>> Y = sub()

>>> X.hello()
>>> X.__dict__
{'data1': 'spam'}

>>> X.hola()
>>> X.__dict__
{'data1': 'spam', 'data2': 'eggs'}

>>> sub.__dict__
{'__module__': '__main__', '__doc__': None, 'hola': <function hola at 0x7f35b8201668>}

>>> super.__dict__
{'__module__': '__main__', 'hello': <function hello at 0x7f35b82015f0>, '__doc__': None}

>>> Y.__dict__
{}
\end{lstlisting}

可以用命名空间字典部分实现点号操作,但是字典不带继承搜索:

\begin{lstlisting}
>>> X.__dict__['data1']
'spam'

>>> X.__dict__['data3'] = 'ham'
>>> X.data3
'ham'
\end{lstlisting}

\subsection{命名空间链接}

每个类的实例,都有成员变量\verb|__class__|连接到它的类;

而在类中有成员变量\verb|__bases__|是一个元组,包含了到超类的链接。

\section{运算符重载}

两头是双下划线的方法名(\verb|__方法名__|)表示对运算符的重载。

\subsection{重载字符串化方法}

当要把实例字符串化时会先使用\verb|__str__|。这样产生的结果在终端下用户体验较好; 如果没有定义,会再调用\verb|__repr__|,它产生的结果可以作为代码直接重建该对象。

\begin{lstlisting}
class adder:
	def __init__(self, value=0):
		self.data = value
	def __str__(self):
		return '[value is: %s]' % self.data
	def __repr__(self):
		return 'addrepr(%s)' % self.data

x = adder(2)
print x          # [value is: 2]
print str(x)     # [value is: 2]
print repr(x)    # addrepr(2)
\end{lstlisting}

\subsection{常用数学方法}

\verb|__add__|表示重载\verb|+|运算。

\verb|__sub__|表示重载\verb|-|运算。

\verb|__mul__|表示重载\verb|*|运算。

对于没有定义或是继承的操作符,表示该操作不被支持。

\begin{lstlisting}
class Number():
	def __init__(self, value):
		self.data = value
	def __add__(self, other):
		return C1(self.data + other)
	def __sub__(self, other):
		return C1(self.data - other)
	def __mul__(self, other):
		return C1(self.data * other)

a = Number(5)       # Number.__init__(a,5)
b = a - 2           # Number.__sub__(a,2)
\end{lstlisting}

以上的方法中,操作符右边的不能是实例对象。还有一个相反的左边不是对象右边是对象的\verb|__radd__|方法:

\begin{lstlisting}
class Computer:
	def __init__(self, val):
		self.val = val
	def __add__(self,other):
		self.val += other
	def __radd__(self, other):
		self.val = self.val + other

x = Computer(2)
x + 3
print x.val

y = Computer(3)
3 + y
print y.val
\end{lstlisting}

\subsection{拦截索引}

索引操作可以通过\verb|__getitem__|方法拦截:

\begin{lstlisting}
class indexer:
	def __getitem__(self,index):
		return index ** 2

x = indexer()
x[2]                      # __getitem__(x,2)  result 4
\end{lstlisting}

索引操作可以模拟迭代效果:

\begin{lstlisting}
class stepper:
	def __getitem__(self,index):
		return self.data[i]

x = stepper()
x.data = "spam"

for item in x:
	print item
\end{lstlisting}

\subsection{拦截迭代}

迭代操作可以通过\verb|__iter__|方法拦截,python在遇到迭代环境时会先尝试迭代,如果对象不支持迭代,就会尝试索引运算:

\begin{lstlisting}
class Squares:
	def __init__(self, start, stop):    # save state when created
		self.value = start -1
		self.stop = stop
	def __iter__(self):          # get iterator object on iter()
		return self
	def next(self):              # return a square on each iteration
		if self.value == self.stop:
			raise StopIteration
		self.value += 1
		return self.value ** 2

for i in Squares(1,5):   # for calls iter(), which calls __iter__()
	print i              # Each iteration calls next()
	                     # get result: 1 4 9 16 25
\end{lstlisting}

\verb|__iter__|只循环一次,循环以后就变为空,每次新的循环就必须重新创建一个新的迭代器对象。

\begin{lstlisting}
x = Squares(1,5)
[n for n in x]            # [1,4,9,16,25]
[n for n in x]            # [] now it's empty
[n for n in Squares(1,5)] # [1,4,9,16,25]
list(Squares(1,3))        # [1,4,9]
\end{lstlisting}

另外一个例子,迭代时跳过下一下元素:

\begin{lstlisting}
class SkipIterator:
	def __init__(self, wrapped):
		self.wrapped = wrapped  # iter state flg
		self.offset = 0
	def next(self):
		if self.offset >= len(self.wrapped):
			raise StopIteration
		else:
			item = self.wrapped[self.offset]
			self.offset += 2
			return item

class SkipObject:
	def __init__(self,wrapped):
		self.wrapped = wrapped
	def __iter__(self):
		return SkipIterator(self.wrapped)

if __name__ == '__main__':
	skipper = SkipObject('0123456789')
	i = iter(skipper)   # make an iterator on it
	print i.next(), i.next(), i.next()
	
	# in earch nest create new iterator
	for c in skipper:
		for d in skipper:
			print c+d
\end{lstlisting}

在最后的嵌套的两个\verb|for|循环中,每个循环都会建立自己的迭代器,相互之间不会混淆。这相的操作相当于:

\begin{lstlisting}
str = 'abcdefg'
for x in s[::2]
	for y in s[::2]
		print x+y
\end{lstlisting}

这里迭代与分片的区别是。分片把结果的整个列表放到了内存里,而迭代器每次产一个值;分片会产一个新的对象,并不同于迭代是对原来的对象进行r操作。

\subsection{拦截成员属性}

\verb|__getattr__|对成员属性的点号操作时行拦截(多数情况下是对没有定义的成员属性):

\begin{lstlisting}
class empty:
	height = 189
	def __getattr__(self,name):
		if attrname == "age":
			return 40

x.age          # 40
x.height       # 189
x.name         # None
\end{lstlisting}

其他属性抛出异常:

\begin{lstlisting}
class empty:
	height = 189
	def __getattr__(self,name):
		if attrname == "age":
			return 40
		else:
			raise AttributeError, attrname

x.age          # 40
x.height       # 189
x.name         # exception
\end{lstlisting}

\verb|__setattr__|对成员属性的赋值操作进行拦截。但有个问题:即使在\verb|__setattr__|内的\verb|self.attr=value|也会再次被拦截,造成无限循环。所以在拦截函数内要通过:\verb|self.__dict__['name']=value|来赋值:

\begin{lstlisting}
class AccessControl:
	def __setattr__(self,attr,value):
		if attr == 'age':
			self.__dict__[attr] = value
		else:
			raise AttributeError, attr + ' not allowed'

x = AccessControl()
x.age = 40
\end{lstlisting}

\subsection{模拟私有成员}

\begin{lstlisting}
class PrivateExc(Exception):pass

class Privacy:
	def __setattr__(self, attrname, value):
		if attrname in self.privates:
			raise PrivateExc(attrname, self)
		else:
			self.__dict__[attrname] = value

class Test1(Privacy):
	privates = ['age']

class Test2(Privacy):
	privates = ['name','pay']
	def __init__(self):
		self.__dict__['name'] = 'Tom'
\end{lstlisting}

\subsection{拦截调用}

如果定义了\verb|__call__|方法,实例的调用就会被该方法拦截:

\begin{lstlisting}
class Prod:
	def __init__(self,value):
		self.value = value
	def __call__(self,other):
		return self.value * other

x = Prod(2)   # x.value is 2
x(3)          # return value is 2*3=6
x(4)          # return value is 2*4=8
\end{lstlisting}

主要的应用场合是作为回调函数调用,比如在GUI程序中:

\begin{lstlisting}
class CallBack:
	def __init__(self,color):
		self.color = color
	def show(self):
		print 'color: ' + self.color

a = CallBack('red')
b = CallBack('blue')

b1 = Button(command=a)
b2 = Button(command=b)
\end{lstlisting}

除了绑定对象还可以直接绑定到方法:

\begin{lstlisting}
b1 = Button(command=a.show)
b2 = Button(command=b.show)
\end{lstlisting}

lambda表达式也经常用来实现回调函数的功能:

\begin{lstlisting}
c = (lambda color='red' : 'color is: '+color)
print c()
\end{lstlisting}

\section{类的设计}

\subsection{通用对象工厂}

python中的对象工厂要比强类型语言中简单得多:

\begin{lstlisting}
def factory(clazz, *args):
	return apply(clazz, args)

class Spam:
	def doit(self,message):
		print message

class Person:
	def __init__(self, name, job):
		self.name = name
		self.job = job

obj1 = factory(Spam)
obj2 = factory(Person, "Guido", "guru")
\end{lstlisting}

\section{静态方法和类方法}

最典型的静态方法应用场景是统计一个类产生了多少实例:

\begin{lstlisting}
class Spam:
	count = 0
	def __init__(self):
		Spam.count = Spam.count +1
	def printCount():
		print "count: " , Spam.count
\end{lstlisting}

但是这样是行不通的。在调用\verb|printCount()|时一定要传入一个实例作为参数,无论你在\verb|def|定义函数是有没有写明\verb|self|。虽然可以给它绑定一个对象,但这样有一个缺点就是一定要通过对象来调用这个函数。

相对方便一点的方法是它移动到类外,作为一个模块下的简单函数。例:

\begin{lstlisting}
class Spam:
	count = 0
	def __init__(self):
		Spam.count = Spam.count +1

def printCount():
	print "count: ", Spam.count
\end{lstlisting}

这样类和函数都是模块命名空间下的变量。也不会和其他文件中的变量名冲突:

\begin{lstlisting}
import spam

a = spam.Spam()
b = spam.Spam()
c = spam.Spam()

spam.printCount()  # module.function : result is 3
spam.Spam.count    # module.class.verb : result is 3
\end{lstlisting}

\subsection{使用静态方法和类方法}

可以通过内置函数\verb|staticmethod|和\verb|classmethod|来生成静态方法和类方法。这两都都不需要在启用时传入实例作为参数。

\begin{lstlisting}
class Multi:
	def imeth(self,x):    # normal instance method
		print self, x
	def smeth(x):         # static: no instance passed
		print x
	def cmeth(clazz,x):   # Class: get class, not instance
		print clazz, x
	smeth = staticmethod(smeth) # make smeth a static method
	cmeth = classmethod(cmeth)  # make cmeth a class method
\end{lstlisting}

实例方法、静态方法、类方法的使用:

\begin{lstlisting}
obj = Multi()
obj.imeth(1)         # normal call
Multi.imeth(obj, 1)  # normal call

Multi.smeth(3)       # static call, through class
obj.smeth(4)         # static call, through instance

Multi.cmethod(5)     # class call, throw class
obj.cmethod(5)       # class call, throw instance
\end{lstlisting}

\subsection{函数装饰器}

函数包装器可以在函数外面包上层逻辑。如静态函数可以这样写:

\begin{lstlisting}
class C:
	@staticmethod
	def method(): pass
\end{lstlisting}

相当于前面的静态函数实现:

\begin{lstlisting}
class C:
	def method(): pass
	method = staticmethod(method)
\end{lstlisting}

实际上装饰器语法可以包上多层逻辑:

\begin{lstlisting}
@A @B @C
def f(): pass
\end{lstlisting}

相当于前面的静态函数实现:

\begin{lstlisting}
def f(): pass
f = A(B(C(f)))
\end{lstlisting}

\subsection{函数装饰器例子}

\begin{lstlisting}
class tracer:
    def __init__(self,func):
            self.calls = 0
            self.func = func
    def __call__(self, *args):
            self.calls += 1
            print 'call %s to %s' % (self.calls, self.func.__name__)
            self.func(*args)

@tracer
def spam(a, b, c):
    print a, b, c
\end{lstlisting}

用\verb|__call__|重载方法替实例实现函数调用接口。因为spam函数是通过tracer装饰器执行的,所以实际上调用的是类中的\verb|__call__|方法。

\begin{lstlisting}
spam(1,2,3)       # call 1 to spam
                  # 1 2 3

spam('a','b','c') # call 2 to spam
                  # a b c

spam(4,5,6)       # call 3 to spam
                  # 4 5 6
\end{lstlisting}

\begin{lstlisting}
\end{lstlisting}

\begin{lstlisting}
\end{lstlisting}

\begin{lstlisting}
\end{lstlisting}

\begin{lstlisting}
\end{lstlisting}

