
\chapter{语句顺序}

	\section{if语句}

		基本结构:
		
\begin{lstlisting}
if ... :
	...
elif ... :
	...
else
	...
\end{lstlisting}

		\subsection{推荐风格}

			对于过长的语句可以使用“\verb|\|”换行:

\begin{lstlisting}
if a==b and b==c \
	 and c==d :
	 print 'all equal!'
\end{lstlisting}

			但推荐还是用“\verb|()|”包起来在风格上更好:

\begin{lstlisting}
if (a==b and b==c
	 and c==d):
	 print 'all equal!'
\end{lstlisting}

	\section{三元表达式}

\begin{lstlisting}
<normal> if <condition> else <other>
\end{lstlisting}
	

	\section{用字典实现判断}

		python中的字典也可以部分实现if-else的功能。因为字典不但可以用函数作为成员,而且取值操作时,可以指定找不到值的默认返回值:

\begin{lstlisting}
myDict.get('aaa','no value');  # default value if key not exist
\end{lstlisting}

	\section{while循环语句}

\begin{lstlisting}
while ... :
	...      # 执行的语句
	...      # 遇到 break 跳出整个循环
	...      # 遇到 continue 回到开头进行
	...      # pass 语句什么也不做,仅在语法上点一个位置
else :
	...      # 只有在循环完全执行后才会运行(没有遇到break)
\end{lstlisting}

	\section{for循环语句}

\begin{lstlisting}
for ... in ... :
	...      # 执行的语句
	...      # 遇到 break 跳出整个循环
	...      # 遇到 continue 回到开头进行
	...      # pass 语句什么也不做,仅在语法上点一个位置
else :
	...      # 只有在循环完全执行后才会运行(没有遇到break)
\end{lstlisting}
		
		\subsection{for循环例子}

			用\verb|xreadlines()|一行一行地读入文本文件,防止文件过大:

\begin{lstlisting}
for line in open('log.txt').xreadlines():
	print line
\end{lstlisting}

			用推荐自动用速度快而内存利用最合理的方法:

\begin{lstlisting}
for line in open('log.txt'):
	print line
\end{lstlisting}

			用\verb|range()|函数生成列表的方法:

\begin{lstlisting}
range(5)        # [0,1,2,3,4]
range(2, 5)     # [2,3,4]
range(0, 10, 2) # [0,2,4,6,8]
\end{lstlisting}

			结合\verb|range()|函数与for循环:

\begin{lstlisting}
for x in range(5, -5, 2):
	print x

for i in range(len('abc')):
	print 'abc'[i]
\end{lstlisting}

	\section{迭代器}

		\subsection{通用迭代器}

			python中所有的对象都可以被迭代,迭代的工具是通用迭代器\verb|next()|。当没有可迭代的项时接收\verb|StopIteration|异常决定离开。

		\subsection{文件迭代}

			文件迭代可以使用包装过了\verb|reanline()|。它会的文件结尾返回空字符串\verb|''|:

\begin{lstlisting}
file = open('log.txt')
file.readline()
file.readline()
file.readline()
...
file.readline()        # return '' at end of file
\end{lstlisting}

		\subsection{常见的迭代器用法}

			处理每个成员
\begin{lstlisting}
list = [line.upper() for line in lines]
\end{lstlisting}

			map()函数,声明用指定的方法处理每个成员

\begin{lstlisting}
map(str.upper, lines)         
\end{lstlisting}

			成员处理

\begin{lstlisting}
'3' in lines      # 是否包含成员
sort(lines)
sum([1,2,3,4,5])
any(['a','','c'])
all(['a','','c'])
\end{lstlisting}

			转换类函数

\begin{lstlisting}
list(line)        # 转为列表
tuple(line)       # 转为元组
\end{lstlisting}

			合并多个可迭代序列

\begin{lstlisting}
zip('ABC','abc','123') 
# result (('A','a','1'), ('B','b','2'), ('C','c','3'))
\end{lstlisting}

			enumerate()把列表解析为“下标”与“值”的形式

\begin{lstlisting}
items = enumerate('abc')
items.next()             # result is (0,'a')
\end{lstlisting}

			结合enumerate()与for循环

\begin{lstlisting}
for (index, value) in enumerate('abc')
\end{lstlisting}

	\section{列表解析}

		列表解析以中括号围起的单行for循环的形式编写:

\begin{lstlisting}
[... for ... in ...]
\end{lstlisting}

		一般来说,列表解析会比用for循环更快。原因是在底层有优化,它是以C运行的。在速度上“列表解析”快于“map”快于“for循环”。

		\subsection{列表解析与for循环结合的例子}
			
			读取文件并且去除最后的换行符:

\begin{lstlisting}
lines = [line.rstrip() for line in open('log.txt')]
\end{lstlisting}

			组合for与if读取以“p”开头的行:

\begin{lstlisting}
lines = [line.rstrip() for line in open('log.txt') if line[0]=='p'] 
\end{lstlisting}

			列表解析也可以嵌套:

\begin{lstlisting}
[x+y for x in 'abc' for y in '123']
# result is ['a1','a2','a3' ... 'c1','c2','c3']
\end{lstlisting}

			列表解析与多维数组结合:

			\lstinputlisting[firstline=13,lastline=26]{py/coreda/exlist.py}
