
\chapter{常用功能}

	\section{输入输出}

		\subsection{python中的输出流}

\begin{lstlisting}
import sys

sys.stdout.write('hello\n')     # 输出到标准输出
sys.stderr.write('Error...\n')  # 输出到标准错误
\end{lstlisting}


		\subsection{print打印输出}

			\verb|print|语句会把对象打印到默认的输出流(标准输出)中:

\begin{lstlisting}
print a, b, b...
\end{lstlisting}

			格式化打印为\verb|a => b|的效果:

\begin{lstlisting}
print '%s => %s' % ('a', 'b')
\end{lstlisting}

		\subsection{重定向print到其他输出流}
		
			方法一:用指定的输出流替换掉标准输出。这样有一个缺点是每次都要手动地打开与关闭输出流:
\begin{lstlisting}
import sys

sys.stdout = open('aa.txt', 'a');
print 'hello'
sys.stdout.close()
\end{lstlisting}

			方法二:可以在\verb|print|语句中指定输出流:
\begin{lstlisting}
import sys

log = open('log.txt', 'w')
print >> log, 'start', 1, 2, 3   # write to log file
log.close()
print >> sys.stderr, "err..."    # write to std err
\end{lstlisting}

	\section{正则表达式}
		
		\lstinputlisting[firstline=1,lastline=13]{py/basefunc/use.reg.py}
