
\chapter{常用功能}

	\section{输入输出}

		\subsection{python中的输出流}

\begin{lstlisting}
import sys

sys.stdout.write('hello\n')     # 输出到标准输出
sys.stderr.write('Error...\n')  # 输出到标准错误
\end{lstlisting}


		\subsection{print打印输出}

			\verb|print|语句会把对象打印到默认的输出流(标准输出)中:

\begin{lstlisting}
print a, b, b...
\end{lstlisting}

			格式化打印为\verb|a => b|的效果:

\begin{lstlisting}
print '%s => %s' % ('a', 'b')
\end{lstlisting}

		\subsection{重定向print到其他输出流}
		
			方法一:用指定的输出流替换掉标准输出。这样有一个缺点是每次都要手动地打开与关闭输出流:
\begin{lstlisting}
import sys

sys.stdout = open('aa.txt', 'a');
print 'hello'
sys.stdout.close()
\end{lstlisting}

			方法二:可以在\verb|print|语句中指定输出流:
\begin{lstlisting}
import sys

log = open('log.txt', 'w')
print >> log, 'start', 1, 2, 3   # write to log file
log.close()
print >> sys.stderr, "err..."    # write to std err
\end{lstlisting}

	\section{文件操作}
		
		\subsection{常用的文件操作}

			\begin{verbatim}
				input  = open('aa.txt')      # 输入文件 r读取(默认)
				str = input.read()           # 读取整个文件
				str = input.read(n)          # 读取多个字节
				str = input.readline()       # 读取多个字节
				lst = input.readlines()

				output = open('aa.txt','w')  # 输出文件 w写入
				output.write(str)            #
				output.writelines(lst)       #
				output.flush()
				output.close()

				anyFile.seek(n)              # move to n
			\end{verbatim}

		\subsection{文本读写}

			\lstinputlisting{py/chap.tools/io/exfile.py}


		\subsection{对象的存取}

			python的内置函数\verb|eval()|可以直接把字符串作为程序代码执行。所以可以用它来把文本变成python对象:

			\lstinputlisting{py/chap.tools/io/str2obj.py}

			但是\verb|eval()|函数有一个安全隐患:它会执行“任何”python代码……你懂的……

			另一种选择是使用可以用来存取几乎任何python对象的pickle模块:

			\lstinputlisting{py/chap.tools/io/usepickle.py}

		\subsection{二进制文件读写}
			
			struct模块能打包并解析二进制文件:

			\lstinputlisting{py/chap.tools/io/opbinfile.py}

	\section{正则表达式}
		
		\lstinputlisting[firstline=1,lastline=13]{py/chap.tools/regexp/use.reg.py}
