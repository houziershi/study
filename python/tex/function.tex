
\chapter{函数}

	\section{函数定义}

		函数都有返回值。没有return或yield语句的函数会返回一个None。

		python中,函数也是一个对象。\verb|def|语句是实时执行的,当执行到\verb|def|语句时就会定义一个函数对象,然后赋值给指定函数名变量:

\begin{lstlisting}
def myAdd(a, b):
	return a+b
\end{lstlisting}

		以上代码生成了一个函数对象,并且把这个对象赋值给了变量\verb|myAdd|。可以通过变量\verb|myAdd|调用这个函数:

\begin{lstlisting}
myAdd(1,2)
\end{lstlisting}

		也可以把函数再赋值给另一个变量:

\begin{lstlisting}
otherFunc = myadd
otherFunc(1,2)
\end{lstlisting}

		函数对象的生成与赋值是在运行时发生的。所以也可以在运行时创建函数:

\begin{lstlisting}
if a>3:
	otherFunc = funcA
if a<3:
	otherFunc = funcB
\end{lstlisting}
		
	\section{参数传递}

		函数参数没有参数类型限制,只要参数支持对应的操作就可以。如果参数不支持对应的操作会抛出异常。

		\subsection{默认参数}
		函数参数列表支持\verb|name=value|形式的默认值:

\begin{lstlisting}
def myAdd(a=1, b=2):
	return a+b
\end{lstlisting}

		函数的默认参数被认为是静态的,不会每次调用都生成一个新的对象。例:
\begin{lstlisting}
def myfunc(x=[]):
	x.addend(1)
	print x

myfunc([2])       # [2,1]
myfunc()       # [1]
myfunc()       # [1,1]
myfunc()       # [1,1,1]
myfunc()       # [1,1,1,1]
\end{lstlisting}

		为了让每次建立新的默认值,可以在函数内写赋值语句:
\begin{lstlisting}
def myfunc(x=None):
	if x is None:
		x = []
	x.addend(1)
	print x
\end{lstlisting}


		\subsection{可变参数}
			python支持可变参数。当定义函数时,\verb|*|表示列表、\verb|**|表示字典:

\begin{lstlisting}
def func(a, * pargs, ** kargs):
    print a, pargs, kargs

func(1,2,3,x=1,y=2)           # 1 (2, 3) {'y': 2, 'x': 1}
\end{lstlisting}

			当调用函数时,\verb|*|表示把列表打散成多个参数、\verb|**|表示把字典打散作为参数:
\begin{lstlisting}
lst = (1,2,3,4,5)
dic = {'a':1,'b':2,'c':3}

func(*lst)                    # 1 (2, 3, 4, 5) {}
func(**dic)                   # 1 () {'c': 3, 'b': 2}

def func(a,b,c):
	print a

func( *(1,2,3) )         # OK
func(  (1,2,3) )         # err
func(  2 )               # err
\end{lstlisting}

			混合使用参数的格式:

\begin{lstlisting}
def func(a,b,c,... j=k,l=m,... *p,*q,... **x,**y...)
         ^         ^           ^         ^
         normal    key-value   seq       dict
\end{lstlisting}

\section{变量的作用域}

\subsection{全局变量}

用\verb|global|关键字声明要用的是全局变量:

\begin{lstlisting}
x = 5
def func():
	global x
	x = 99
\end{lstlisting}

\subsection{内置变量}

内置作用域变量要导入后才能使用:

\begin{lstlisting}
import __builtin__
\end{lstlisting}

\section{lambda表达式}

lambda表达式中不能出现语句,只能用表达式。例:

\begin{lstlisting}
f = lambda a,b : a+b
f(1,2)
\end{lstlisting}

		用表达式替代常用语句。例:

\begin{lstlisting}
lbd = lambda x : sys.stdout.write(x+'\n')
lbd('aaa')

lbd = lambda x : map(sys.stdout.write, x)
lbd(['aaa\n','bbb\n','ccc\n'])

lbd = lambda x : [sys.stdout.write(line) for line in x]
lbd(['aaa\n','bbb\n','ccc\n'])
\end{lstlisting}

	\section{闭包}

		一个函数的成员变量可以被定义在这个函数内部的内部函数访问。

		在早期不支持闭包的版本中,内部函数不能访问外部函数的变量;只能通过参数的默认值来取得外部函数变量的值:

\begin{lstlisting}
def func1():
	x = 88
	def func2(x=x):    # 作为默认参数赋值
		print x
	func2()
\end{lstlisting}

		在使用闭包的情况下:

\begin{lstlisting}
def func1():
	x = 88
	def func2():
		print x
	return func2

action = func1() # call func1()
action()         # call func2()
\end{lstlisting}

		要注意的一点是:函数只有在第一次被调用时存在。
		
		所以下面的例子中,变量\verb|i|的值永远是4。而不是随循环从0增长到4:

\begin{lstlisting}
def func1():
	acts = []
	for i in range(5):
		acts.append(lambda x: i ** x)
	return acts
\end{lstlisting}

		在这种情况下,只能使用默认参数:

\begin{lstlisting}
def func1():
	acts = []
	for i in range(5):
		acts.append(lambda x,i=i: i ** x)
	return acts
\end{lstlisting}

\section{结合函数处理序列}

\subsection{过滤(filter)}

过滤大于0的元素:

\begin{lstlisting}
filter( (lamabda x : x>0), [-2,-1,0,1,2,3,4])
# result: [1,2,3,4,]
\end{lstlisting}

\subsection{汇总(reduce)}

所有的元素加起来:
\begin{lstlisting}
reduce( (lamabda x,y : x+y), [1,2,3,4])   # 10
\end{lstlisting}

所有的元素乘起来:
\begin{lstlisting}
reduce( (lamabda x,y : x*y), [1,2,3,4])   # 24
\end{lstlisting}

在python3中可能移除reduce,但自己实现一个并不难:
\begin{lstlisting}
def myreduce(func, seq):
	tally = seq[0]
	for i in seq[1:]:
		tally = function(tally,i)
	return tally
\end{lstlisting}
