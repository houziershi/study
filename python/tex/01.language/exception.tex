\chapter{异常}

\section{捕获异常}

\begin{lstlisting}
try:
	statements
except type1:                 # match type
	statements
except type2, data:           # type and data
	statements
except (type3, type4):        # match those type
	statements
except (type3, type4), value: # match type get data
	statements
except:                       # all other type
	statements
else:
	statements
finally:
	statements
\end{lstlisting}

\subsection{基于字符串的异常}

基于字符串的异常马上就是被淘汰了。

字符串异常的捕获是按照同一对象来匹配的。不是同一个字符串对象即便值相同也不会被捕获到。

\subsection{基于类的异常}

通过\verb|__repr__|或是\verb|__str__|定义异常文本:

\begin{lstlisting}
class MyBad:
		def __repr__(self):
			return "Sorry-my mistake..."
\end{lstlisting}

\section{产生异常}

\begin{lstlisting}
raise string          # maches same string object
raise string, data    # extra data(default = None)
raise class, instance # 
raise instance        # same as: raise ins.__class__, ins
raise                 # re-raise current exception
\end{lstlisting}

\section{附加信息}

\subsection{异常类中附加信息}

在except分句中:

\begin{lstlisting}
except type, instance:
	statement
\end{lstlisting}

type和instance分别代表异常的类和实例:

\begin{lstlisting}
class FE:
	def __init__(self, line, file):
		self.line = line
		self.file = file

try:
	raise FormatError(42, file='spam.txt')
except FormatError, err
	print 'error at: ', err.file, err.line
\end{lstlisting}

\subsection{字符串异常中附加信息}

字符串中要额外附加一个对象来添加附加信息:

\begin{lstlisting}
errstr = 'FormatError'

try:
	raise errstr, {'line':42, 'file':'spam.txt'}
except errstr, err:
	print 'error at: ', err['file'], err['line']
\end{lstlisting}

\section{获得最新的异常}

可以通过\verb|sys.exc_info|取得最新引发的异常。返回值是包含\verb|type|、\verb|value|、\verb|traceback'|三个成员的元组;在没有异常的情况下会返回成员是三个\verb|None|的无组。


\section{断言}

\begin{lstlisting}
import asserter

def f(x):
	assert x < 0, 'x must be nagative'
	return x ** 2

asserter.f(1)
\end{lstlisting}

\section{环境管理器}

\subsection{使用环境管理}

使用\verb|with-as|环境管理会包装一个异常处理工作:

格式:

\begin{lstlisting}
with expression [as variable]:
	with-block
\end{lstlisting}

表达式产生的对象一定要是支持环境管理协议的对象,如文件对象已经支持环境管理协议:

\begin{lstlisting}
with open('aa.txt') as myfile:
    for line in myfile:
            print line
\end{lstlisting}

\subsection{实现环境管理协议}

必须要有\verb|__enter__|和\verb|__exit__|方法。

\verb|__enter__|方法被调用后的返回值会赋值给\verb|as|指定的变量。如果没有\verb|as|则丢弃这个变量。

如果with代码块引发了异常,\verb|__exit__(type, value, traceback)|会被调用。如果此方法返回值为False,会再次抛出异常;否则表示异常已经被处理完毕。

如果iwth代码块没有发生异常,\verb|__exit__|方法还是会被调用,其中的type、value、trackback参数都会为None。

例子:

\lstinputlisting[]{py/chap.exception/with/usewith.py}
