
\chapter{模块}

\section{模块与路径}

在linux环境下查找python的安装目录:

\begin{lstlisting}
whereis python
\end{lstlisting}

在python交互环境中查看python路径:

\begin{lstlisting}
sys.path
\end{lstlisting}

\section{导入模块}

\subsection{导入模块}

导入模块使用\verb|import|语句:

\begin{lstlisting}
import module01
\end{lstlisting}

当模块被导入后,会包含源文件的目录信息:<目录>/<文件名>.<扩展名>。

\subsection{重新导入模块}

一个模块只能有一个实例,导入一个模块以后不能再次导入。当一个模块的代码被修改以后必须使用\verb|reload|语句重新加载模块才能生效:

\begin{lstlisting}
reload module01
\end{lstlisting}

\verb|reload|只会重新导入写出的那个模块,不会自动导入相关的模块。

\subsection{部分导入模块}
如果只导入模块中的部分变量,则使用\verb|from|语句:

\begin{lstlisting}
from module01 import a, b, c
\end{lstlisting}

\verb|from|语句不导入模块,只复制模块中的变量到本地。复制模块中全部变量的例子:

\begin{lstlisting}
from module01 import *
\end{lstlisting}

\verb|import|与\verb|from|语句都都有赋值效果,例如:

\begin{lstlisting}
from module1 import a, b, c
\end{lstlisting}

等价于:

\begin{lstlisting}
import module1

a = module1.a
b = module1.b
c = module1.c
\end{lstlisting}

\subsection{防止变量被from复制}

方法一:下划线开头的变量不会被复制,如:“\verb|_X|”。

方法二:在顶层变量“\verb|__all__|”列表中的成员不会被复制出。如:

\begin{lstlisting}
__all__ = ['Error', 'encode']
\end{lstlisting}

\subsection{模块别名}

导入了模块以后,使用时要加上包名,所以有必要使用简写:

\begin{lstlisting}
import longname as name
\end{lstlisting}

等价于:

\begin{lstlisting}
import longname
name = longname
del longname
\end{lstlisting}

\verb|from|语句也可以用简写:

\begin{lstlisting}
from mod import longnmae as name
\end{lstlisting}

\subsection{判断是否是直接执行}

模块有\verb|__name__|属性。如果是被导入的,值为模块的名字;如果是被直接调用的,值为:\verb|'__main__'|。可以通过它判断程序是否是被导入的:

\begin{lstlisting}
if __name__ =  '__main__' :
	# in main
\end{lstlisting}

\section{直接执行脚本文件}

可以不加载模块而是直接以脚本文件的方式执行:

\begin{lstlisting}
execfile('mymodule.py')
\end{lstlisting}

\section{包的使用}

\subsection{建立自己的包}

包放在python检索的目录下。包目录与包下的每一级都要有\verb|__init__.py|文件(内容可以为空)表明这是一个包,\verb|__init__.py|文件中的语句都会在导入时被执行。例:

\begin{lstlisting}
# dir1/__init__.py
print 'dir1'
x = 1
\end{lstlisting}

\begin{lstlisting}
# dir1/dir2/__init__.py
print 'dir2'
y = 2
\end{lstlisting}

\begin{lstlisting}
# dir1/dir2/mod.py
print 'mod.py'
z = 3
\end{lstlisting}

\subsection{使用自己的包}

导入自己包的例子:

\begin{lstlisting}
% python
>>> import dir1.dir2.mod
dir1
dir2
mod.py

>>> import dir1.dir2.mod

>>> reload(dir1)
dir1

>>> reload(dir1.dir2)
dir2

>>> dir1.x
1
>>> dir1.dir2.y
2
>>> dir1.dir2.mod.z
3
\end{lstlisting}
