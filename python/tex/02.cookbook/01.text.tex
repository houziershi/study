
\chapter{文本}

	\section{字符与其值的转换}

		\lstinputlisting[]{py/02.cookbook/0102/exp.py}

	\section{判断一个对象是否是字符串}

		\verb|basestring|是str类型与Unicode类型的共同基类,但是Python标准库中的\verb|UserString|模块提供的\verb|UserString|类不是从它派生出来的。这种情况下可能尝试对象是否能像字符串一样工作(鸭子判断法)来判断,缺点是异常块的执行性能较差。

		\lstinputlisting[]{py/02.cookbook/0103/exp.py}

	\section{对齐字符串}

		本来对齐字符串很简单。
		
		\lstinputlisting[]{py/02.cookbook/0104/exp.py}

		但是全角字符的显示宽度不一样,要特别处理一下:

		\lstinputlisting[]{py/02.cookbook/0104/exp02.py}


	\section{拼接字符串}
		
		字符串可以直接用加号拼接,不过性能低下:

		\lstinputlisting[firstline=3,lastline=8]{py/02.cookbook/0106/exp0106.py}

		下面的代码相同与:

		\lstinputlisting[firstline=10,lastline=11]{py/02.cookbook/0106/exp0106.py}

		用\verb|join|方法通过指定的字符串把各个部分拼接起来:

		\lstinputlisting[firstline=13,lastline=13]{py/02.cookbook/0106/exp0106.py}

		更加复杂的格式可以用格式把字符串方法:

		\lstinputlisting[firstline=15,lastline=15]{py/02.cookbook/0106/exp0106.py}

	\section{反转字符串}
		
		反转每一个字符:

		\lstinputlisting[firstline=3,lastline=4]{py/02.cookbook/0106/exp0107.py}

		反转单词可以先按空格打散,然后再接起来:

		\lstinputlisting[firstline=6,lastline=9]{py/02.cookbook/0106/exp0107.py}

		挤成一行的写法:

		\lstinputlisting[firstline=12,lastline=12]{py/02.cookbook/0106/exp0107.py}

		用正则来保留原来是用空格还是用其他空白字符分隔的:

		\lstinputlisting[firstline=15,lastline=19]{py/02.cookbook/0106/exp0107.py}

		挤成一行的写法:

		\lstinputlisting[firstline=21,lastline=21]{py/02.cookbook/0106/exp0107.py}

	\section{包含与不包含}

		判断字符(不光是字符串,还有其他的集合)是否属于一个集合:

		\lstinputlisting[firstline=3,lastline=7]{py/02.cookbook/0106/exp0108.py}

		用\verb|itertools|可以提高一点性能:

		\lstinputlisting[firstline=12,lastline=16]{py/02.cookbook/0106/exp0108.py}
		
	\section{控制大小写}
		
		\lstinputlisting[firstline=3,lastline=6]{py/02.cookbook/0106/exp0109.py}
		
	\section{字符串模板替换}
		
		\lstinputlisting[firstline=5,lastline=9]{py/02.cookbook/0106/exp0117.py}
		
	\section{一次完成多个替换}
		
		可以先根据dict的key建立了一个正则表达式;然后在\verb|re.sub|调用过程中使用了回调函数代替了用于替换的字符串。这样每当匹配时,\verb|re.sub|会调用该回调函数,以它的返回值作为替换的字符串:

		\lstinputlisting[firstline=3,lastline=14]{py/02.cookbook/0106/exp0118.py}
		
	\section{检查字符串的结尾}
		
		字符串的\verb|endwith|函数判断是否以指定字符串结尾。主要的思路是把字符串的\verb|endwith|作为参数传递给一个中间函数\verb|anyTrue|。然后用\verb|itertools.imap|来映射检查每一项:

		\lstinputlisting[firstline=3,lastline=9]{py/02.cookbook/0106/exp0119.py}

		上例中

		一个应用的场景是查找图片文件:

		\lstinputlisting[firstline=11,lastline=15]{py/02.cookbook/0106/exp0119.py}

		
	\section{应用Unicode}

		python中Unicode文本和普通的字符串不是同一种对象,不能混合操作。所以要时刻明白你正在处理的文本对象是普通字符串还是Unicode字串。一般来说,会用以下措施来防止错误发生:

		在收到外部的文本数据时,应该创建一个uncode对象,通过查看HTTP头之类的方法来确定所用的编码方式。不然错误的操作会引发UnicodeDeocdeError。

		同理在向外部发送文本之前要用正确的编码把文本转化为字节串,不然有可能发生UnicodeEncodeError。
		
	\section{Unicode与字符串之间的转换}

		转换时要先确定编码:

		\lstinputlisting[firstline=3,lastline=18]{py/02.cookbook/0106/exp0121.py}

	\section{在标准输出中打印Unicode}

		通过cocdes模块把sys.stdout流用转换器包装起来。如果你要输出的终端以iso-8859-1编码显示字符:

		\lstinputlisting[firstline=3,lastline=5]{py/02.cookbook/0106/exp0122.py}

		上面的代码把\verb|sys.stdout|绑定到一个使用Unicode输入和ISO-8859-1输入的流。这样的好处是不会改变\verb|sys.stdout|上原来的编码。

		Python会尝试确认“终端”所使用的编码(并不一定正确)并把编码的名字存放到\verb|sys.stdout.encoding|中作为一个属性。

	\section{在XML与HTML中使用Unicode}


		

		

		
