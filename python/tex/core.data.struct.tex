

\chapter{核心数据类型}

	在python的核数据类型中,数字、字符串、元组这三个对象是不可变的。

	\section{基本数字类型}
		
		数字对象是不可变的,许多对它的操作会生成一个新的对象返回,而不是改变原来的对象。

		\begin{verbatim}
			123, 5, 0                       # 同C语言的长整数
			999999999999L                   # 无限制长度
			1.23, 3.14e-10, 4E210, 4.0e+210 # 浮点数类似C语言双精度数
			0177, 0x13e, 0X3e               # 八进制与十六进制
			3+4j, 3.0+4.0j, 3J              # 复数,实数加上复数
		\end{verbatim}

		格式化数字:

		\lstinputlisting[firstline=1,lastline=7]{py/coreda/exnum.py}

		常用操作:加(\verb|+|)、减(\verb|-|)、乘(\verb|*|)、除(\verb|/|)、下取整(\verb|//|)、乘方(\verb|**|)、位移(\verb|>>|)等:

		\lstinputlisting[firstline=9,lastline=16]{py/coreda/exnum.py}

		调用内置的函数进行强制类型转换:

		\lstinputlisting[firstline=18,lastline=27]{py/coreda/exnum.py}

		\subsection{常用数学模块}

		\lstinputlisting{py/coreda/exmatch.py}

	\section{十进制小数类型}

		集合类型set可以支持一般的数学集合操作:

		\lstinputlisting{py/coreda/exdecimal.py}

	\section{序列类型的共同点}
		
		字符串、列表、元组都属于序列类型类型,会支持一些共同的操作。

		索引操作:

		\lstinputlisting[firstline=1,lastline=10]{py/coreda/exstr.py}

		分片操作:
		\lstinputlisting[firstline=11,lastline=17]{py/coreda/exstr.py}

		连接与重复操作会生成新的对象:
		\lstinputlisting[firstline=19,lastline=20]{py/coreda/exstr.py}

	\section{字符串类型}

		字符串也是一种序列类型,所以支持所有的序列操作。

		字符串对象是不可变的,许多对它的操作会生成一个新的对象返回,而不是改变原来的对象。

		\lstinputlisting[firstline=22,lastline=32]{py/coreda/exstr.py}

	\section{列表}

		列表也是一种序列类型,所以支持所有的序列操作。

		虽然列表没有固定大小,但是还是会越界。越界会抛出IndexError。
		
		列表常用的特定操作:

		\lstinputlisting[firstline=1,lastline=11]{py/coreda/exlist.py}

		列表可以进行嵌套,并且经常被用在多维数组上:

		\lstinputlisting[firstline=13,lastline=26]{py/coreda/exlist.py}

	\section{字典}

		不能读取一个字典中不存在的键:

		\lstinputlisting[firstline=1,lastline=25]{py/coreda/exdict.py}

		对字典内容进行排序:

		\lstinputlisting[firstline=27,lastline=35]{py/coreda/exdict.py}

	\section{元组}

		元组也是一种序列类型,所以支持所有的序列操作。

		元组对象是不可变的,许多对它的操作会生成一个新的对象返回,而不是改变原来的对象。

%		\lstinputlisting[firstline=1,lastline=12]{py/coreda/exfile.py}
		
	\section{集合类型}

		集合类型set可以支持一般的数学集合操作:

		\lstinputlisting{py/coreda/exset.py}
		
	\section{bool类型与空}

		bool类型实际是在以定制后的逻辑来显示整数的1和0。

		None对象表示空
		
		\lstinputlisting{py/coreda/exbool.py}

	\section{type类型}

		再次强调python中的类型信息与变量无关,类型是关联在对象上的。

		\lstinputlisting{py/coreda/extype.py}

	\section{对象的副本}
		\subsection{对象的副本}

			要为序列类对象为了建立一个用于修改的副本,用全部分片是最方便的方法:

			\lstinputlisting[firstline=1,lastline=7]{py/coreda/excopyobj.py}

			对于字典对象,有成员方法\verb|copy|:

			\lstinputlisting[firstline=9,lastline=13]{py/coreda/excopyobj.py}

			标准库中的copy可以拷贝任意对象,分只复制顶层与深层复制:

			\lstinputlisting[firstline=15,lastline=31]{py/coreda/excopyobj.py}

		\subsection{对象的相等与同一}

			判断两个对象是否相等用“\verb|==|”操作符,判断两个对象是不是同一个对象用“\verb|is|”操作符:

			\lstinputlisting[firstline=33,lastline=41]{py/coreda/excopyobj.py}

			在\verb|sys|模块中的\verb|getrefcount|函数会返回一个对象被引用的次数:

			\lstinputlisting[firstline=43,lastline=45]{py/coreda/excopyobj.py}
