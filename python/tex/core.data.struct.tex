

\chapter{核心数据类型}

	在python的核数据类型中,数字、字符串、元组这三个对象是不可变的。

	\section{基本数字类型}
		
		数字对象是不可变的,许多对它的操作会生成一个新的对象返回,而不是改变原来的对象。

		支持的操作:加(\verb|+|)、减(\verb|-|)、乘(\verb|*|)、除(\verb|/|)、乘方(\verb|**|)。

		\subsection{常用数学模块}


\begin{lstlisting}
import math
import random

print math.pi
print math.sqrt(85)

print random.random()
print random.choice([1,2,3,4,5])
\end{lstlisting}

	\section{序列类型的共同点}
		
		字符串、列表、元组都属于序列类型类型,会支持一些共同的操作。

		索引操作:

		\lstinputlisting[firstline=1,lastline=10]{py/coreda/exstr.py}

		分片操作:
		\lstinputlisting[firstline=11,lastline=17]{py/coreda/exstr.py}

		连接与重复操作会生成新的对象:
		\lstinputlisting[firstline=19,lastline=20]{py/coreda/exstr.py}

	\section{字符串类型}

		字符串也是一种序列类型,所以支持所有的序列操作。

		字符串对象是不可变的,许多对它的操作会生成一个新的对象返回,而不是改变原来的对象。

		\lstinputlisting[firstline=22,lastline=32]{py/coreda/exstr.py}

	\section{列表}

		列表也是一种序列类型,所以支持所有的序列操作。

		虽然列表没有固定大小,但是还是会越界。越界会抛出IndexError。
		
		列表常用的特定操作:

		\lstinputlisting[firstline=1,lastline=11]{py/coreda/exlist.py}

		列表可以进行嵌套,并且经常被用在多维数组上:

		\lstinputlisting[firstline=13,lastline=26]{py/coreda/exlist.py}

	\section{字典}

		不能读取一个字典中不存在的键:

		\lstinputlisting[firstline=1,lastline=25]{py/coreda/exdict.py}

		对字典内容进行排序:

		\lstinputlisting[firstline=27,lastline=35]{py/coreda/exdict.py}

	\section{元组}

		元组也是一种序列类型,所以支持所有的序列操作。

		元组对象是不可变的,许多对它的操作会生成一个新的对象返回,而不是改变原来的对象。

		\lstinputlisting[firstline=1,lastline=12]{py/coreda/exfile.py}
		
	\section{集合类型}

		集合类型set可以支持一般的数学集合操作:

		\lstinputlisting{py/coreda/exset.py}
		
	\section{bool类型与空}

		bool类型实际是在以定制后的逻辑来显示整数的1和0。

		None对象表示空
		
		\lstinputlisting{py/coreda/exbool.py}


	\section{十进制小数类型}

		集合类型set可以支持一般的数学集合操作:

		\lstinputlisting{py/coreda/exdecimal.py}

	\section{type类型}

		再次强调python中的类型信息与变量无关,类型是关联在对象上的。

		\lstinputlisting{py/coreda/extype.py}
