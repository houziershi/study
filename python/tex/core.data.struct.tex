
\chapter{核心数据类型}

在python的核数据类型中,数字、字符串、元组这三个对象是不可变的。

\section{基本数字类型}

数字对象是不可变的,许多对它的操作会生成一个新的对象返回,而不是改变原来的对象。

\begin{lstlisting}
123, 5, 0                       # 同C语言的长整数
999999999999L                   # 无限制长度
1.23, 3.14e-10, 4E210, 4.0e+210 # 浮点数类似C语言双精度数
0177, 0x13e, 0X3e               # 八进制与十六进制
3+4j, 3.0+4.0j, 3J              # 复数,实数加上复数
\end{lstlisting}

格式化数字:

\begin{lstlisting}
num = 1 / 3.0

print num
print '%e'    % num            # string formating
print '%2.2f' % num            # string formating
print '%o %x %X' % (64,64,255) # '100 40 FF'
\end{lstlisting}

常用操作:加(\verb|+|)、减(\verb|-|)、乘(\verb|*|)、除(\verb|/|)、下取整(\verb|//|)、乘方(\verb|**|)、位移(\verb|>>|)等:

\begin{lstlisting}
print 1 /  3         # 0
print 1 // 3         # 0
print 1.0 / 3        # 0.33333333333
print 1.0 // 3       # 0.0
\end{lstlisting}

调用内置的函数进行强制类型转换:

\begin{lstlisting}
print int(3.1415)    # 3
print float(3)       # 3.0
print long(4)        # 4

print oct(64)        # 0100
print hex(64)        # 0x40

print int('0100')    # 100
print int('0100', 8) # 64
print int('0x40',16) # 64
\end{lstlisting}

\subsection{常用数学模块}

\begin{lstlisting}
# coding=utf-8
import random
import math

print random.random()                 # 0.323492834792
print random.randint(1,10)            # 3
print random.choice([1,2,3,4,5])      # 2
print random.choice(['aa','bb','cc'])  # cc

print math.e
print math.pi
print math.sqrt(85)
print math.sin(2*math.pi / 180)

print abs(-42)       # 42
print pow(2,4)       # 16
print round(2.567)   # 3.0
print round(2.567,2) # 2.57
\end{lstlisting}

\section{十进制小数类型}

集合类型set可以支持一般的数学集合操作:

\begin{lstlisting}
# coding=utf-8

import decimal

d = decimal.Decimal('3.141')

print d+1

print decimal.Decimal('0.1') + decimal.Decimal('0.1') \
	+ decimal.Decimal('0.1') 
print decimal.Decimal('1') / decimal.Decimal('7') 

# set precision
decimal.getcontext().prec = 4
print decimal.Decimal('1') / decimal.Decimal('7') 
\end{lstlisting}

\section{序列类型的共同点}

字符串、列表、元组都属于序列类型类型,会支持一些共同的操作。

索引操作:
\begin{lstlisting}
s = 'Spam'
print len(s)          # length
print s[0]            # index start from 0
print s[1]            # 
print s[-1]           # first from end
print s[len(s)-1]     # first from end
print s[-2]           # second from end
\end{lstlisting}

分片操作:
\begin{lstlisting}
print s[1:3]
print s[1:]
print s[:3]
print s[:-1]
print s[:]
print s[::2]          # set step
print s[::-1]         # backward
\end{lstlisting}

连接与重复操作会生成新的对象:
\begin{lstlisting}
print s + '^_^'       # create new object, s still the same
print s * 3           # create new object, s still the same
\end{lstlisting}

\section{字符串类型}

字符串也是一种序列类型,所以支持所有的序列操作。

字符串对象是不可变的,许多对它的操作会生成一个新的对象返回,而不是改变原来的对象。

\subsection{常用方法}

python中的字符串使用双引号和单引号都是一样的。而且空值不会像在C语言里中断字符串。

三重引号可以按原文格式生成文本。可以用来注释大段的代码。

\begin{lstlisting}
# coding=utf-8

# use oct and hex
print str(42)
print int("42")
print str(3.14159265)
print float("1.234E-10")

print ord('\n')       # show 10, because'\n' binary value is 0x10
print ord('s')          # 115
print chr(115)          # s

s = '\001\002\x03'
print s
print len(s)            # 3

# user unicode
print u'spam'
print u'ab\x20cd'       # 1-byte char : ab cd
print u'ab\u0020cd'     # 2-byte char : ab cd
print u'ab\U00000020cd' # 4-byte char : ab cd

# do not escape
print r'c:\new\text.txt'

# corss lines 
print """ threre is 
cross lines...
hahahah....  """

print ''' threre is 
cross lines...
hahahah....  '''


line = '   aaa,bbb,ccc,ddd,eee   '
print line.lstrip()
print line.rstrip()
print line.upper()
print line.isalpha()       # False
print 'abc'.endswith('c')  # True

# join up the string
print 'aaa' + "bbb"    # aaabbb
print 'aaa' "bbb"      # aaabbb

# create a list
ll =  'aaa',"bbb"      # ('aaa','bbb')

# string format
exclamation = "Ni"

print "%d %s %d you" % (1, 'spam', 4)
# use turple to set more element into one place
print "%s -- %s -- %s" % (42, 3.14159, [1,2,3])
# format by dict
print "%(n)d %(x)s" % {"n":1, "x":"spam"}

s = 'hello a hello b hello c'
print s.find('hello')           # offsit is 0
print s.replace('hello', 'bye') # bye a bye b bye c

# break string to list
l = list('abcdefg') 
print l                # ['a', 'b', 'c', 'd', 'e', 'f', 'g']
# join list to string
print ''.join(l)       # abcdefg
print '-'.join(l)       # a-b-c-d-e-f-g

# split string
print 'aa bb cc'.split()    # ['aa','bb','cc']
print 'aa,bb,cc'.split(',') # ['aa','bb','cc']
\end{lstlisting}

\section{列表}

列表也是一种序列类型,所以支持所有的序列操作。而且列表是可变类型。

虽然列表没有固定大小,但是还是会越界。越界会抛出IndexError。

列表常用的特定操作:

\begin{lstlisting}
l = [123, 'spam', 1.23, 1, 2, 3, 4, 5, 6, 7, 8, 9, 10]

l.index(1)

l.append('NI') # [123, 'spam', 1.23, 'NI']
l.extend([5,6,7])
l.insert(1,88)

l.pop(2)       # 1.23 # [123, 'spam', 'NI']
l.remove(2)
del l[1]
del l[1:2]

l[0] = 5
l[3:5] = [66,77,88]

range(4)       # create new list [1,2,3,4]
xrange(0,4)    # xrange(4)

l.sort()
l.reverse()
\end{lstlisting}

列表可以进行嵌套,并且经常被用在多维数组上:

\begin{lstlisting}
m = [[1,2,3], 
		[4,5,6], 
		[7,8,9]]

print m[1]    # [4,5,6] print row 2 
print m[1][2] # 6       print row 2, item 3

# show col 3 use list comprehension expression
print [row[1] for row in m]                     # [2,5,8]
print [row[1]+1 for row in m]                   # [3,6,9]
print [row[1] for row in m if row[1] % 2 == 0 ] # [2,8]

print [m[i][i] for i in [0,1,2]]          # [1,5,9]
print [c*2 for c in 'span']               # ['ss','pp','aa','mm']
\end{lstlisting}

\section{元组}

元组也是一种序列类型,所以支持所有的序列操作。

元组对象是不可变的,许多对它的操作会生成一个新的对象返回,而不是改变原来的对象。

新建一个元组的操作:

\begin{lstlisting}
tuple(l)
\end{lstlisting}

\section{字典}

创建一个字典的方法:

\begin{lstlisting}
d = {'a':1, 'b':2, 'c':3, 'd':4, 'e':5, 'f':6, 'g':7}
d2 = dict.fromkeys(['a','b']) # {'a':None, 'b':None}
d3 = dict(zip(d.keys(),d.values()))
d4 = dict(name='skinner', age=18)

d.copy()

# create new dict and add attr
d = {}
d['name'] = 'Bob'
d['job'] = 'Dev'
d['age'] = 25
\end{lstlisting}

字典的常用方法:

\begin{lstlisting}
d = {'a':1, 'b':2, 'c':3, 'd':4, 'e':5, 'f':6, 'g':7}
d['a']
d['a'] = 33
d['a'] += 1

d2.update(d3)  # update d2's value as d3's same key value

d.pop('a')
del d['c']

len(d)

d.keys()
d.values()
d.items()
\end{lstlisting}

字典中的键不一定要用字符串,可以用任何不可变对象。不能读取一个字典中不存在的键,但可以给不存在的键设一个默认值:

\begin{lstlisting}
'f' in d
d.has_key('f')
d.get('f', 'default')
\end{lstlisting}

字典中的项目可以是不同的类型:

\begin{lstlisting}
# nesting
rec = {'name': {'first':'Jade', 'last':'Shan'},
		'job': ['dev','mgr'],
		'age': 25}

print rec['name']['last']     # Shan
print rec['job'][-1]          # mgr
\end{lstlisting}

对字典内容进行排序:

\begin{lstlisting}
ks = d.keys()                 # ['a','c','b']
ks.sort()                     # sort keys
for key in ks:
	print key, ' => ', d[key]

for key in sorted(d):
	print key, ' => ', d[key]
\end{lstlisting}

用整数来作为字典的键,模拟一个不会越界的列表:

\begin{lstlisting}
d = {}
d[99] = 'spam'
\end{lstlisting}

用字典实现稀疏数据结构:

\begin{lstlisting}
mtx = {}
mtx[(2,3,4)] = 88
mtx[(7,8,9)] = 99

print mtx.get((2,3,4),0)
print mtx.get((7,8,9),0)
print mtx.get((0,8,9),0)
\end{lstlisting}

字典可以直接迭代:

\begin{lstlisting}
for key in dic :
	print key, dic[key]
\end{lstlisting}

\section{集合类型}

集合类型set可以支持一般的数学集合操作:

\begin{lstlisting}
x = set('spam')
y = set(['h','a','m'])

print x,y   # set(['a', 'p', 's', 'm']) set(['a', 'h', 'm'])
print x & y # set(['a', 'm'])
print x | y # set(['a', 'p', 's', 'h', 'm'])
print x - y # set(['p', 's'])

engineers = set(['bob','sue','ann','vic'])
managers  = set(['tom','sue'])

print engineers & managers # set(['sue'])
print engineers - managers # set(['vic', 'bob', 'ann'])
print engineers | managers # set(['vic', 'sue', 'tom', 
                           # 'bob', 'ann'])
\end{lstlisting}

\section{bool类型与空}

bool类型实际是在以定制后的逻辑来显示整数的1和0。

None对象表示空

\begin{lstlisting}
print 1>2           # False
print bool('spam')  # True
print None          # None
\end{lstlisting}

常见对象的布尔值状态:

\begin{lstlisting}
"spam"   True
""       False
[]       False
{}       False
1        True
0.0      False
None     False
\end{lstlisting}

常见操作:

\begin{lstlisting}
X and Y
X or  Y
not x
\end{lstlisting}

\section{type类型}

再次强调python中的类型信息与变量无关,类型是关联在对象上的。

\begin{lstlisting}
l = [None] * 100

print type(l)                   # <type 'list'>
print type(type(l))             # <type 'type'>

print type(l) == type([])       # True
print type(l) == list           # True
print isinstance(l, list)       # True
\end{lstlisting}

\section{对象的副本}

当一个复杂的对象内部有一个引用是它自己的话,就形成了循环引用的情况。python会把循环引用的对象打印为\verb|[...]|。但是程序在处理的时候还是会造成无限循环。	

\subsection{对象的副本}

要为序列类对象为了建立一个用于修改的副本,用全部分片是最方便的方法:

\begin{lstlisting}
l1 = [1,2,3,4,5]
l2 = l1[:]
l2[0] = 0
print l1          # [1, 2, 3, 4, 5]
print l2          # [0, 2, 3, 4, 5]
\end{lstlisting}

对于字典对象,有成员方法\verb|copy|:

\begin{lstlisting}
dic1 = {'a':1,'b':2,'c':3}
dic2 = dic1.copy()
dic2['a'] = 0
print dic1        # {'a': 1, 'c': 3, 'b': 2}
print dic2        # {'a': 0, 'c': 3, 'b': 2}
\end{lstlisting}

标准库中的copy可以拷贝任意对象,分只复制顶层与深层复制:

\begin{lstlisting}
import copy

l1 = [1,2,3,4,5]
dic1 = {'a':1,'b':2,'c':l1}
dic2 = copy.copy(dic1)
dic2['a'] = 99
dic2['c'][0] = 99
print dic1         # {'a': 1, 'c': [99, 2, 3, 4, 5], 'b': 2}
print dic2         # {'a': 99, 'c': [99, 2, 3, 4, 5], 'b': 2}

l1 = [1,2,3,4,5]
dic1 = {'a':1,'b':2,'c':l1}
dic2 = copy.deepcopy(dic1)
dic2['a'] = 99
dic2['c'][0] = 99
print dic1         # {'a': 1, 'c': [1, 2, 3, 4, 5], 'b': 2}
print dic2         # {'a': 99, 'c': [99, 2, 3, 4, 5], 'b': 2}
\end{lstlisting}

序列的重复会生成新的序列:

\begin{lstlisting}
l = [4,5,6]
x = l * 4    # new list
y = [l] * 4  # reference l 4 time

print x # [4, 5, 6, 4, 5, 6, 4, 5, 6, 4, 5, 6]
print y # [[4, 5, 6], [4, 5, 6], [4, 5, 6], [4, 5, 6]]

l[1] = 0
print x # [4, 5, 6, 4, 5, 6, 4, 5, 6, 4, 5, 6]
print y # [[4, 0, 6], [4, 0, 6], [4, 0, 6], [4, 0, 6]]
\end{lstlisting}

\subsection{对象的相等与同一}

判断两个对象是否相等用“\verb|==|”操作符,判断两个对象是不是同一个对象用“\verb|is|”操作符:

\begin{lstlisting}
l1 = [1,2,3]
l2 = [1,2,3]
print l1 == l2     # True
print l1 is l2     # False

x = 42
y = 42
print x == y       # True
print x is y       # True
\end{lstlisting}

在\verb|sys|模块中的\verb|getrefcount|函数会返回一个对象被引用的次数:

\begin{lstlisting}
import sys
print sys.getrefcount(1)  # 86
print sys.getrefcount(x)  # 11
\end{lstlisting}
