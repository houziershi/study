\chapter{与java整合}

\section{通过java工具编译}

\begin{lstlisting}[language=Bash]
>>> javac ch05/HelloWorld.java 
>>> java  ch05.HelloWorld 
hello world
\end{lstlisting}






\section{通过ant构建Java程序}

java源代码都放在“java”目录下面:

\lstinputlisting[language=Java]{src/05/java/ch05/HelloWorld.java}

构建文件:

\lstinputlisting[language=XML]{src/05/build.01.xml}







\section{编译Java程序(javac)}

\subsection{主要属性}

srcdir、destdir、includes、includesfile、excludes、excludesfile、classpath、sourcepath、classpathref、sourcepathref、nowarn(不显示警告信息)、debug、debuglevel、optimize(优化等级)、verbose、failonerror。

source:当jvm支持-source参数时,指定。

compiler:指定编译器。

fork:编译时使用外部java编译器。默认为false。

excutable:当fork为true时,指定外部编译器的位置。

memoryInitialSize与memoryMaximunSize:当fork为true时使用的内存大小。

depend:编译时依赖跟踪(只有ibm-jikes和classic两种编译器才能用)。

includeantruntime:包含ant库。

includejavaruntime:包含java运行环境库。

deprecation:源文件是否带描述信息。

target:生成程序对应的jvm的版本。

extdirs:安装扩展文件的位置。

bootclasspath:启动class位置

bootcalsspathref:引用已经定义的\verb|<path>|元素。

\subsection{示例}

基本配置,都用默认值。

\lstinputlisting[language=XML,firstline=6,lastline=8]{src/05/build.02.xml}

指定源文件:

\lstinputlisting[language=XML,firstline=11,lastline=18]{src/05/build.02.xml}

使用外部编译器:

\lstinputlisting[language=XML,firstline=20,lastline=22]{src/05/build.02.xml}





\section{运行Java程序(java)}

\subsection{主要属性}

jar、args、classpath、classpathref、fork、jvm、jvmargs、maxmemory、failonerror。

classname:要执行的主类。

dir:java虚拟机的目录。

output、error:输出信息。

logError:输出错误到ant日志。

appended:覆盖输出文件。

resultproperty:指定一个已经定义的属性来记录命令执行时的返回值。

outputproperty、errorproperty:输出与标准错误的内容到指定属性。

input:定义标准输入。

inputstring:定义输入的字符串。

newenvironment:有了新的环境变量就不再传递旧的环境变量。

outlive:当fork为true时,允许打开另一个进程在ant工具外执行。

timeout:规定时间内没有完成就中止。

\subsection{示例}

基本配置,都用默认值。

\lstinputlisting[language=XML]{src/05/build.03.xml}






\section{打包jar(jar)}

\subsection{主要属性}

destfile、basedir、compress、keepcompression、encoding、fileonly、includes、includesfile、excludes、excludesfile、defaultexcludes、manifest、

filesetmanifest:当有多个manifest时,是忽略(skip)、合并(merge)、合并但是不包含manifest中的main函数定义(mergewithoutmain)。

update:是否覆盖已经存在jar中的目标文件。

whenempty:没有找到符合的文件打成jar包时,是报错(fail)、忽略(skip)、生成空文件(create)。

duplicate:重复时,覆盖(add)、保持原来的文件(preserve)、报错(fail)。

index:建立索引文件记录jar文件的内容。

manifestencoding:manifest文件的编码。

\subsection{定义MANIFEST.MF}

要用到manifest任务,主要属性有:

file:指定建立或更新manifest-file文件.

mode:是update还是replace。

encoding:编码。

attribute:属性和“键-值”对应。

section:区块,一个section可以包含多个attribute。

\lstinputlisting[language=XML]{src/05/build.04.xml}

\lstinputlisting[language=Bash]{src/05/MANIFEST.MF}

\subsection{打jar包的示例}

打包时选择要包含的文件:

\lstinputlisting[language=XML,firstline=4,lastline=7]{src/05/build.05.xml}

通过FileSet选择文件:

\lstinputlisting[language=XML,firstline=9,lastline=13]{src/05/build.05.xml}

\subsection{打jar包并运行}

\lstinputlisting[language=XML]{src/05/build.06.xml}







\section{打包WAR(war)}

启动tomcat:

\begin{lstlisting}[language=Bash]
>>> ./catalina.sh run
\end{lstlisting}


\subsection{主要属性}

destfile、webxml、basedir、compress、keepcompression、encoding、fileonly、includes、includesfile、excludes、excludesfile、defaultexcludes、manifest、update。

duplicate:覆盖重复的文件(add)、不覆盖(preserve)、报错(fail)。

\subsection{放置文件到war包对应位置}

\lstinputlisting[language=XML]{src/05/build.08.xml}

\subsection{例子}

\lstinputlisting[language=XML]{src/05/build.07.xml}








\section{打包EAR(ear)}


\subsection{主要属性}

destfile、appxml、basedir、compress、keepcompression、encoding、fileonly、includes、includesfile、excludes、excludesfile、defaultexcludes、manifest、update。

duplicate:覆盖重复的文件(add)、不覆盖(preserve)、报错(fail)。

\subsection{例子}

\lstinputlisting[language=XML]{src/05/build.09.xml}








\section{解压压缩文件}

\subsection{主要属性}

src、dest、overwrite、encoding。

compression:none(默认)、gzip、bzip2。

还可以使用FileSet和PatternSet。

\subsection{例子}

\lstinputlisting[language=XML]{src/05/build.09.xml}








\section{扩展Ant Task}

\subsection{方法}

可以通过继承不同的基类来实现扩展功能:

org.apache.tools.ant.Task:抽象类,所以ant任务的基类。

org.apache.tools.ant.taskdefs.AbstractCvsTask:访问CVS相关。

org.apache.tools.ant.taskdefs.MatchingTask:与include和excludes文件匹配模式相关。

org.apache.tools.ant.taskdefs.Pack:与Zip打包相关。

org.apache.tools.ant.taskdefs.Unpack:与Zip解包相关。

一般来说继承一个对应的类就可以实现相关的功能了。大概有以下六个步聚:

1)实现一个子类。

2)为每一个要用到的属性编写“set”方法(必须是public void)。在构建文件中通过对properties的引用来给属性赋值。

3)如果一个Task要包含其他的Task,那就一定要实现“org.apache.tools.ant.TaskContainer”接口。

4)如果需要支持字符输入(XML中的text形式),要添加“public void addText(String text)”方法。

5)如果需要嵌套Ant的元素或类型,要添加create、add、addConfigured方法。例如要嵌套自定义的<inner>元素,可能要用到的方法有:

建立一个NestedElement实例:

\begin{verbatim}
public NestedElement createInner();
\end{verbatim}

在任务构造时,把NestedElement实例加入到任务里:

\begin{verbatim}
public void addInner(NestedElement anInner);
\end{verbatim}

在任务构造时其他属性设置后,把NestedElement实例加入到任务里:

\begin{verbatim}
public void addConfiguredInner(NestedElement anInner);
\end{verbatim}

6)编写public void execute()实现主逻辑。在这个方法中可以抛出BuildException。

\subsection{Ant与Java类型的转换}

当任务中定义的属性为不同的Java类型时,xml文件中的property的value要有对应的内容。

1)String:value只要是字符都可以。

2)boolean:true、false、yes、no。

3)char:只会收到第一个字符。

4)int、sort:自动转换,失败会抛出异常。

5)java.io.File:根据是否为绝对路径,project的basedir属性综合决定。

6)org.apache.tools.ant.types.Path:可以用“:”或“;”作为分隔符。

7)java.lang.Class:作为类名,调用Classloader加载。

8)其他Java类:如果这个类带一个String的构造函数,就用它来构造对象。

\subsection{例子}

实现一个输出信息的任务类:

\lstinputlisting[language=Java]{src/05/java/ch05/HelloWorldTask.java}

\lstinputlisting[language=XML]{src/05/build.11.xml}

\section{声明自定义任务TaskDef}

TaskDef在xml构建文件中调用类作为一个任务。

\subsection{主要属性}

name、classname、classpath、classpathref。

onerror:fail、report、ignore。

adapter:定义一个适配器(必须实现org.apache.tools.ant.TaskAdapter和org.apache.tools.ant.Task),这个适配器把已经指定的类转为另外一个接口或类的实例。

adaptto:指定一个任务要适配的类。如果任务没有实现这个类接口,会使用adapter属性指定的类对这个adaptto类进行封装。

\lstinputlisting[language=XML]{src/05/build.13.xml}

\subsection{例子}

\lstinputlisting[language=XML]{src/05/build.12.xml}











