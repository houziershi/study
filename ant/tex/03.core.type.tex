\chapter{核心类型}

\section{断言类型:Asseritions Type}

	断言类型指定哪些类需要让ant工具执行java断言。enableSystemAssertions属性设定是否允许系统断言,默认为unspecified。

	断言类型内还可以定义enable类型与disable类型,通过这两个类型的class属性与package属性来设置指定的类或是包是否可以执行断言。
	
	允许所有的用户代码执行断言:

	\lstinputlisting[language=XML,firstline=5,lastline=8]{src/03/build.01.xml}

	只允许Test这个类执行断言:

	\lstinputlisting[language=XML,firstline=9,lastline=11]{src/03/build.01.xml}

	允许一个包下所有的类都执行断言:

	\lstinputlisting[language=XML,firstline=13,lastline=15]{src/03/build.01.xml}

	更复杂的定义:

	\lstinputlisting[language=XML,firstline=17,lastline=21]{src/03/build.01.xml}

	引用一个已经存在的定义:

	\lstinputlisting[language=XML,firstline=23,lastline=27]{src/03/build.01.xml}

\section{匹配模式:PatternSet}

	\subsection{包含与排除}

		匹配格式通过以下四个属性指定包含与排除:

		include属性:可以指定多个模式用逗号或空格分隔,相当于一个列表。

		includesfile属性:指定具体包含的文件。可以引用在properties文件中指定的内容。

		exclude属性:和上面的相反。

		excludesfile属性:和上面的相反。

	\subsection{过滤的模式}

		以上四个属性都有三个属性来匹配指定的内容:

		name属性:文件名、路径或格式。

		if属性:指定的属性有定义则生效。

		unless属性:指定的属性没有定义则生效。


	\subsection{例子}

		类名中不包含Test的类:

		\lstinputlisting[language=XML,firstline=5,lastline=8]{src/03/build.02.xml}

		引用已经定义的模式:

		\lstinputlisting[language=XML,firstline=10,lastline=10]{src/03/build.02.xml}

		可以包含多个定义:

		\lstinputlisting[language=XML,firstline=12,lastline=16]{src/03/build.02.xml}

		指定文件名中包含了"some-file"的文件:

		\lstinputlisting[language=XML,firstline=18,lastline=20]{src/03/build.02.xml}

		也可以简写成下面的格式:

		\lstinputlisting[language=XML,firstline=22,lastline=22]{src/03/build.02.xml}

		配合if条件的形式:

		\lstinputlisting[language=XML,firstline=24,lastline=26]{src/03/build.02.xml}

\section{匹配目录:DirSet}

	DirSet类似于PatternSet,用于匹配目录。必填属性dir用于指定一个目录。
	
	DirSet可以引用已经存在的PatternSet,也可以直接使用include、includesfile、exclude、excludesfile属性。

	另外,DirSet还有casesensitive属性来设置是否大小写敏感;followsymlinks属性来设置采用操作系统差异(如linux与windows的路径符号)等。

	定义目录集合的例子:

	\lstinputlisting[language=XML,firstline=5,lastline=19]{src/03/build.03.xml}

\section{匹配目录:FileSet}

	\subsection{设置匹配}

		FileSet匹配符合的文件,它不仅包含DirSet的属性外,还有一些其他的属性:

		file属性:指定单个文件,fileSet定义中至少要dir属性或是file属性。

		defaultexcludes:当值为yes时默认忽略指定的文件(如版本控制文件),常用的模式有:

		\verb|**/*~, **/#*#, **/.#*, **/%*%, **/._*, |

		\verb|**/CVS, **/CVS/**, **/.cvsignore, |

		\verb|**/SCCS, **/SCCS/**, **/vssver.scc, |

		\verb|**/.svn, **/.svn/**, **/.DS_Store |

	\subsection{过滤筛选:selector}

		selector可以看作是FileSet中的一个元素。有以下几种常见的过滤工具:

	\subsection{内容过滤:contains}

		只选择包含text属性定义字符串的文件。

		text属性:文件包含的字符串,不能为空。

		casesenitive属性:大小写敏感。

		ignorewhitespace属性:忽略空白字符。

		\lstinputlisting[language=XML,firstline=21,lastline=23]{src/03/build.03.xml}

	\subsection{时间过滤:date}

		datetime属性:指定时间格式为:MM/DD/YYYY HH:MM AM 。例如:09/09/2006 10:10 am

		milis属性:指定亳秒时间。datetime与milis两个必选一个。

		when属性:文件修改时间的比较方式,默认为equals。其他值还有:before与after。

		granularity属性:允许的时间误差毫秒数。

		pattern属性:指定datettime是否兼容Java中的SimpleDateFormat格式。

		checkdirs属性:是否检查文件目录创建时间,默认为false。

		\lstinputlisting[language=XML,firstline=25,lastline=27]{src/03/build.03.xml}

	\subsection{比较过滤:depend}
		
		比较两个目录下同名的文件,选择最后修改的那个。

		targetdir属性:指定进行比较的目录。

		granularity属性:允许的时间误差毫秒数。

		比较1.4版与1.5版中有修改过的源文件。

		\lstinputlisting[language=XML,firstline=29,lastline=31]{src/03/build.03.xml}

	\subsection{深度过滤:depth}
		
		min最小与max最大。

		\lstinputlisting[language=XML,firstline=33,lastline=35]{src/03/build.03.xml}

	\subsection{差异过滤:different}

		比较目录下的文件中否有不同(当前目录和一层子目录)。比较的内容有:

		1)一个地方有另一个地方没有。
		
		2)文件大小不同。

		3)ignoreFileTimes不为off,且文件修改时间不同。

		4)ignoreContents为true,内容不同。

		属性有:targetdir、granularity(允许的时间误差范围)、ignoreFileTimes、ignoreContents。

		\lstinputlisting[language=XML,firstline=37,lastline=39]{src/03/build.03.xml}

	\subsection{文件名过滤:filename}

		主要属性:name(文件名匹配)、casesensitive、negate(反选,选中不匹配的)。

		例如,选择所有的css样式文件:

		\lstinputlisting[language=XML,firstline=41,lastline=43]{src/03/build.03.xml}
		
	\subsection{目录过滤:paresent}

		选择一个与当前目录对应的目录(targetdir),找出指定目录中不存在或是两个目录中都存在的文件。主要属性有:

		present属性:值为srconly本目录有面targetDir没有的文件;值为both选中两个目录都有的文件。

		\lstinputlisting[language=XML,firstline=45,lastline=47]{src/03/build.03.xml}

	\subsection{正则过滤:containsregexp}

		\lstinputlisting[language=XML,firstline=49,lastline=51]{src/03/build.03.xml}

	\subsection{大小过滤:size}

		value属性:值。

		units属性:单位。以1000为进制的K,M,G。或是以1024为单位的Ki,Mi,Gi。默认为bytes。

		when属性:more,less,equal。

		\lstinputlisting[language=XML,firstline=53,lastline=58]{src/03/build.03.xml}

	\subsection{类型过滤:type}
		
		指定是文件(file)还是目录(dir)。

		\lstinputlisting[language=XML,firstline=60,lastline=62]{src/03/build.03.xml}

\section{文件列表:FileList}

	\lstinputlisting[language=XML,firstline=64,lastline=72]{src/03/build.03.xml}

\section{文件过滤器:FilterSet}
	
	过滤器可以在对文件进行复制或移动时对文件内容进行替换操作。

	\subsection{主要属性}

		1)begintoken属性:用一个字符(默认为@)指定过滤字符串的开始。如:@DATE@

		2)endtoken属性:定义结束。

		3)recurse属性:是否查找多个替换标记,默认为true。

	\subsection{替换版权与日期的例子}

		\lstinputlisting[language=XML,firstline=5,lastline=19]{src/03/build.04.xml}
		
		可引用的版本:

		\lstinputlisting[language=XML,firstline=21,lastline=39]{src/03/build.04.xml}
		
	\subsection{把替换的内容放在属性文件中}

		要替换的文本:

		\lstinputlisting[language=XML,firstline=1,lastline=2]{src/03/src/src.txt}

		属性文件:

		\lstinputlisting[language=XML,firstline=1,lastline=1]{src/03/abc.properties}

		构建文件:

		\lstinputlisting[language=XML,firstline=41,lastline=52]{src/03/build.04.xml}
		
\section{属性集合:PropertySet}

	定义一套可以组其他标签引用的属性集合。

	\subsection{属性与功能}
		
		dynamic属性:是否动态地加载属性。

		negate属性:取反默认为false,如果为true代表返回除PropertySet外的属性。

	\subsection{引用已经存在的属性}

		Propertyref类型可以引用一个已经定义的property。主要属性有:

		name属性:引用的名字。

		prefix属性:引用指定开头的多个属性。

		regex属性:用正则匹配。

		builtin属性:引用ant内建的属性,值为all时表示所有内建属性。

	\subsection{使用属性集合的例子}

		组建一个属性集合:

		\lstinputlisting[language=XML,firstline=5,lastline=11]{src/03/build.05.xml}
		
		projectset间相互引用的例子:
		
		\lstinputlisting[language=XML,firstline=13,lastline=24]{src/03/build.05.xml}
		
\section{文件映射:mapper}

	用来定义文件之间的对应关系的,主要属性有:

	type属性:定义一个实现的类型,可以用现成的也可以自己实现一个。

	classname属性:通过类名指定一个实现类型。type和classname一定要选一个。

	classpath属性:查找类的路径。

	classpathref属性:引用已经有的path作为classpath。

	from属性:源文件的位置。

	to属性:目标文件的位置。

	对于文件映射,不同的实现类提供了不同的映射实现方法,以下各小节是现有的实现。每个类的调用都可以通过类名调用与type属性调用两种方法来写:

	\subsection{identity}

		源文件与目标文件同名。只取文件名,忽略路径:
		
		\lstinputlisting[language=XML,firstline=5,lastline=12]{src/03/build.06.xml}

	\subsection{flatten}

		忽略目录把文件放到一个压缩文件中:
		
		\lstinputlisting[language=XML,firstline=15,lastline=22]{src/03/build.06.xml}

	\subsection{glob}

		匹配路径与文件名:
		
		\lstinputlisting[language=XML,firstline=25,lastline=32]{src/03/build.06.xml}

	\subsection{merge}

		把源文件打包到压缩文件中:
		
		\lstinputlisting[language=XML,firstline=35,lastline=42]{src/03/build.06.xml}

	\subsection{regexp}

		通过正则来映射:
		
		\lstinputlisting[language=XML,firstline=45,lastline=52]{src/03/build.06.xml}

	\subsection{package}

		替换目录名称。
		
		\lstinputlisting[language=XML,firstline=55,lastline=62]{src/03/build.06.xml}

	\subsection{composite}

		多个mapper都对源文件进行操作。Composite Mapper不能通过Mapper类型的type属性定指定:
		
		\lstinputlisting[language=XML,firstline=65,lastline=72]{src/03/build.06.xml}

	\subsection{chained}

		包含多个mapper,源文件依次经过每个mapper操作。
		
		\lstinputlisting[language=XML,firstline=75,lastline=88]{src/03/build.06.xml}

	\subsection{filtermapper}

		对文件名进行过滤:
		
		\lstinputlisting[language=XML,firstline=91,lastline=96]{src/03/build.06.xml}
	
\section{压缩文件:zip}
	
	有两种压缩文件:

	1)当使用src属性时,目录下的文件会以.zip文件格式进行组织。

	2)当使用dir属性时,目录下的文件会以文件系统形式进行组织。

	\subsection{属性与功能}
		
		prefix属性:文件路径前缀,符合的会被选中。

		fullpath属性:包含全路径。

		src属性:用于替代当前的目录位置。

		filemode属性与dirmode属性:文件的权限,如linux下的权限777。

	\subsection{例子}

		以zip格式压缩htdocs/manual目录下的所有文件。存放到docs/user-guide目录下。

		同时添加ChangeLog27.txt文件到zip文件的docs目录下。

		这个zip文件还包含example.zip文件。example.zip文件包含docs/examples目录及其子目录下的所有html文件。
		
		\lstinputlisting[language=XML,firstline=5,lastline=12]{src/03/build.07.xml}

\section{过滤链与过滤读取器:FilterChains and FilterReader}

	一组有序的FilterReader组成FilterChains,用户可以实现自己的FilterReader。

	FilterReader通过classname属性指定实现类。

	在ant任务concat、copy、loadFile、loadProperties、move中都可以直接使用FilterChain进行过滤操作。
		
	\lstinputlisting[language=XML,firstline=5,lastline=21]{src/03/build.08.xml}

\section{定制与扩展}

	用户可以定制的有:conditions、selecters和filters类型。

	\subsection{定制条件判断:Condition}

		实现一个判断字符串是大写的条件判断:

		\lstinputlisting[language=Java]{src/03/src/exp/ant/AllUpperCaseCondition.java}

		构建文件中通过typedef导入,然后使用定义的类。
		
		\lstinputlisting[language=XML,firstline=5,lastline=11]{src/03/build.09.xml}

	\subsection{定制选择器:Selector}

		\lstinputlisting[language=Java]{src/03/src/exp/ant/JavaSelector.java}
		
		\lstinputlisting[language=XML,firstline=13,lastline=21]{src/03/build.09.xml}

		ant已经提供了一个BaseSelector做了一些预处理功能:setError(String errMsg)可提供出错信息;validate()方法会在isSelected()前进行验证。

	\subsection{定制过滤器:Filter}


		\lstinputlisting[language=Java]{src/03/src/exp/ant/RevomeOddCharacters.java}

		直接通过类名调用:
		
		\lstinputlisting[language=XML,firstline=23,lastline=27]{src/03/build.09.xml}
