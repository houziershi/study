\chapter{核心类型}

\section{断言类型:Asseritions Type}

	断言类型指定哪些类需要让ant工具执行java断言。enableSystemAssertions属性设定是否允许系统断言,默认为unspecified。

	断言类型内还可以定义enable类型与disable类型,通过这两个类型的class属性与package属性来设置指定的类或是包是否可以执行断言。
	
	允许所有的用户代码执行断言:

	\lstinputlisting[language=XML,firstline=5,lastline=8]{src/03/build.01.xml}

	只允许Test这个类执行断言:

	\lstinputlisting[language=XML,firstline=9,lastline=11]{src/03/build.01.xml}

	允许一个包下所有的类都执行断言:

	\lstinputlisting[language=XML,firstline=13,lastline=15]{src/03/build.01.xml}

	更复杂的定义:

	\lstinputlisting[language=XML,firstline=17,lastline=21]{src/03/build.01.xml}

	引用一个已经存在的定义:

	\lstinputlisting[language=XML,firstline=23,lastline=27]{src/03/build.01.xml}

\section{匹配模式:PatternSet}

	\subsection{包含与排除}

		匹配格式通过以下四个属性指定包含与排除:

		include属性:可以指定多个模式用逗号或空格分隔,相当于一个列表。

		includesfile属性:指定具体包含的文件。可以引用在properties文件中指定的内容。

		exclude属性:和上面的相反。

		excludesfile属性:和上面的相反。

	\subsection{过滤的模式}

		以上四个属性都有三个属性来匹配指定的内容:

		name属性:文件名、路径或格式。

		if属性:指定的属性有定义则生效。

		unless属性:指定的属性没有定义则生效。


	\subsection{例子}

		类名中不包含Test的类:

		\lstinputlisting[language=XML,firstline=5,lastline=8]{src/03/build.02.xml}

		引用已经定义的模式:

		\lstinputlisting[language=XML,firstline=10,lastline=10]{src/03/build.02.xml}

		可以包含多个定义:

		\lstinputlisting[language=XML,firstline=12,lastline=16]{src/03/build.02.xml}

		指定文件名中包含了"some-file"的文件:

		\lstinputlisting[language=XML,firstline=18,lastline=20]{src/03/build.02.xml}

		也可以简写成下面的格式:

		\lstinputlisting[language=XML,firstline=22,lastline=22]{src/03/build.02.xml}

		配合if条件的形式:

		\lstinputlisting[language=XML,firstline=24,lastline=26]{src/03/build.02.xml}

\section{匹配目录:DirSet}

	DirSet类似于PatternSet,用于匹配目录。DirSet可以引用已经存在的PatternSet,也可以直接使用include、includesfile、exclude、excludesfile属性。

	另外,DirSet还有casesensitive属性来设置是否大小写敏感;followsymlinks属性来设置采用操作系统差异(如linux与windows的路径符号)等。

	\subsection{包含与排除}

