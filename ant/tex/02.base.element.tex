
\chapter{起步}

\section{project元素}

	一个ant文件中一定要有一个project元素。name属性定义工作名字;default属性代表默认启动target;description属性定义说明文本。

	\lstinputlisting[language=XML]{src/02/build.01.xml}

\section{target元素}

	target元素一定要有name属性。

	\subsection{depends属性}

		表示当前target执行依赖于其他的target成功执行。可以有多个depends之间用“,”分隔。

		\lstinputlisting[language=XML]{src/02/build.01.xml}

	\subsection{if属性}

		验证执行前必须设定的属性。例如设定JDK:

		\lstinputlisting[language=XML]{src/02/build.03.xml}

	\subsection{unless属性}

		没有设置才执行

		\lstinputlisting[language=XML]{src/02/build.04.xml}

\section{property元素}

	property可以看作为参数或是变量的定义。可以在构建文件中通过property元素建立,也可以在构建文件外建立一个build.property文件来存放。

	\lstinputlisting[language=XML]{src/02/build.05.xml}

\section{task元素}
	
	task元素是一系列完成指定功能的脚本和程序。比如自带的编译用的javac、打包用的jar、建立目录的mkdir等。用户也可以编写自己的task。


\section{命令行使用ant}

	\subsection{指定文件}

		\verb|-buildfile|或\verb|-file|或\verb|-f|都可以。

	\subsection{显示任务描述}

		\verb|-projecthelp|。

	\subsection{指定变量的值}

		\verb|-D<property>=<value>|
