\chapter{核心任务}

\section{任务调用(Ant Task)}

有一种类型的Ant任务就叫“Ant任务”(Ant Task)。这种类型的Task可以去调用另一个Ant项目。

\subsection{主要属性}

指定要执行的文件(antfile属性):

\lstinputlisting[language=XML]{src/04/build.01.xml}

\lstinputlisting[language=XML]{src/04/subfile/projectB.xml}

指定文件所在目录(dir属性):

\lstinputlisting[language=XML]{src/04/build.02.xml}

调用指定的任务(target属性):

\lstinputlisting[language=XML]{src/04/build.03.xml}

\lstinputlisting[language=XML]{src/04/subfile/projectB2.xml}

指定输出流(output属性):

\lstinputlisting[language=XML]{src/04/build.04.xml}

被调用文件可以使用调用它文件中的属性(inheritAll属性):类似于Java中的继承属性,默认值为“true”。

被调用文件可以使用调用它文件中的reference任务(inheritRefs属性):reference任务的作用是把当前属性复制到被调用的任务中使用。它有两个可配置的属性:

1)refid属性:当前project中的属性id。
2)torefid属性:指定被调用的project中的引用id。

例:把当前project中的path1属性传递给被调用的project的属性path2使用:

\begin{lstlisting}[language=XML]
	<reference refid="path1" torefid="path2" />
\end{lstlisting}

\subsection{实例:一个任务整合了多个子任务}

实际工作中,一个大的项目会被独立为几个小的项目:

\lstinputlisting[language=XML]{src/04/build.06.xml}
\lstinputlisting[language=XML]{src/04/subfile/sub1.xml}
\lstinputlisting[language=XML]{src/04/subfile/sub2.xml}





\section{执行过程中调用其他target(AntCall Task)}

\subsection{主要属性}

三个主要属性:target、inheritAll、inheritRefs参照前一部分Ant Task。

\subsection{例子}

\lstinputlisting[language=XML]{src/04/build.07.xml}





\section{调用系统命令(Apply/ExecOn Task)}

\subsection{主要属性}

executable:指定要执行的命令,不带命令行参数。必填。

dest:执行命令的目标文件位置。

spawn:不输出日志,默认为false表示输出日志。

dir:在哪个目录下执行这个命令。

relative:是否支持相对路径。默认false不支持。

forwardslash:文件路径是否支持斜线分隔符。

os:支持这个命令的操作系统。

output:输出重定向。

error:错误输出重定向。

logError:错误输出重定向到ant的日志中去。

append:追加内容面不是覆盖已经有的文件。默认为false。

outputproperty:指定输出定向到的属性名字(定义一个文件则输出到文件中)。

errorproperty:错误重定向到属性的名字。

input:从指定文件中读取属性,可以以命令执行过程中引用。

inputstring:把指定的字符串传递给执行的命令。

resultproperty:执行后存放结果。

timeout:设定执行的超时时间。

failonerror:出错是中否中断。

failifexecutionfails:不能执行程序时中断。默认true。

skipemptyfilesets:如果目录中没有文件,则跳过执行。

parallel:如果为ture,则构建命令只执行一次,并把附加的文件作为命令参数。如果为false则每一个附加文件都会执行一次这个命令。默认为false。

type:说明参数类型:文件(file)、目录(dir)、路径(path)。默认为file。

newenvironment:如果当前环境变量被声明,则不传递旧的环境变量。默认为false。

vmlauncher:默认为true。通过java虚拟机的特性来执行构建文件;如果为false则通过操作系统本身的脚本来执行。

resolveExecutable:默认为false。如为true,命令会在project的根目录下执行。在UNIX或Linux下只允许用户在自己的路径下执行这个命令,要把这个属性设为false。

maxparallel:最大的平行值,指定一次执行源文件的最大数目。如果小于0表示没有限制(默认)。

addsourcefile:自动添加源文件名到执行命令中,默认为true。

verbose:输出命令执行时的概要信息,默认为false不输出。

ignoremissing:忽略不存在的文件,默认为true。

force:是否通过timestame来对target文件进行对比。默认为false。

\subsection{主要参数}

FileSet、FileList、DirSet、Arg(\verb|<arg>|指定参数)。

Mapper(可能指定dest属性的文件的映射关系)。

SrcFile(在\verb|<arg>|参数后使用,指定源文件)。

TargetFile(与srcFile作用相似,用于指定目录文件的参数)。

Env(环境变量)。

\subsection{例子}

调用“ls”命令,参数“-l”。分别排除和包含“properties”文件。

\lstinputlisting[language=XML]{src/04/build.08.xml}

\subsection{使用Mapper、SrcFile的例子}

以一个编译C源文件的例子:对于每一个比\verb|.o|文件更加新的\verb|.c|文件,执行:

\verb|cc -c -o targetfile sourcefile|

在这个文件中用\verb|.o|文件的名称替换targetfile,用\verb|.c|文件的名称替换sourcefile。

\begin{lstlisting}[language=Bash]
<apply executable="cc" dest="src/c" parallel="false">
	<arg value="-c" />
	<arg value="-c" />
	<targetfile />
	<srcfile />
	<fileset dir="src/c" includes="*.c" />
	<mapper type="glob" from="*.c" to="*.o" />
</apply>
\end{lstlisting}




\section{改变文件的权限(Chmod Task)}

\subsection{主要属性}

file、dir、include、excludes、perm(新的权限)。

parallel:是否为每个文件单独执行chmod。默认为true。

type:只改文件的权限(file);只改目录的权限(dir);都改权限(both)

maxparallel:最大的平行值,指定一次执行源文件的最大数目。如果小于0表示没有限制(默认)。

verbose:输出命令执行时的概要信息,默认为false不输出。

defaultexcludes:当值为yes时默认忽略指定的文件(如版本控制文件),常用的模式有:

\verb|**/*~, **/#*#, **/.#*, **/%*%, **/._*, |

\verb|**/CVS, **/CVS/**, **/.cvsignore, |

\verb|**/SCCS, **/SCCS/**, **/vssver.scc, |

\verb|**/.svn, **/.svn/**, **/.DS_Store |

\subsection{例子}

所有cgi或old结尾的文件,\verb|private_|开头的目录以及内部的文件。

\lstinputlisting[language=XML]{src/04/build.09.xml}






\section{删除文件(Delete Task)}

\subsection{主要属性}

file、dir、verbose、quiet(当文件不存在时,不显示提示信息)、failonerror、includes、includesfile、excludes、excludesfile(不推荐使用)、defaultexcludes(不推荐使用)。

deleteonexit:当文件存在时才删除。默认为false。

includeemptydirs:当使用FileSet类型时是否删除空的目录。

\subsection{例子}

\lstinputlisting[language=XML]{src/04/build.10.xml}






\section{输出信息(Echo Task)}

\subsection{主要属性}

message、file、append(追加到原有文件后面)、level(error、warning、info、verbose、debug)。

\subsection{例子}

\begin{lstlisting}[language=Bash]
<echo message="hello" />
<echo message="hello" file="logs/01.log" />
\end{lstlisting}






\section{创建目录(Mkdir Task)}

\begin{lstlisting}[language=Bash]
<property name="dist" value="dist" />
<property name="tmp" value="tmp" />
<mkdir dir="${dist}" />
<mkdir dir="${tmp}" />
\end{lstlisting}






\section{移动文件与目录(Move Task)}

\subsection{主要属性}

file、tofile、todir、overwrite、failonerror、verbose、

preservelastmodified:移动后文件的时间与源文件相同。

filtering:允许使用过滤符号。

flatten:没有目录结构,都在一级目录下。

includeEmptyDirs:忽略空目录。

encoding:过滤器的编码方式。

outputencoding:输出文件的编码。

granularity:允许文件修改时间的误差。默认0,DOS系统为2。

\subsection{例子}

\lstinputlisting[language=XML]{src/04/build.11.xml}






\section{压缩zip文件(Zip Task)}

\subsection{主要属性}

distfile、basedir、compress(是否压缩默认true)、encoding、fileonly、includes、includesfile、excludes、excludesfile、defaultexcludes、

update:覆盖目标文件。

whenempty:当没有可压缩的文件时结果为:报错(fail)、忽略(skip)、创建空zip文件(create)。

duplicate:文件重复时:默认为覆盖(add)、跳过(preserve)、报错(fail)。

roundup:文件修改时间是否采用一下个连续的秒数。

keepcompression:已经压缩的文件保持原先的压缩数据。

comment:zip文件的备注。

\subsection{例子}

任务t1直接按目录生成了压缩文件。任务t2指定了不同文件在生成的压缩文件中的位置。任务t3把一些其他的压缩文件放到了产生的压缩文件中。

\lstinputlisting[language=XML]{src/04/build.12.xml}






\section{加载属性文件(LoadProperties Task)}

\subsection{主要属性}

srcFile、resource(同srcFile)、encoding、classpath、classpathref。

\subsection{例子}

把复制的目标和来源都定义在属性文件中。

\lstinputlisting[language=XML]{src/04/copy.properties}

加载过程中只加载“copy”开头的属性。

\lstinputlisting[language=XML]{src/04/build.13.xml}






\section{定义日期格式(Tstamp Task)}

\subsection{主要属性}

property:定义名称,可以在以后引用。

pattern:格式。同java中的SimpleDateFormat。

timezone:时区。同java中的Timezone。

unit:设定与当时时间相差的单元,可为:millisecond、second、minute、hour、day、week、month、year。

offset:设定与当时时间差,单元由unit设定。

locale:地区设置。

\subsection{例子}

\lstinputlisting[language=XML]{src/04/build.14.xml}



