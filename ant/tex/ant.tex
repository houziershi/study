% 纸张与版面大小:电纸书版本
% \documentclass[10pt,a4paper]{report}
% \addtolength{\textheight}{-4cm}
% \addtolength{\textwidth}{+3cm}
%------------------------------------------------------------------------------------------------------
% 纸张与版面大小:普通A4纸版本
\documentclass[a4paper]{report}
\usepackage[margin=4cm]{geometry}      % 使用geometry包设置边距为2cm
%------------------------------------------------------------------------------------------------------
\usepackage{xltxtra,fontspec,xunicode}
\usepackage[slantfont,boldfont]{xeCJK} % 允许斜体和粗体

\setCJKmainfont{WenQuanYi Micro Hei}             % 缺省中文字体
\setCJKmonofont{WenQuanYi Micro Hei Mono}        % 中文等宽字体
\setmainfont{DejaVu Serif}                       % 英文衬线字体
\setsansfont{DejaVu Sans}                        % 英文无衬线字体
\setmonofont{Monaco}                             % 英文等宽字体
%-------------------------------------------------------------------------------------------------------
\linespread{1.3}                                 % 1.5倍行距,值1.6产生双倍行距
% \setlength{\parindent}{0pt}                    % 段落首行缩进
\setlength{\parskip}{1ex plus 0.5ex minus 0.2ex} % 段落间距为1ex,可让TeX在+0.5到-0.8范围内微调
                                                 % 即实际范围在0.8ex~1.5ex之间
%-------------------------------------------------------------------------------------------------------
% 页眉与页脚设置
\usepackage{fancyhdr}
\pagestyle{fancy}
\fancyhf{}                                       %清空页眉页脚
\fancyhead[LE,RO]{\thepage}                      %页眉偶数页左,奇数页右
\fancyhead[RE]{\leftmark}                        %页眉偶数页右
\fancyhead[LO]{\rightmark}                       %页眉奇数页左
% \fancyfoot[LE,RO]{\thepage}                    %页脚偶数页左,奇数页右
% \fancyfoot[RE]{\leftmark}                      %页脚偶数页右
% \fancyfoot[LO]{\rightmark}                     %页脚奇数页左
\fancypagestyle{plain}{                          %重定义plain页面样式
    \fancyhf{}
    \renewcommand{\headrulewidth}{0pt}
}
%-------------------------------------------------------------------------------------------------------
% listings 与 xcolor 配合实现源代码的语法高亮
\usepackage{xcolor}
\usepackage{listings}
\lstset{
	language=Java,
	frame = shadowbox,
	basicstyle = \ttfamily\small,
	columns = fixed,
	numbers = left,
	numberstyle = \footnotesize,
	stepnumber = 1,
	tabsize = 2,
	showspaces = false,
	showstringspaces = false,
	showtabs = false,
	captionpos = b,
	breaklines = tr[],
	breakatwhitespace = false,
	backgroundcolor = \color{white},
	keywordstyle=\color{blue},
	numberstyle=\color[RGB]{0,192,192},
	commentstyle=\color[RGB]{0,96,96},
	stringstyle=\ttfamily\slshape\color[RGB]{128,0,0},
	escapeinside=``
}
%-------------------------------------------------------------------------------------------------------
\usepackage{graphicx}                            % 引入图片
\graphicspath{{img/}{images/}}                   % 要导入的图片的位置。可以有多个目录,但就算只有一个目录,也要用两级花括号
%-------------------------------------------------------------------------------------------------------
	\title{Agile Java}                             % 文章的标题
	\author{                                     % 作者与致谢
		阿左 \thanks{感谢读者} \and 
		Nobody \thanks{感谢国家}
	}
	\date{\today}                                % 日期
	
%-------------------------------------------------------------------------------------------------------
\begin{document}
	\maketitle                                   % 制作标题
	\tableofcontents                             % 生成章节目录
	\setcounter{tocdepth}{5}                     % 生成章节的目录深度
	\listoffigures                               % 生成图片目录
	\listoftables                                % 生成表格目录



	\part{基础入门}

		
\chapter{开发环境}

\section{开发环境配置}


\section{编写第一个ant程序}




		
\chapter{起步}

\section{project元素}

	一个ant文件中一定要有一个project元素。name属性定义工作名字;default属性代表默认启动target;description属性定义说明文本。

	\lstinputlisting[language=XML]{src/02/build.01.xml}

\section{target元素}

	target元素一定要有name属性。

	\subsection{depends属性}

		表示当前target执行依赖于其他的target成功执行。可以有多个depends之间用“,”分隔。

		\lstinputlisting[language=XML]{src/02/build.01.xml}

	\subsection{if属性}

		验证执行前必须设定的属性。例如设定JDK:

		\lstinputlisting[language=XML]{src/02/build.03.xml}

	\subsection{unless属性}

		没有设置才执行

		\lstinputlisting[language=XML]{src/02/build.04.xml}

\section{property元素}

	property可以看作为参数或是变量的定义。可以在构建文件中通过property元素建立,也可以在构建文件外建立一个build.property文件来存放。

	\lstinputlisting[language=XML]{src/02/build.05.xml}

\section{task元素}
	
	task元素是一系列完成指定功能的脚本和程序。比如自带的编译用的javac、打包用的jar、建立目录的mkdir等。用户也可以编写自己的task。


\section{命令行使用ant}

	\subsection{指定文件}

		\verb|-buildfile|或\verb|-file|或\verb|-f|都可以。

	\subsection{显示任务描述}

		\verb|-projecthelp|。

	\subsection{指定变量的值}

		\verb|-D<property>=<value>|


		\chapter{核心类型}

\section{断言类型:Asseritions Type}

	断言类型指定哪些类需要让ant工具执行java断言。enableSystemAssertions属性设定是否允许系统断言,默认为unspecified。

	断言类型内还可以定义enable类型与disable类型,通过这两个类型的class属性与package属性来设置指定的类或是包是否可以执行断言。
	
	允许所有的用户代码执行断言:

	\lstinputlisting[language=XML,firstline=5,lastline=8]{src/03/build.01.xml}

	只允许Test这个类执行断言:

	\lstinputlisting[language=XML,firstline=9,lastline=11]{src/03/build.01.xml}

	允许一个包下所有的类都执行断言:

	\lstinputlisting[language=XML,firstline=13,lastline=15]{src/03/build.01.xml}

	更复杂的定义:

	\lstinputlisting[language=XML,firstline=17,lastline=21]{src/03/build.01.xml}

	引用一个已经存在的定义:

	\lstinputlisting[language=XML,firstline=23,lastline=27]{src/03/build.01.xml}

\section{匹配模式:PatternSet}

	\subsection{包含与排除}

		匹配格式通过以下四个属性指定包含与排除:

		include属性:可以指定多个模式用逗号或空格分隔,相当于一个列表。

		includesfile属性:指定具体包含的文件。可以引用在properties文件中指定的内容。

		exclude属性:和上面的相反。

		excludesfile属性:和上面的相反。

	\subsection{过滤的模式}

		以上四个属性都有三个属性来匹配指定的内容:

		name属性:文件名、路径或格式。

		if属性:指定的属性有定义则生效。

		unless属性:指定的属性没有定义则生效。


	\subsection{例子}

		类名中不包含Test的类:

		\lstinputlisting[language=XML,firstline=5,lastline=8]{src/03/build.02.xml}

		引用已经定义的模式:

		\lstinputlisting[language=XML,firstline=10,lastline=10]{src/03/build.02.xml}

		可以包含多个定义:

		\lstinputlisting[language=XML,firstline=12,lastline=16]{src/03/build.02.xml}

		指定文件名中包含了"some-file"的文件:

		\lstinputlisting[language=XML,firstline=18,lastline=20]{src/03/build.02.xml}

		也可以简写成下面的格式:

		\lstinputlisting[language=XML,firstline=22,lastline=22]{src/03/build.02.xml}

		配合if条件的形式:

		\lstinputlisting[language=XML,firstline=24,lastline=26]{src/03/build.02.xml}

\section{匹配目录:DirSet}

	DirSet类似于PatternSet,用于匹配目录。必填属性dir用于指定一个目录。
	
	DirSet可以引用已经存在的PatternSet,也可以直接使用include、includesfile、exclude、excludesfile属性。

	另外,DirSet还有casesensitive属性来设置是否大小写敏感;followsymlinks属性来设置采用操作系统差异(如linux与windows的路径符号)等。

	定义目录集合的例子:

	\lstinputlisting[language=XML,firstline=5,lastline=19]{src/03/build.03.xml}

\section{匹配目录:FileSet}

	\subsection{设置匹配}

		FileSet匹配符合的文件,它不仅包含DirSet的属性外,还有一些其他的属性:

		file属性:指定单个文件,fileSet定义中至少要dir属性或是file属性。

		defaultexcludes:当值为yes时默认忽略指定的文件(如版本控制文件),常用的模式有:

		\verb|**/*~, **/#*#, **/.#*, **/%*%, **/._*, |

		\verb|**/CVS, **/CVS/**, **/.cvsignore, |

		\verb|**/SCCS, **/SCCS/**, **/vssver.scc, |

		\verb|**/.svn, **/.svn/**, **/.DS_Store |

	\subsection{过滤筛选:selector}

		selector可以看作是FileSet中的一个元素。有以下几种常见的过滤工具:

	\subsection{内容过滤:contains}

		只选择包含text属性定义字符串的文件。

		text属性:文件包含的字符串,不能为空。

		casesenitive属性:大小写敏感。

		ignorewhitespace属性:忽略空白字符。

		\lstinputlisting[language=XML,firstline=21,lastline=23]{src/03/build.03.xml}

	\subsection{时间过滤:date}

		datetime属性:指定时间格式为:MM/DD/YYYY HH:MM AM 。例如:09/09/2006 10:10 am

		milis属性:指定亳秒时间。datetime与milis两个必选一个。

		when属性:文件修改时间的比较方式,默认为equals。其他值还有:before与after。

		granularity属性:允许的时间误差毫秒数。

		pattern属性:指定datettime是否兼容Java中的SimpleDateFormat格式。

		checkdirs属性:是否检查文件目录创建时间,默认为false。

		\lstinputlisting[language=XML,firstline=25,lastline=27]{src/03/build.03.xml}

	\subsection{比较过滤:depend}
		
		比较两个目录下同名的文件,选择最后修改的那个。

		targetdir属性:指定进行比较的目录。

		granularity属性:允许的时间误差毫秒数。

		比较1.4版与1.5版中有修改过的源文件。

		\lstinputlisting[language=XML,firstline=29,lastline=31]{src/03/build.03.xml}

	\subsection{深度过滤:depth}
		
		min最小与max最大。

		\lstinputlisting[language=XML,firstline=33,lastline=35]{src/03/build.03.xml}

	\subsection{差异过滤:different}

		比较目录下的文件中否有不同(当前目录和一层子目录)。比较的内容有:

		1)一个地方有另一个地方没有。
		
		2)文件大小不同。

		3)ignoreFileTimes不为off,且文件修改时间不同。

		4)ignoreContents为true,内容不同。

		属性有:targetdir、granularity(允许的时间误差范围)、ignoreFileTimes、ignoreContents。

		\lstinputlisting[language=XML,firstline=37,lastline=39]{src/03/build.03.xml}

	\subsection{文件名过滤:filename}

		主要属性:name(文件名匹配)、casesensitive、negate(反选,选中不匹配的)。

		例如,选择所有的css样式文件:

		\lstinputlisting[language=XML,firstline=41,lastline=43]{src/03/build.03.xml}
		
	\subsection{目录过滤:paresent}

		选择一个与当前目录对应的目录(targetdir),找出指定目录中不存在或是两个目录中都存在的文件。主要属性有:

		present属性:值为srconly本目录有面targetDir没有的文件;值为both选中两个目录都有的文件。

		\lstinputlisting[language=XML,firstline=45,lastline=47]{src/03/build.03.xml}

	\subsection{正则过滤:containsregexp}

		\lstinputlisting[language=XML,firstline=49,lastline=51]{src/03/build.03.xml}

	\subsection{大小过滤:size}

		value属性:值。

		units属性:单位。以1000为进制的K,M,G。或是以1024为单位的Ki,Mi,Gi。默认为bytes。

		when属性:more,less,equal。

		\lstinputlisting[language=XML,firstline=53,lastline=58]{src/03/build.03.xml}

	\subsection{类型过滤:type}
		
		指定是文件(file)还是目录(dir)。

		\lstinputlisting[language=XML,firstline=60,lastline=62]{src/03/build.03.xml}

\section{文件列表:FileList}

	\lstinputlisting[language=XML,firstline=64,lastline=72]{src/03/build.03.xml}

\section{文件过滤器:FilterSet}
	
	过滤器可以在对文件进行复制或移动时对文件内容进行替换操作。

	\subsection{主要属性}

		1)begintoken属性:用一个字符(默认为@)指定过滤字符串的开始。如:@DATE@

		2)endtoken属性:定义结束。

		3)recurse属性:是否查找多个替换标记,默认为true。

	\subsection{替换版权与日期的例子}

		\lstinputlisting[language=XML,firstline=5,lastline=19]{src/03/build.04.xml}
		
		可引用的版本:

		\lstinputlisting[language=XML,firstline=21,lastline=39]{src/03/build.04.xml}
		
	\subsection{把替换的内容放在属性文件中}

		要替换的文本:

		\lstinputlisting[language=XML,firstline=1,lastline=2]{src/03/src/src.txt}

		属性文件:

		\lstinputlisting[language=XML,firstline=1,lastline=1]{src/03/abc.properties}

		构建文件:

		\lstinputlisting[language=XML,firstline=41,lastline=52]{src/03/build.04.xml}
		
\section{属性集合:PropertySet}

	定义一套可以组其他标签引用的属性集合。

	\subsection{属性与功能}
		
		dynamic属性:是否动态地加载属性。

		negate属性:取反默认为false,如果为true代表返回除PropertySet外的属性。

	\subsection{引用已经存在的属性}

		Propertyref类型可以引用一个已经定义的property。主要属性有:

		name属性:引用的名字。

		prefix属性:引用指定开头的多个属性。

		regex属性:用正则匹配。

		builtin属性:引用ant内建的属性,值为all时表示所有内建属性。

	\subsection{使用属性集合的例子}

		组建一个属性集合:

		\lstinputlisting[language=XML,firstline=5,lastline=11]{src/03/build.05.xml}
		
		projectset间相互引用的例子:
		
		\lstinputlisting[language=XML,firstline=13,lastline=24]{src/03/build.05.xml}
		
\section{文件映射:mapper}

	用来定义文件之间的对应关系的,主要属性有:

	type属性:定义一个实现的类型,可以用现成的也可以自己实现一个。

	classname属性:通过类名指定一个实现类型。type和classname一定要选一个。

	classpath属性:查找类的路径。

	classpathref属性:引用已经有的path作为classpath。

	from属性:源文件的位置。

	to属性:目标文件的位置。

	对于文件映射,不同的实现类提供了不同的映射实现方法,以下各小节是现有的实现。每个类的调用都可以通过类名调用与type属性调用两种方法来写:

	\subsection{identity}

		源文件与目标文件同名。只取文件名,忽略路径:
		
		\lstinputlisting[language=XML,firstline=5,lastline=12]{src/03/build.06.xml}

	\subsection{flatten}

		忽略目录把文件放到一个压缩文件中:
		
		\lstinputlisting[language=XML,firstline=15,lastline=22]{src/03/build.06.xml}

	\subsection{glob}

		匹配路径与文件名:
		
		\lstinputlisting[language=XML,firstline=25,lastline=32]{src/03/build.06.xml}

	\subsection{merge}

		把源文件打包到压缩文件中:
		
		\lstinputlisting[language=XML,firstline=35,lastline=42]{src/03/build.06.xml}

	\subsection{regexp}

		通过正则来映射:
		
		\lstinputlisting[language=XML,firstline=45,lastline=52]{src/03/build.06.xml}

	\subsection{package}

		替换目录名称。
		
		\lstinputlisting[language=XML,firstline=55,lastline=62]{src/03/build.06.xml}

	\subsection{composite}

		多个mapper都对源文件进行操作。Composite Mapper不能通过Mapper类型的type属性定指定:
		
		\lstinputlisting[language=XML,firstline=65,lastline=72]{src/03/build.06.xml}

	\subsection{chained}

		包含多个mapper,源文件依次经过每个mapper操作。
		
		\lstinputlisting[language=XML,firstline=75,lastline=88]{src/03/build.06.xml}

	\subsection{filtermapper}

		对文件名进行过滤:
		
		\lstinputlisting[language=XML,firstline=91,lastline=96]{src/03/build.06.xml}
	
\section{压缩文件:zip}
	
	有两种压缩文件:

	1)当使用src属性时,目录下的文件会以.zip文件格式进行组织。

	2)当使用dir属性时,目录下的文件会以文件系统形式进行组织。

	\subsection{属性与功能}
		
		prefix属性:文件路径前缀,符合的会被选中。

		fullpath属性:包含全路径。

		src属性:用于替代当前的目录位置。

		filemode属性与dirmode属性:文件的权限,如linux下的权限777。

	\subsection{例子}

		以zip格式压缩htdocs/manual目录下的所有文件。存放到docs/user-guide目录下。

		同时添加ChangeLog27.txt文件到zip文件的docs目录下。

		这个zip文件还包含example.zip文件。example.zip文件包含docs/examples目录及其子目录下的所有html文件。
		
		\lstinputlisting[language=XML,firstline=5,lastline=12]{src/03/build.07.xml}

\section{过滤链与过滤读取器:FilterChains and FilterReader}

	一组有序的FilterReader组成FilterChains,用户可以实现自己的FilterReader。

	FilterReader通过classname属性指定实现类。

	在ant任务concat、copy、loadFile、loadProperties、move中都可以直接使用FilterChain进行过滤操作。
		
	\lstinputlisting[language=XML,firstline=5,lastline=21]{src/03/build.08.xml}

\section{定制与扩展}

	用户可以定制的有:conditions、selecters和filters类型。

	\subsection{定制条件判断:Condition}

		实现一个判断字符串是大写的条件判断:

		\lstinputlisting[language=Java]{src/03/src/exp/ant/AllUpperCaseCondition.java}

		构建文件中通过typedef导入,然后使用定义的类。
		
		\lstinputlisting[language=XML,firstline=5,lastline=11]{src/03/build.09.xml}

	\subsection{定制选择器:Selector}

		\lstinputlisting[language=Java]{src/03/src/exp/ant/JavaSelector.java}
		
		\lstinputlisting[language=XML,firstline=13,lastline=21]{src/03/build.09.xml}

		ant已经提供了一个BaseSelector做了一些预处理功能:setError(String errMsg)可提供出错信息;validate()方法会在isSelected()前进行验证。

	\subsection{定制过滤器:Filter}


		\lstinputlisting[language=Java]{src/03/src/exp/ant/RevomeOddCharacters.java}

		直接通过类名调用:
		
		\lstinputlisting[language=XML,firstline=23,lastline=27]{src/03/build.09.xml}


		\chapter{核心任务}

\section{任务调用(Ant Task)}

有一种类型的Ant任务就叫“Ant任务”(Ant Task)。这种类型的Task可以去调用另一个Ant项目。

\subsection{主要属性}

指定要执行的文件(antfile属性):

\lstinputlisting[language=XML]{src/04/build.01.xml}

\lstinputlisting[language=XML]{src/04/subfile/projectB.xml}

指定文件所在目录(dir属性):

\lstinputlisting[language=XML]{src/04/build.02.xml}

调用指定的任务(target属性):

\lstinputlisting[language=XML]{src/04/build.03.xml}

\lstinputlisting[language=XML]{src/04/subfile/projectB2.xml}

指定输出流(output属性):

\lstinputlisting[language=XML]{src/04/build.04.xml}

被调用文件可以使用调用它文件中的属性(inheritAll属性):类似于Java中的继承属性,默认值为“true”。

被调用文件可以使用调用它文件中的reference任务(inheritRefs属性):reference任务的作用是把当前属性复制到被调用的任务中使用。它有两个可配置的属性:

1)refid属性:当前project中的属性id。
2)torefid属性:指定被调用的project中的引用id。

例:把当前project中的path1属性传递给被调用的project的属性path2使用:

\begin{lstlisting}[language=XML]
	<reference refid="path1" torefid="path2" />
\end{lstlisting}

\subsection{实例:一个任务整合了多个子任务}

实际工作中,一个大的项目会被独立为几个小的项目:

\lstinputlisting[language=XML]{src/04/build.06.xml}
\lstinputlisting[language=XML]{src/04/subfile/sub1.xml}
\lstinputlisting[language=XML]{src/04/subfile/sub2.xml}





\section{执行过程中调用其他target(AntCall Task)}

\subsection{主要属性}

三个主要属性:target、inheritAll、inheritRefs参照前一部分Ant Task。

\subsection{例子}

\lstinputlisting[language=XML]{src/04/build.07.xml}





\section{调用系统命令(Apply/ExecOn Task)}

\subsection{主要属性}

executable:指定要执行的命令,不带命令行参数。必填。

dest:执行命令的目标文件位置。

spawn:不输出日志,默认为false表示输出日志。

dir:在哪个目录下执行这个命令。

relative:是否支持相对路径。默认false不支持。

forwardslash:文件路径是否支持斜线分隔符。

os:支持这个命令的操作系统。

output:输出重定向。

error:错误输出重定向。

logError:错误输出重定向到ant的日志中去。

append:追加内容面不是覆盖已经有的文件。默认为false。

outputproperty:指定输出定向到的属性名字(定义一个文件则输出到文件中)。

errorproperty:错误重定向到属性的名字。

input:从指定文件中读取属性,可以以命令执行过程中引用。

inputstring:把指定的字符串传递给执行的命令。

resultproperty:执行后存放结果。

timeout:设定执行的超时时间。

failonerror:出错是中否中断。

failifexecutionfails:不能执行程序时中断。默认true。

skipemptyfilesets:如果目录中没有文件,则跳过执行。

parallel:如果为ture,则构建命令只执行一次,并把附加的文件作为命令参数。如果为false则每一个附加文件都会执行一次这个命令。默认为false。

type:说明参数类型:文件(file)、目录(dir)、路径(path)。默认为file。

newenvironment:如果当前环境变量被声明,则不传递旧的环境变量。默认为false。

vmlauncher:默认为true。通过java虚拟机的特性来执行构建文件;如果为false则通过操作系统本身的脚本来执行。

resolveExecutable:默认为false。如为true,命令会在project的根目录下执行。在UNIX或Linux下只允许用户在自己的路径下执行这个命令,要把这个属性设为false。

maxparallel:最大的平行值,指定一次执行源文件的最大数目。如果小于0表示没有限制(默认)。

addsourcefile:自动添加源文件名到执行命令中,默认为true。

verbose:输出命令执行时的概要信息,默认为false不输出。

ignoremissing:忽略不存在的文件,默认为true。

force:是否通过timestame来对target文件进行对比。默认为false。

\subsection{主要参数}

FileSet、FileList、DirSet、Arg(\verb|<arg>|指定参数)。

Mapper(可能指定dest属性的文件的映射关系)。

SrcFile(在\verb|<arg>|参数后使用,指定源文件)。

TargetFile(与srcFile作用相似,用于指定目录文件的参数)。

Env(环境变量)。

\subsection{例子}

调用“ls”命令,参数“-l”。分别排除和包含“properties”文件。

\lstinputlisting[language=XML]{src/04/build.08.xml}

\subsection{使用Mapper、SrcFile的例子}

以一个编译C源文件的例子:对于每一个比\verb|.o|文件更加新的\verb|.c|文件,执行:

\verb|cc -c -o targetfile sourcefile|

在这个文件中用\verb|.o|文件的名称替换targetfile,用\verb|.c|文件的名称替换sourcefile。

\begin{lstlisting}[language=Bash]
<apply executable="cc" dest="src/c" parallel="false">
	<arg value="-c" />
	<arg value="-c" />
	<targetfile />
	<srcfile />
	<fileset dir="src/c" includes="*.c" />
	<mapper type="glob" from="*.c" to="*.o" />
</apply>
\end{lstlisting}




\section{改变文件的权限(Chmod Task)}

\subsection{主要属性}

file、dir、include、excludes、perm(新的权限)。

parallel:是否为每个文件单独执行chmod。默认为true。

type:只改文件的权限(file);只改目录的权限(dir);都改权限(both)

maxparallel:最大的平行值,指定一次执行源文件的最大数目。如果小于0表示没有限制(默认)。

verbose:输出命令执行时的概要信息,默认为false不输出。

defaultexcludes:当值为yes时默认忽略指定的文件(如版本控制文件),常用的模式有:

\verb|**/*~, **/#*#, **/.#*, **/%*%, **/._*, |

\verb|**/CVS, **/CVS/**, **/.cvsignore, |

\verb|**/SCCS, **/SCCS/**, **/vssver.scc, |

\verb|**/.svn, **/.svn/**, **/.DS_Store |

\subsection{例子}

所有cgi或old结尾的文件,\verb|private_|开头的目录以及内部的文件。

\lstinputlisting[language=XML]{src/04/build.09.xml}






\section{删除文件(Delete Task)}

\subsection{主要属性}

file、dir、verbose、quiet(当文件不存在时,不显示提示信息)、failonerror、includes、includesfile、excludes、excludesfile(不推荐使用)、defaultexcludes(不推荐使用)。

deleteonexit:当文件存在时才删除。默认为false。

includeemptydirs:当使用FileSet类型时是否删除空的目录。

\subsection{例子}

\lstinputlisting[language=XML]{src/04/build.10.xml}






\section{输出信息(Echo Task)}

\subsection{主要属性}

message、file、append(追加到原有文件后面)、level(error、warning、info、verbose、debug)。

\subsection{例子}

\begin{lstlisting}[language=Bash]
<echo message="hello" />
<echo message="hello" file="logs/01.log" />
\end{lstlisting}






\section{创建目录(Mkdir Task)}

\begin{lstlisting}[language=Bash]
<property name="dist" value="dist" />
<property name="tmp" value="tmp" />
<mkdir dir="${dist}" />
<mkdir dir="${tmp}" />
\end{lstlisting}






\section{移动文件与目录(Move Task)}

\subsection{主要属性}

file、tofile、todir、overwrite、failonerror、verbose、

preservelastmodified:移动后文件的时间与源文件相同。

filtering:允许使用过滤符号。

flatten:没有目录结构,都在一级目录下。

includeEmptyDirs:忽略空目录。

encoding:过滤器的编码方式。

outputencoding:输出文件的编码。

granularity:允许文件修改时间的误差。默认0,DOS系统为2。

\subsection{例子}

\lstinputlisting[language=XML]{src/04/build.11.xml}






\section{压缩zip文件(Zip Task)}

\subsection{主要属性}

distfile、basedir、compress(是否压缩默认true)、encoding、fileonly、includes、includesfile、excludes、excludesfile、defaultexcludes、

update:覆盖目标文件。

whenempty:当没有可压缩的文件时结果为:报错(fail)、忽略(skip)、创建空zip文件(create)。

duplicate:文件重复时:默认为覆盖(add)、跳过(preserve)、报错(fail)。

roundup:文件修改时间是否采用一下个连续的秒数。

keepcompression:已经压缩的文件保持原先的压缩数据。

comment:zip文件的备注。

\subsection{例子}

任务t1直接按目录生成了压缩文件。任务t2指定了不同文件在生成的压缩文件中的位置。任务t3把一些其他的压缩文件放到了产生的压缩文件中。

\lstinputlisting[language=XML]{src/04/build.12.xml}






\section{加载属性文件(LoadProperties Task)}

\subsection{主要属性}

srcFile、resource(同srcFile)、encoding、classpath、classpathref。

\subsection{例子}

把复制的目标和来源都定义在属性文件中。

\lstinputlisting[language=XML]{src/04/copy.properties}

加载过程中只加载“copy”开头的属性。

\lstinputlisting[language=XML]{src/04/build.13.xml}






\section{定义日期格式(Tstamp Task)}

\subsection{主要属性}

property:定义名称,可以在以后引用。

pattern:格式。同java中的SimpleDateFormat。

timezone:时区。同java中的Timezone。

unit:设定与当时时间相差的单元,可为:millisecond、second、minute、hour、day、week、month、year。

offset:设定与当时时间差,单元由unit设定。

locale:地区设置。

\subsection{例子}

\lstinputlisting[language=XML]{src/04/build.14.xml}





		\chapter{与java整合}

\section{通过java工具编译}

\begin{lstlisting}[language=Bash]
>>> javac ch05/HelloWorld.java 
>>> java  ch05.HelloWorld 
hello world
\end{lstlisting}






\section{通过ant构建Java程序}

java源代码都放在“java”目录下面:

\lstinputlisting[language=Java]{src/05/java/ch05/HelloWorld.java}

构建文件:

\lstinputlisting[language=XML]{src/05/build.01.xml}







\section{编译Java程序(javac)}

\subsection{主要属性}

srcdir、destdir、includes、includesfile、excludes、excludesfile、classpath、sourcepath、classpathref、sourcepathref、nowarn(不显示警告信息)、debug、debuglevel、optimize(优化等级)、verbose、failonerror。

source:当jvm支持-source参数时,指定。

compiler:指定编译器。

fork:编译时使用外部java编译器。默认为false。

excutable:当fork为true时,指定外部编译器的位置。

memoryInitialSize与memoryMaximunSize:当fork为true时使用的内存大小。

depend:编译时依赖跟踪(只有ibm-jikes和classic两种编译器才能用)。

includeantruntime:包含ant库。

includejavaruntime:包含java运行环境库。

deprecation:源文件是否带描述信息。

target:生成程序对应的jvm的版本。

extdirs:安装扩展文件的位置。

bootclasspath:启动class位置

bootcalsspathref:引用已经定义的\verb|<path>|元素。

\subsection{示例}

基本配置,都用默认值。

\lstinputlisting[language=XML,firstline=6,lastline=8]{src/05/build.02.xml}

指定源文件:

\lstinputlisting[language=XML,firstline=11,lastline=18]{src/05/build.02.xml}

使用外部编译器:

\lstinputlisting[language=XML,firstline=20,lastline=22]{src/05/build.02.xml}





\section{运行Java程序(java)}

\subsection{主要属性}

jar、args、classpath、classpathref、fork、jvm、jvmargs、maxmemory、failonerror。

classname:要执行的主类。

dir:java虚拟机的目录。

output、error:输出信息。

logError:输出错误到ant日志。

appended:覆盖输出文件。

resultproperty:指定一个已经定义的属性来记录命令执行时的返回值。

outputproperty、errorproperty:输出与标准错误的内容到指定属性。

input:定义标准输入。

inputstring:定义输入的字符串。

newenvironment:有了新的环境变量就不再传递旧的环境变量。

outlive:当fork为true时,允许打开另一个进程在ant工具外执行。

timeout:规定时间内没有完成就中止。

\subsection{示例}

基本配置,都用默认值。

\lstinputlisting[language=XML]{src/05/build.03.xml}






\section{打包jar(jar)}

\subsection{主要属性}

destfile、basedir、compress、keepcompression、encoding、fileonly、includes、includesfile、excludes、excludesfile、defaultexcludes、manifest、

filesetmanifest:当有多个manifest时,是忽略(skip)、合并(merge)、合并但是不包含manifest中的main函数定义(mergewithoutmain)。

update:是否覆盖已经存在jar中的目标文件。

whenempty:没有找到符合的文件打成jar包时,是报错(fail)、忽略(skip)、生成空文件(create)。

duplicate:重复时,覆盖(add)、保持原来的文件(preserve)、报错(fail)。

index:建立索引文件记录jar文件的内容。

manifestencoding:manifest文件的编码。

\subsection{定义MANIFEST.MF}

要用到manifest任务,主要属性有:

file:指定建立或更新manifest-file文件.

mode:是update还是replace。

encoding:编码。

attribute:属性和“键-值”对应。

section:区块,一个section可以包含多个attribute。

\lstinputlisting[language=XML]{src/05/build.04.xml}

\lstinputlisting[language=Bash]{src/05/MANIFEST.MF}

\subsection{打jar包的示例}

打包时选择要包含的文件:

\lstinputlisting[language=XML,firstline=4,lastline=7]{src/05/build.05.xml}

通过FileSet选择文件:

\lstinputlisting[language=XML,firstline=9,lastline=13]{src/05/build.05.xml}

\subsection{打jar包并运行}

\lstinputlisting[language=XML]{src/05/build.06.xml}







\section{打包WAR(war)}

启动tomcat:

\begin{lstlisting}[language=Bash]
>>> ./catalina.sh run
\end{lstlisting}


\subsection{主要属性}

destfile、webxml、basedir、compress、keepcompression、encoding、fileonly、includes、includesfile、excludes、excludesfile、defaultexcludes、manifest、update。

duplicate:覆盖重复的文件(add)、不覆盖(preserve)、报错(fail)。

\subsection{放置文件到war包对应位置}

\lstinputlisting[language=XML]{src/05/build.08.xml}

\subsection{例子}

\lstinputlisting[language=XML]{src/05/build.07.xml}








\section{打包EAR(ear)}


\subsection{主要属性}

destfile、appxml、basedir、compress、keepcompression、encoding、fileonly、includes、includesfile、excludes、excludesfile、defaultexcludes、manifest、update。

duplicate:覆盖重复的文件(add)、不覆盖(preserve)、报错(fail)。

\subsection{例子}

\lstinputlisting[language=XML]{src/05/build.09.xml}








\section{解压压缩文件}

\subsection{主要属性}

src、dest、overwrite、encoding。

compression:none(默认)、gzip、bzip2。

还可以使用FileSet和PatternSet。

\subsection{例子}

\lstinputlisting[language=XML]{src/05/build.09.xml}








\section{扩展Ant Task}

\subsection{方法}

可以通过继承不同的基类来实现扩展功能:

org.apache.tools.ant.Task:抽象类,所以ant任务的基类。

org.apache.tools.ant.taskdefs.AbstractCvsTask:访问CVS相关。

org.apache.tools.ant.taskdefs.MatchingTask:与include和excludes文件匹配模式相关。

org.apache.tools.ant.taskdefs.Pack:与Zip打包相关。

org.apache.tools.ant.taskdefs.Unpack:与Zip解包相关。

一般来说继承一个对应的类就可以实现相关的功能了。大概有以下六个步聚:

1)实现一个子类。

2)为每一个要用到的属性编写“set”方法(必须是public void)。在构建文件中通过对properties的引用来给属性赋值。

3)如果一个Task要包含其他的Task,那就一定要实现“org.apache.tools.ant.TaskContainer”接口。

4)如果需要支持字符输入(XML中的text形式),要添加“public void addText(String text)”方法。

5)如果需要嵌套Ant的元素或类型,要添加create、add、addConfigured方法。例如要嵌套自定义的<inner>元素,可能要用到的方法有:

建立一个NestedElement实例:

\begin{verbatim}
public NestedElement createInner();
\end{verbatim}

在任务构造时,把NestedElement实例加入到任务里:

\begin{verbatim}
public void addInner(NestedElement anInner);
\end{verbatim}

在任务构造时其他属性设置后,把NestedElement实例加入到任务里:

\begin{verbatim}
public void addConfiguredInner(NestedElement anInner);
\end{verbatim}

6)编写public void execute()实现主逻辑。在这个方法中可以抛出BuildException。

\subsection{Ant与Java类型的转换}

当任务中定义的属性为不同的Java类型时,xml文件中的property的value要有对应的内容。

1)String:value只要是字符都可以。

2)boolean:true、false、yes、no。

3)char:只会收到第一个字符。

4)int、sort:自动转换,失败会抛出异常。

5)java.io.File:根据是否为绝对路径,project的basedir属性综合决定。

6)org.apache.tools.ant.types.Path:可以用“:”或“;”作为分隔符。

7)java.lang.Class:作为类名,调用Classloader加载。

8)其他Java类:如果这个类带一个String的构造函数,就用它来构造对象。

\subsection{例子}

实现一个输出信息的任务类:

\lstinputlisting[language=Java]{src/05/java/ch05/HelloWorldTask.java}

\lstinputlisting[language=XML]{src/05/build.11.xml}

\section{声明自定义任务TaskDef}

TaskDef在xml构建文件中调用类作为一个任务。

\subsection{主要属性}

name、classname、classpath、classpathref。

onerror:fail、report、ignore。

adapter:定义一个适配器(必须实现org.apache.tools.ant.TaskAdapter和org.apache.tools.ant.Task),这个适配器把已经指定的类转为另外一个接口或类的实例。

adaptto:指定一个任务要适配的类。如果任务没有实现这个类接口,会使用adapter属性指定的类对这个adaptto类进行封装。

\lstinputlisting[language=XML]{src/05/build.13.xml}

\subsection{例子}

\lstinputlisting[language=XML]{src/05/build.12.xml}













	\part{工具整合}

		\chapter{整合数据库}

\section{Sql任务}

\subsection{主要属性}

driver、url、userid、password、src、encoding、delimiter(默认为“;”)、autocommit、classpath、classpathref、onerror、

rdbms:只有关系型数据库才能用。

print:显示数据结果

showheaders:显示结果字段名。

output:结果输出文件。

append:追加输出文件内容。

escapeprocessing:使用java statement对象不替换。

\subsection{执行sql的例子}

\lstinputlisting[language=XML]{src/06/build.01.xml}

\subsection{导入导出的例子}

mysql中可以通过简单地事务来导入数据;通过输出数据到文件来导出数据。

\lstinputlisting[language=XML]{src/06/build.02.xml}

mysqldump工具不能用sql执行,要用系统调用或是外部命令执行。











		\chapter{整合tomcat}


\lstinputlisting[language=Bash]{src/07/build.properties}

\lstinputlisting[language=XML]{src/07/build.01.xml}



	\part{实践}



\end{document}
